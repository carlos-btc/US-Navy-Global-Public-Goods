\documentclass[aspectratio=169]{beamer}
\usetheme{Madrid}
\usecolortheme{default}
\usepackage{graphicx}
\usepackage{booktabs}
\usepackage{siunitx}
\usepackage{xcolor}
\usepackage{amsmath,amssymb}
\usepackage{natbib}
\usepackage{caption}
\usepackage{subcaption}
\usepackage{gensymb}
% --- Preamble (add once) ---
\usepackage{tikz}
% Smaller citations
%\renewcommand{\bibsection}{}
%\bibliography{Bibliography/references}
%\usepackage[backend=biber,style=authoryear]{biblatex}
%\addbibresource{Bibliography\references.bib}
\setbeamertemplate{bibliography item}[text]
\bibliographystyle{apalike}

% Compact itemize
\setbeamertemplate{itemize items}[circle]

\title[Maritime Security \& Chokepoint Vulnerability]{Maritime Security as a Global Public Good:\\
Trade Frictions, Chokepoint Vulnerability, and a\\Scenario Engine Using AIS Shipping Density}
\author{Carlos Carpi}
\institute{MACS 30200}
\date{\today}

\begin{document}


% ============================================================
% Frame 1: Title
% ============================================================
\begin{frame}
  \titlepage
\end{frame}

% ============================================================
% Frame 2: Motivation
% ============================================================
\begin{frame}{Motivation}
  \begin{itemize}
    \item $\sim$80\% of world trade by volume traverses the oceans \citep{verschuur2023systemic}
    \item Trade funnels through a small number of physical bottlenecks: straits, canals, narrow sea lanes
    \item When security breaks down $\Rightarrow$ rerouting, congestion, elevated trade costs
    \item The U.S. Navy provides security at these chokepoints as a \textbf{global public good} \citep{bueger2024securing}
    \item \textbf{Key question:} What is the economic value of this security provision?
  \end{itemize}
\end{frame}

% ============================================================
% Frame 3: Research Question
% ============================================================
\begin{frame}{Research Question}
  \begin{block}{Central Question}
    How sensitive is the global maritime trading system to disruptions at critical chokepoints, and what is the economic value of security provision that prevents or mitigates such disruptions?
  \end{block}
  \vspace{8pt}
  \begin{itemize}
    \item Approach: scenario-based quantitative spatial framework
    \item Not causal estimation --- counterfactual simulation
    \item Builds on transportation networks literature \citep{fajgelbaum2020optimal, allen2022welfare}
    \item AIS data from IMF \citep{cerdeiro2020wstms} for empirical grounding
  \end{itemize}
\end{frame}

% ============================================================
% Frame 4: Literature --- Maritime Security
% ============================================================
\begin{frame}{Literature: Maritime Security as a Public Good}
  \begin{itemize}
    \item Maritime security exhibits classic public-good properties: \textbf{non-rivalry} and partial \textbf{non-excludability} \citep{bueger2019maritime, bueger2024securing}
    \item Legal foundations: freedom of navigation, passage through straits \citep{kraska2015law}
    \item Economic costs of insecurity documented:
    \begin{itemize}
      \item Piracy: ransoms, insurance, rerouting, deterred investment \citep{oef2010economic}
      \item African development impacts \citep{mbekeani2011piracy}
    \end{itemize}
    \item Gap: the \emph{value} of security provision remains poorly quantified
  \end{itemize}
\end{frame}


% ============================================================
% Frame 5: Literature --- Trade Frictions
% ============================================================
\begin{frame}{Literature: Trade Frictions \& Maritime Risk}
  \begin{itemize}
    \item \citet{besley2015welfare}: piracy acts as a \textbf{cost shifter} for international shipping
    \begin{itemize}
      \item Spatial/temporal variation in piracy risk $\Rightarrow$ shipping cost impacts
    \end{itemize}
    \item \citet{verschuur2023systemic}: chokepoint disruptions cause systemic GDP impacts
    \begin{itemize}
      \item Some scenarios: several \% of GDP for vulnerable economies
    \end{itemize}
    \item Port congestion literature: capacity constraints $\Rightarrow$ freight rate increases \citep{notteboom2006impact, du2020port}
    \item Maritime transport economics: market structure, pricing \citep{morabito2023thesis}
  \end{itemize}
\end{frame}

% ============================================================
% Frame 6: Literature --- Quantitative Spatial Economics
% ============================================================
\begin{frame}{Literature: Quantitative Spatial Economics}
  \begin{itemize}
    \item Foundational framework: \citet{redding2017quantitative}
    \begin{itemize}
      \item Locations differ in productivity; trade costs govern flows across space
    \end{itemize}
    \item Regional and urban extensions: \citet{allen2025quantitative}, \citet{redding2025urban}
    \item Key toolkit: iceberg trade costs, gravity trade flows, hat-algebra counterfactuals
    \item Agglomeration foundations: \citet{duranton2004micro}
    \item Geography of development: \citet{desmet2015geography}
    \item ``What can be learned'': \citet{proost2019what}
  \end{itemize}
\end{frame}

% ============================================================
% Frame 7: Literature --- Transportation Networks
% ============================================================
\begin{frame}{Literature: Transportation Networks \& Congestion}
  \begin{itemize}
    \item \citet{fajgelbaum2020optimal}: optimal transport networks in spatial equilibrium
    \begin{itemize}
      \item Endogenous route choice; optimal networks concentrate on high-traffic corridors
    \end{itemize}
    \item \citet{allen2022welfare}: welfare effects of infrastructure with endogenous congestion
    \item \citet{donaldson2025transport}: comprehensive survey of infrastructure evaluation
    \item Congestion pricing: \citet{hierons2024spreading}
    \item Political fragmentation: \citet{bordeu2025commuting}
    \item Network model structure: \citet{rossihansberg2020transportation}
  \end{itemize}
\end{frame}

% ============================================================
% Frame 8: Literature --- AIS Data & Gap
% ============================================================
\begin{frame}{Literature: AIS Data \& Research Gap}
  \begin{itemize}
    \item AIS data: cornerstone of modern maritime economics \citep{kerbl2022ais}
    \item COVID shock on container flows: \citet{cariou2021ais}
    \item Nowcasting world trade from AIS: \citet{cerdeiro2020nowcasting}
    \item IMF World Seaborne Trade Monitoring System: \citet{cerdeiro2020wstms}
  \end{itemize}
  \vspace{6pt}
  \begin{alertblock}{Gap}
    No existing work combines a global maritime network model with endogenous rerouting and congestion to value security provision under counterfactual disruptions.
  \end{alertblock}
\end{frame}

% ============================================================
% Frame 9: Data --- AIS Raster
% ============================================================
\begin{frame}{Data: AIS Ship Density Raster}
  \begin{itemize}
    \item Source: IMF World Seaborne Trade Monitoring System \citep{cerdeiro2020wstms}
    \item Period: January 2015 -- February 2021
    \item Resolution: $0.005\degree \times 0.005\degree$ ($\approx$ 500m $\times$ 500m at equator)
    \item Each cell: total count of AIS position reports
    \item Format: single GeoTIFF ($\sim$9 GB) + overview pyramids ($\sim$3 GB)
    \item Processing: overviews for visualization, windowed reads for chokepoints
  \end{itemize}
  \vspace{4pt}
  \begin{block}{HIGLY Skewed Data}
    Median cell = 0, P90 = 1, P99 $\approx$ 8.6M, Max $\approx$ 50.6M
  \end{block}
\end{frame}

% ============================================================
% Frame 10: Data --- Global Density Map
% ============================================================
\begin{frame}{Global Shipping Density (Log Scale)}
  \centering
  \includegraphics[width=0.80\textwidth]{Figures/global_quicklook.png}
  \vspace{2pt}
  {\small \\Extremely right-skewed: most ocean is empty; traffic concentrates at sea lanes and chokepoints.}
\end{frame}

% ============================================================
% Frame 11: Chokepoint Bounding Box Example
% ============================================================
\begin{frame}{Chokepoint Definitions: Bounding Box Approach}
  \begin{columns}[T]
    \begin{column}{0.45\textwidth}
      \small
      Six chokepoints defined as geographic bounding boxes:
      \begin{enumerate}
        \footnotesize
        \item \textbf{Suez Canal}
        \item Bab el-Mandeb
        \item Strait of Malacca
        \item Panama Canal
        \item Bosporus
        \item Gibraltar
      \end{enumerate}
      \vspace{4pt}
      Selected per \citet{verschuur2023systemic, bueger2024securing}.
    \end{column}
    \begin{column}{0.52\textwidth}
      \centering
      \includegraphics[width=0.7\textwidth]{Figures/bounding_box_example.png}
      {\tiny \\Suez Canal: AIS density with bounding box}
    \end{column}
  \end{columns}
\end{frame}

% ============================================================
% Frame 12: Chokepoint Rankings
% ============================================================
\begin{frame}{Chokepoint Traffic Intensity}
  \centering
  \includegraphics[width=0.85\textwidth]{Figures/chokepoint_ranking.png}
  \vspace{2pt}
  {\small \\ Traffic intensity ranking across six chokepoints from AIS density data.}
\end{frame}

% ============================================================
% Frame 13: Suez Density Detail
% ============================================================
\begin{frame}{AIS Density Detail: Suez Canal}
  \centering
  \includegraphics[width=0.33\textwidth]{Figures/suez_intensity_detail.png}
  \vspace{2pt}
  {\small Traffic concentrates along the canal corridor and approach channels from Port Said and Suez.}
\end{frame}

% ============================================================
% Frame 14: Suez Descriptive Statistics
% ============================================================
\begin{frame}{Suez Canal: Descriptive Statistics}
  \centering
  \includegraphics[width=0.51\textwidth]{Figures/suez_density_stats.png}
\end{frame}

% ============================================================
% Frame 15: Methods --- Network Construction
% ============================================================
\begin{frame}{Methods: Maritime Transport Network}
  \begin{columns}[T]
    \begin{column}{0.42\textwidth}
      \small
      $G = (\mathcal{J}, \mathcal{E})$: 78 nodes, $\sim$155 edges
      \begin{itemize}
        \item \textbf{56 ports}: major world ports at real coordinates
        \item \textbf{6 chokepoints}: maritime bottlenecks
        \item \textbf{16 waypoints}: ocean routing nodes
      \end{itemize}
      \vspace{4pt}
      Key design: \textbf{bypass-dependent}
      \begin{itemize}
        \footnotesize
        \item Black Sea $\rightarrow$ Bosporus (bypass: rail at 5$\times$)
        \item Malacca bypass: Lombok Strait
        \item Suez bypass: Cape of Good Hope
      \end{itemize}
      \vspace{4pt}
      1,540 port-to-port OD pairs
    \end{column}
    \begin{column}{0.56\textwidth}
      \centering
      \includegraphics[width=\textwidth]{Figures/network_world_map.png}
      {\tiny 78-node maritime network with bypass routes}
    \end{column}
  \end{columns}
\end{frame}

% ============================================================
% Frame 16: Methods --- Edge Cost
% ============================================================
\begin{frame}{Methods: Edge Cost Specification}
  \begin{columns}[T]
    \begin{column}{0.45\textwidth}
      Generalized cost of edge $e$:
      \begin{equation*}
        t_e = \bar{t}_e \cdot \left(\Xi_e\right)^{\lambda}
      \end{equation*}
      \begin{equation*}
        \bar{t}_e = \text{dist}_e \cdot m_e
      \end{equation*}
      \begin{itemize}
        \small
        \item $\Xi_e$: AIS congestion proxy
        \item $\lambda \geq 0$: congestion elasticity
        \item $m_e$: security/risk multiplier
      \end{itemize}
      \citep{allen2022welfare, hierons2024spreading}
    \end{column}
    \begin{column}{0.52\textwidth}
      \centering
      \includegraphics[width=\textwidth]{Figures/edge_cost_diagram.png}
      {\tiny Suez--Red Sea corridor with edge distances}
    \end{column}
  \end{columns}
\end{frame}

% ============================================================
% Frame 17: Methods --- Security Cost Shifter
% ============================================================
\begin{frame}{Methods: Security as a Cost Shifter}
  \begin{equation*}
    m_e = m_e^{\text{base}} \cdot (1 + \delta_{\text{risk}} \cdot \text{Risk}_e) \cdot (1 - \delta_{\text{sec}} \cdot S_e)
  \end{equation*}
  \centering
  \includegraphics[width=0.62\textwidth]{Figures/security_shifter_comparison.png}
  \vspace{2pt}
  {\small \\ Left: baseline costs. Right: +100\% risk premium doubles edge costs at Bab el-Mandeb \citep{besley2015welfare}.}
\end{frame}

% ============================================================
% Frame 18: Methods --- Congestion Calibration
% ============================================================
\begin{frame}{Methods: Congestion Calibration from AIS}
  \centering
  \includegraphics[width=0.82\textwidth]{Figures/congestion_calibration.png}
  \vspace{2pt}
  {\small AIS intensity $\rightarrow$ congestion multiplier $m_b \in [1, 1.5]$ via log-scaling \citep{notteboom2006impact, du2020port}.}
\end{frame}

% ============================================================
% Frame 19: Methods --- Route Choice
% ============================================================
\begin{frame}{Methods: Route Choice Before/After}
  \centering
  \includegraphics[width=0.92\textwidth]{Figures/route_choice_before_after.png}
  \vspace{2pt}
  {\small Shanghai $\rightarrow$ Rotterdam: Suez closure forces rerouting via North Pacific $\rightarrow$ Panama \citep{cariou2021ais}.}
\end{frame}

% ============================================================
% Frame 20: Methods --- Scenario Design
% ============================================================
\begin{frame}{Methods: Three Scenario Types}
  \begin{enumerate}
    \item \textbf{Full closure}: Remove chokepoint $b$; recompute all 1,540 shortest paths
    \item \textbf{Partial degradation}: Multiply edge costs by $\alpha \in \{1.25, \ldots, 10.0\}$
    \item \textbf{Security risk premium}: Apply $\delta_{\text{risk}} \in \{10\%, \ldots, 200\%\}$ per chokepoint
  \end{enumerate}
  \vspace{4pt}
  \centering
  \includegraphics[width=0.82\textwidth]{Figures/suez_scenario_3panel.png}
  {\tiny Suez Canal: 3 scenarios illustrated}
\end{frame}

% ============================================================
% Frame 21: Results --- Full Closure Ranking
% ============================================================
\begin{frame}{Results: Full Closure --- Vulnerability Ranking}
  \centering
  \small
  \begin{tabular}{lrrrc}
    \toprule
    Chokepoint Removed & Mean $\Delta$ (km) & Affected & Disconn. & Total \\
    \midrule
    Panama Canal      & 17,840 & 298 & 0 & 1,540 \\
    Gibraltar         & 10,624 & 590 & 0 & 1,540 \\
    Suez Canal        & 9,949  & 469 & 0 & 1,540 \\
    Bab el-Mandeb     & 4,700  & 454 & 0 & 1,540 \\
    Bosporus          & 3,275  & 159 & 0 & 1,540 \\
    Strait of Malacca & 1,754  & 290 & 0 & 1,540 \\
    \bottomrule
  \end{tabular}
  \vspace{4pt}
  {\small 0 disconnected pairs: bypass routes (overland, Lombok, Cape of Good Hope) ensure all ports remain connected.}
\end{frame}

% ============================================================
% Frame 22: Results --- Closure Summary Chart
% ============================================================
\begin{frame}{Results: Full Closure --- Visual Summary}
  \centering
  \includegraphics[width=0.92\textwidth]{Figures/scenario_summary.png}
  \vspace{2pt}
  {\small Left: mean rerouting cost. Right: fraction of 1,540 pairs affected by each closure.}
\end{frame}

% ============================================================
% Frame 23: Results --- Suez Closure Detail
% ============================================================
\begin{frame}{Results: Suez Canal Closure}
  \centering
  \includegraphics[width=0.72\textwidth]{Figures/scenario_example.png}
  \vspace{2pt}
  {\small \\ 469 pairs affected. Asia--Europe traffic rerouted via Cape of Good Hope or Panama Canal.\\Mean $\Delta$ = 9,949 km, Max $\Delta$ = 23,317 km. 0 pairs disconnected.}
\end{frame}

% ============================================================
% Frame 24: Results --- Closure Impact World Map
% ============================================================
\begin{frame}{Results: Port Vulnerability Under Closure}
  \centering
  \includegraphics[width=0.92\textwidth]{Figures/closure_impact_map.png}
  \vspace{2pt}
  {\small \\ Port color = mean \% cost increase. X = closed chokepoint. Panama, Gibraltar, Suez shown.}
\end{frame}

% ============================================================
% Frame 25: Results --- Partial Degradation Surface
% ============================================================
\begin{frame}{Results: Partial Degradation Surface}
  \centering
  \includegraphics[width=0.45\textwidth]{Figures/partial_degradation_surface.png}
  \vspace{2pt}
  {\small At $\alpha=5$: Gibraltar +26.5\%, Suez +12.7\%, Bab el-Mandeb +10.1\%, Bosporus +4.5\%.}
\end{frame}

% ============================================================
% Frame 26: Results --- Port Vulnerability Heatmap
% ============================================================
\begin{frame}{Results: Port Vulnerability Matrix}
  \centering
  \includegraphics[width=0.33\textwidth]{Figures/port_vulnerability_heatmap.png}
  \vspace{2pt}
  {\small 56 ports $\times$ 6 chokepoints. Color = mean \% cost increase under full closure.}
\end{frame}



% Normalized crop helper:
% (x,y) are in [0,1] with (0,0)=bottom-left of the ORIGINAL image.
\newcommand{\heatzoom}[5][]{%
  \begin{tikzpicture}
    % invisible reference node (sets corners at the final rendered size)
    \node[anchor=south west,inner sep=0,opacity=0] (img) at (0,0)
      {\includegraphics[#1]{Figures/port_vulnerability_heatmap.png}};
    % draw the same image, but clipped to the requested rectangle
    \begin{scope}[x={(img.south east)},y={(img.north west)}]
      \clip (#2,#3) rectangle (#4,#5);
      \node[anchor=south west,inner sep=0] at (0,0)
        {\includegraphics[#1]{Figures/port_vulnerability_heatmap.png}};
    \end{scope}
  \end{tikzpicture}%
}

% ============================================================
% Frame 26: Results --- Port Vulnerability Heatmap (Hotspots)
% ============================================================
\begin{frame}{Results: Port Vulnerability Matrix --- Hotspots (Zoom-ins)}
\begin{columns}[T,onlytextwidth]
  \column{0.50\textwidth}
  \centering
  {\small\bfseries Panama Canal (highest cells)}\\[-2pt]
  \heatzoom[width=\linewidth,height=0.26\textheight,keepaspectratio]{0.000}{0.664}{0.229}{0.966}\\[-2pt]
  {\tiny Colon; Balboa; Cartagena; Callao; Norfolk; New York; Savannah; Houston; Miami; Manzanillo MX.}\\[0.8ex]

  {\small\bfseries Gibraltar (highest cells)}\\[-2pt]
  \heatzoom[width=\linewidth,height=0.26\textheight,keepaspectratio]{0.000}{0.541}{0.357}{0.927}\\[-2pt]
  {\tiny Le Havre; Tangier Med; Algeciras; Barcelona; Genoa; Felixstowe; Antwerp; Rotterdam; Bremerhaven; Hamburg.}

  \column{0.50\textwidth}
  \centering
  {\small\bfseries Suez Canal (highest cells)}\\[-2pt]
  \heatzoom[width=\linewidth,height=0.34\textheight,keepaspectratio]{0.000}{0.590}{0.485}{0.897}\\[-2pt]
  {\tiny Port Said (peak); Jeddah (very high). Secondary (visually lighter): Piraeus; Djibouti.}\\[0.8ex]

  {\small\bfseries Bosporus (highest cell)}\\[-2pt]
  \heatzoom[width=\linewidth,height=0.34\textheight,keepaspectratio]{0.000}{0.761}{0.613}{0.878}\\[-2pt]
  {\tiny Piraeus is the dominant outlier under Bosporus closure.}
\end{columns}

\vspace{2pt}
{\tiny 56 ports $\times$ 6 chokepoints. Color = mean \% cost increase under full closure (averaged across all routes to/from each port).}
\end{frame}



% ============================================================
% Frame 27: Results --- Security Scenarios
% ============================================================
\begin{frame}{Results: Per-Chokepoint Security Risk}
  \centering
  \includegraphics[width=0.7\textwidth]{Figures/security_scenario_comparison.png}
  \vspace{2pt}
  {\small \\Pure scenario: risk premium applied to \emph{one chokepoint at a time}. No ``high/low'' categorization.}
\end{frame}

% ============================================================
% Frame 28: Results --- Cost Per Day
% ============================================================
\begin{frame}{Results: Daily Rerouting Cost of Closure}
  \centering
  \small
  \begin{tabular}{lrrrr}
    \toprule
    Chokepoint & Daily Vessels & Detour (km) & Cost/Day (Low) & Cost/Day (High) \\
    \midrule
    Gibraltar    & 219 & 10,624 & \$116M & \$233M \\
    Panama       & 37  & 17,840 & \$33M  & \$66M  \\
    Suez Canal   & 53  & 9,949  & \$27M  & \$53M  \\
    Malacca      & 233 & 1,754  & \$20M  & \$41M  \\
    Bosporus     & 118 & 3,275  & \$19M  & \$39M  \\
    Bab el-Mandeb & 68 & 4,700  & \$16M  & \$32M  \\
    \bottomrule
  \end{tabular}
  \vspace{4pt}
  {\small \\ Low = \$50/km; High = \$100/km. Aggregate: \$230--460M/day across all chokepoints.}
\end{frame}

% ============================================================
% Frame 29: Discussion --- Hegemonic Dividend
% ============================================================
\begin{frame}{The Hegemonic Dividend}
  \begin{itemize}
    \item \textbf{Daily costs}: Gibraltar closure alone $\Rightarrow$ \$117--233M/day in rerouting costs
    \item \textbf{Aggregate}: simultaneous closure of all six chokepoints $\Rightarrow$ \$85--170B/year
    \item The ``hegemonic dividend'' = avoided cost from maintaining security
    \item Dividend is \textbf{increasing in risk}: value of security rises as conditions worsen
    \item Heterogeneous benefits: bypass-dependent ports (Black Sea) benefit most
  \end{itemize}
  \vspace{4pt}
  \citep{bueger2024securing, besley2015welfare, oef2010economic}
\end{frame}

% ============================================================
% Frame 30: Discussion --- Structural Patterns
% ============================================================
\begin{frame}{Structural Patterns}
  \begin{itemize}
    \item \textbf{Two types of chokepoints}:
    \begin{itemize}
      \item \emph{Through-corridor} (Malacca, Gibraltar, Suez, Panama): rerouting costs, gradual impact
      \item \emph{Bypass-dependent} (Bosporus): expensive overland rerouting at 5$\times$ cost
    \end{itemize}
    \item \textbf{Policy implications differ}: capacity expansion vs.\ redundancy investment
    \item \textbf{Nonlinear severity}: severity curves accelerate at high $\alpha$ for corridor chokepoints
    \item \textbf{Supply chain resilience}: Panama Canal closure produces largest mean detour (17,840 km)
    \item \textbf{Connects to:} optimal networks \citep{fajgelbaum2020optimal}, congestion \citep{hierons2024spreading}, spatial policy \citep{fajgelbaum2020spatial}
  \end{itemize}
\end{frame}

% ============================================================
% Frame 31: Limitations
% ============================================================
\begin{frame}{Limitations}
  \begin{itemize}
    \item \textbf{Static aggregate}: single 2015--2021 snapshot; no temporal dynamics
    \item \textbf{Aggregate density}: AIS counts, not shipped value; no vessel-type distinction \citep{kerbl2022ais}
    \item \textbf{No welfare computation}: delta costs, not GE welfare changes \citep{redding2017quantitative}
    \item \textbf{Exogenous security}: risk premiums are sensitivity inputs, not calibrated
    \item \textbf{Stylized network}: 78 nodes captures major routes but not all alternatives
    \item All results labeled as \textbf{``partial-exercise'' estimates}
  \end{itemize}
\end{frame}

% ============================================================
% Frame 32: Policy Implications
% ============================================================
\begin{frame}{Policy Implications}
  \begin{enumerate}
    \item \textbf{Resource allocation}: prioritize through-corridor chokepoints (Gibraltar, Suez, Bab el-Mandeb) for capacity maintenance \citep{fajgelbaum2020spatial}
    \item \textbf{Redundancy investment}: bypass-dependent chokepoints need alternative infrastructure (overland routes)
    \item \textbf{Burden-sharing}: economic case for security strengthens as risk rises \citep{oef2010economic, mbekeani2011piracy}
    \item \textbf{Supply chain diversification}: corridor-dependent ports should develop alternative routing \citep{bordeu2025commuting}
    \item \textbf{Optimal security allocation}: spatial policy framework \citep{fajgelbaum2025optimal, gaubert2025place}
  \end{enumerate}
\end{frame}

% ============================================================
% Frame 33: Future Research
% ============================================================
\begin{frame}{Future Research Directions}
  \begin{enumerate}
    \item \textbf{Vessel-level AIS}: OD flow matrices, route-choice models \citep{kerbl2022ais, fajgelbaum2020optimal, fajgelbaum2020supplement}
    \item \textbf{Bilateral trade data}: calibrate full spatial equilibrium \citep{allen2025quantitative, redding2025urban}
    \item \textbf{Insurance \& risk data}: calibrate security parameters \citep{besley2015welfare}
    \item \textbf{Naval presence proxies}: move from simulation to causal inference \citep{proost2019what}
    \item \textbf{Dynamic extensions}: stochastic disruptions \citep{bordeu2025commuting, desmet2015geography}
    \item \textbf{Optimal allocation}: security budget across chokepoints \citep{rossihansberg2004optimal, fajgelbaum2019state}
  \end{enumerate}
\end{frame}

% ============================================================
% Frame 34: Conclusion
% ============================================================
\begin{frame}{Conclusion}
  \begin{block}{Takeaways}
    \begin{enumerate}
      \item Chokepoint disruptions produce \textbf{large, heterogeneous} trade-cost impacts
      \begin{itemize}
        \item Panama: 17,840 km mean rerouting; Suez: 9,949 km; Gibraltar: 10,624 km
      \end{itemize}
      \item Two vulnerability types: \textbf{rerouting} (through-corridor) vs \textbf{expensive bypass} (bypass-dependent)
      \item The \textbf{hegemonic dividend} is increasing in the risk environment
      \item Daily closure costs: \textbf{\$16M--\$233M/day} depending on chokepoint
      \item This is a \textbf{replicable framework} identifying data needs for full causal assessment
    \end{enumerate}
  \end{block}
  \vspace{4pt}
  \centering
  {\small \citet{bueger2024securing, fajgelbaum2020optimal, besley2015welfare, verschuur2023systemic}}
\end{frame}


% ============================================================
% Frame 35: Appendix Overview
% ============================================================
\begin{frame}{Appendix: All Chokepoint Bounding Boxes}
  \centering
  \includegraphics[width=0.8\textwidth]{Figures/appendix_all_bboxes_world.png}
  \vspace{1pt}
  {\small \\ Six chokepoint bounding boxes overlaid on world map. Individual zooms in paper appendix.}
\end{frame}

% ============================================================
% Frame 36: References
% ============================================================
\begin{frame}[allowframebreaks]{References}
  \tiny
  \bibliography{Bibliography/references}
\end{frame}


\end{document}
