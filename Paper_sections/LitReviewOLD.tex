% If you want to compile this section only, make sure to include relevant document headers and the \being \end document commands.
% You can make this a bit easier if you use the subfile package

\documentclass{article}
\usepackage[authordate,backend=biber]{biblatex-chicago}
\usepackage{geometry}
\geometry{a4paper, margin=1in}

\begin{filecontents*}{references.bib}
@article{posen2003,
  author  = {Posen, Barry R.},
  title   = {Command of the Commons: The Military Foundation of U.S. Hegemony},
  journal = {International Security},
  volume  = {28},
  number  = {1},
  pages   = {5--46},
  year    = {2003},
}
@article{vonboemcken2025,
  author  = {von Boemcken, Marc and Bola{\~n}os Su{'a}rez, Rodrigo},
  title   = {A good investment in sustainable development? A literature review on the economic and social effects of military spending},
  journal = {Defence and Peace Economics},
  year    = {2025},
  doi     = {10.1080/10242694.2025.2533766},
}
@techreport{nicastro2024,
  author      = {Nicastro, Luke A. and Tilghman, Andrew},
  title       = {U.S. Overseas Basing: Background and Issues for Congress},
  institution = {Congressional Research Service},
  year        = {2024},
  number      = {R48123},
}
@article{howell2025,
    author = {Howell, Sabrina T. and Rathje, Jason and Van Reenen, John and Wong, Jun},
    title = {Opening up Military Innovation: Causal Effects of Reforms to U.S. Defense Research},
    year = {2025},
    journal = {NBER Working Paper},
    number = {28700}
}
@article{mukaigawara2025,
  author  = {Mukaigawara, Mitsuru and Imai, Kosuke and Lyall, Jason and Papadogeorgou, Georgia},
  title   = {Spatiotemporal Causal Inference with Arbitrary Spillover and Carryover Effects: Airstrikes and Insurgent Violence in the Iraq War},
  journal = {arXiv preprint arXiv:2504.03464},
  year    = {2025},
}
@article{papadogeorgou2022,
    author = {Papadogeorgou, Georgia and Imai, Kosuke and Lyall, Jason and Li, Fan},
    title = {Causal inference with spatio-temporal data: Estimating the effects of airstrikes on insurgent violence in Iraq},
    journal = {Journal of the Royal Statistical Society: Series B (Statistical Methodology)},
    volume = {84},
    number = {5},
    pages = {1969--1999},
    year = {2022}
}
@article{hirano2003,
    author = {Hirano, Keisuke and Imbens, Guido W. and Ridder, Geert},
    title = {Efficient Estimation of Average Treatment Effects Using the Estimated Propensity Score},
    journal = {Econometrica},
    volume = {71},
    number = {4},
    pages = {1161--1189},
    year = {2003}
}
@presentation{imai2025spatial,
    author = {Imai, Kosuke},
    title = {Spatiotemporal causal inference with arbitrary spillover and carryover effects},
    year = {2025},
    note = {Summer Meeting, Japanese Society for Quantitative Political Science}
}
\end{filecontents*}

\addbibresource{references.bib}

	\title{The Hegemonic Dividend: A Literature Review on the Causal Effects of U.S. Naval Forward Presence}
\author{Carlos Carpi}
\date{February 10th, 2026}

\begin{document}

\maketitle
\url{https://www.overleaf.com/read/jgsvsjqsrsyn#a5ef30}

\begin{abstract}
We examine the intellectual foundations for quantifying the ``hegemonic dividend" generated by the United States Navy's global forward presence. We synthesize scholarship on global public goods, the economic effects of military security, and maritime commerce, arguing that a critical gap exists in empirically measuring the causal effects of naval operations on maritime risk and transaction costs. The methodological frontier of spatiotemporal causal inference identifies a promising pathway to address this gap by adapting recently developed models of interference and spillover to the maritime domain. We conclude by outlining a research agenda to integrate these theoretical and methodological streams, providing a framework for a data-driven ``cookbook" to guide future data collection and analysis for the U.S. Navy.
\end{abstract}

\section{Introduction}

Does U.S. naval forward presence---manifested through overseas basing, port calls, and freedom of navigation operations (FONOPs)---causally reduce maritime risk and lower transaction costs for global commerce? This question motivates a research agenda to quantify the ``hegemonic dividend": the extent to which U.S. naval power functions as a global public good that subsidizes the global maritime system. To establish the conceptual and methodological basis for such an undertaking, we proceed in five parts. First, we examine the theoretical logic of American hegemony and the provision of global public goods, focusing on the concept of "command of the commons." Second, we delve into the literature on defense economics, particularly the "security channel" through which military spending can affect economic outcomes. Third, we survey the literature on maritime commerce and risk, identifying key variables and data sources. Fourth, we review the literature on military posture and basing, which informs the operationalization of the treatment variables in a causal model. Finally, we assess the state of the art in spatiotemporal causal inference, arguing that recent methodological developments provide a powerful toolkit for isolating the causal effects of naval presence while accounting for the inherent complexities of the maritime domain.

\section{Global Public Goods and the Command of the Commons}

The project's intellectual bedrock is the idea that a hegemonic power can provide global public goods. The modern formulation of this concept in international relations is often traced to hegemonic stability theory, which posits that a liberal international economic order is most likely to emerge and persist when a single state has a preponderance of power and is willing to use it to create and enforce rules. \footnote{See \url{https://personal.lse.ac.uk/WYATTWAL/images/THEUS.pdf} or \url{https://www.eajournals.org/wp-content/uploads/The-Theory-of-Hegemonic-Stability-Hegemonic-Power-and-International-Political-Economic-Stability-1.pdf}} A key component of this is the provision of security, particularly for the global commons.

Barry Posen's seminal work, 	extit{Command of the Commons}, provides the central theoretical pillar for this project \parencite{posen2003}. Posen argues that U.S. dominance in the sea, air, and space commons is the military foundation of its global hegemony. This command allows the United States to project power globally, protect trade routes, and deny access to adversaries. Posen's work frames the U.S. Navy's role not merely as a tool for warfighting, but as the primary underwriter of the global system of commerce. This directly aligns with the project's goal of quantifying the "hegemonic dividend."

The logic of the commons is further elucidated in the Congressional Research Service report on overseas basing, which notes that a forward-deployed military can "signal U.S. intentions to other international actors, which may support a variety of strategic aims" \parencite{nicastro2024}. This signaling function is a key mechanism through which naval presence might generate public goods, by deterring would-be aggressors and reassuring allies and commercial actors.

\section{Defense Economics and the Security Channel}

The literature on defense economics provides a broader context for understanding the relationship between military spending and economic outcomes. A recent and comprehensive literature review by von Boemcken and Bola{\~n}os Su{'a}rez \parencite{vonboemcken2025} analyzes 140 studies and finds that the effects of military spending are highly context-dependent. While often associated with impaired economic growth and increased public debt, military spending can also generate positive externalities, particularly in high-income countries, through technological spillovers.

Crucially, von Boemcken and Bola{\~n}os Su{'a}rez identify a theoretical "security channel" where military spending creates a stable environment for development, a mechanism they note is rarely modeled empirically. This is precisely the gap the current project aims to fill. By focusing on the maritime domain and using high-frequency data on shipping and naval presence, the project can directly test the "security channel" hypothesis in a novel and policy-relevant context.

The work of Howell et al. on military innovation offers a complementary perspective \parencite{howell2025}. Their study of the Air Force's SBIR program reveals that "Open" innovation models, which allow firms to propose their own ideas, are more effective at fostering commercial innovation and military technology adoption than "Conventional" models with highly specified topics. This suggests that the *way* in which the military procures technology and engages with the private sector can have significant downstream effects. For the present project, this implies that the *nature* of naval presence—not just its intensity—may matter. For example, joint exercises with allies or the use of open-architecture systems could have different economic effects than unilateral patrols.

\section{Maritime Commerce, Shipping, and Risk}

To quantify the hegemonic dividend, the project will measure its impact on maritime commerce. We identify several key outcome variables (Y): Global Shipping Density (AIS-based), the Liner Shipping Connectivity Index (UNCTAD), and Port throughput. These are well-established indicators in the maritime economics literature \footnote{FOr examples, see \url{https://www.mdpi.com/2071-1050/13/14/7961} or models of port congestion \url{https://www.sciencedirect.com/science/article/abs/pii/S1366554524000784} or modeling container flows\url{https://www.tandfonline.com/doi/full/10.1080/03088839.2023.2220703}}. That said the World Bank datasets on "Global Shipping Traffic Density" is the  key data source so far \footnote{See \url{https://datacatalog.worldbank.org/search/dataset/0037580/global-shipping-traffic-density}}.

The project also aspires to measure the impact of naval presence on maritime risk, with a focus on insurance premiums. While acknowledging that this data is "not yet accessed," we identify this as a promising avenue for research. The logic is straightforward: if U.S. naval presence reduces the risk of piracy, conflict, or other disruptions, this should be reflected in lower insurance costs for commercial shipping. This would be a direct measure of the "hegemonic dividend."

\section{Military Posture and Forward Presence}

The treatment variable (D) in this study is U.S. naval presence. We document and the Congressional Research Service report on overseas basing \parencite{nicastro2024} provide a clear framework for operationalizing this concept. Key indicators include:

\begin{itemize}
    \item 	\textbf{Basing Footprint:} Geocoded locations of U.S. naval facilities. The CRS report provides a comprehensive list of U.S. overseas bases, which can be used to construct a detailed spatial dataset of the Navy's global footprint.
    \item 	\textbf{FONOPs Events:} Event data on Freedom of Navigation Operations, which can be sourced from the Lowy Institute FONOP tracker and DoD press releases.
    \item 	\textbf{Port Calls and Exercises:} Data on naval visits to foreign ports and joint exercises with allied navies.
    \item 	\textbf{Host Nation Agreements:} Treaty data and other agreements that provide for long-term security guarantees.
\end{itemize}

These variables capture different facets of naval presence, from the "hard power" of a carrier strike group to the "soft power" of a port visit. The richness of these data allows for a nuanced analysis of how different types of naval presence affect maritime risk and commerce.

\section{Methodological Approach: Spatiotemporal Causal Inference}

The core methodological challenge of this project is to isolate the causal effect of naval presence on maritime outcomes, in a context rife with spatial and temporal dependencies. The project's proposed solution is to adapt the spatiotemporal causal inference framework developed by Imai and his colleagues for the Air Force to measure the effect of interventions on the insurgency \parencite{imai2025spatial}.

The foundational work of Hirano, Imbens, and Ridder on propensity score methods provides the starting point for much of modern causal inference \parencite{hirano2003}. However, the standard propensity score framework is not well suited to the complexities of the maritime domain, where the "treatment" (naval presence) in one location can have spillover effects on other locations, and its effects can persist over time.

The papers by Mukaigawara et al. \parencite{mukaigawara2025} and Papadogeorgou et al. \parencite{papadogeorgou2022} offer a direct solution to this problem. Their framework, originally developed to study the effects of airstrikes on insurgent violence in Iraq, is explicitly designed to handle "arbitrary spillover and carryover effects." In fact, Imai provides a helpful conceptual overview of the methodology \parencite{imai2025spatial}.

[WILL ADD PICTURE]

The key innovation of this approach is to model treatments and outcomes as spatiotemporal point processes, and to define causal estimands based on "stochastic interventions." This allows researchers to compare the effects of different hypothetical treatment distributions, rather than being limited to the binary treatment/control framework of traditional causal inference. This is particularly well-suited to the current project, which seeks to understand the effects of different *patterns* of naval presence, not just its presence or absence.

\section{Conclusion and Implications for Measurement}

This literature review has established a firm theoretical and methodological foundation for the project's goal of quantifying the hegemonic dividend. The concept of the U.S. Navy as a provider of global public goods, articulated by Posen and others, provides the theoretical framework. The defense economics literature, particularly the "security channel" identified by von Boemcken and Bola{\~n}os Su{'a}rez, provides a causal mechanism to be tested. The maritime economics literature offers a rich set of outcome variables and data sources. And the recent work of Imai and colleagues provides a cutting-edge methodological toolkit for tackling the complex causal inference challenges of the maritime domain.

The synthesis of this literature points to a clear research agenda. The first step is to construct a comprehensive spatiotemporal dataset of U.S. naval presence and maritime outcomes. The second is to apply the Imai et al. framework to estimate the causal effects of different patterns of naval presence on shipping density, connectivity, and, if possible, insurance premiums.

The findings of this research will have significant implications for both theory and policy. For theory, it will provide a much-needed empirical test of the "security channel" and the concept of the hegemonic dividend. For policy, it will offer a data-driven "cookbook" for the U.S. Navy, providing insights into how different types of naval deployments can be used to achieve specific economic and security objectives. This will allow the Navy to better articulate its value proposition as a provider of global public goods and to make more informed decisions about resource allocation. The work by Howell et al. on "Open Innovation" in the Air Force further suggests that the *way* the Navy conducts its operations and engages with commercial actors may significantly affect the economic dividend it generates. Future research should explore these more nuanced aspects of naval presence.

\printbibliography

\end{document}
