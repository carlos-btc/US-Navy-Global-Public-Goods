% Conclusion.tex — Standalone conclusion section

Maritime security is a global public good whose economic value has been under-studied relative to its importance. This paper contributes a transparent, replicable framework for measuring chokepoint vulnerability and quantifying the hegemonic dividend from security provision. Using AIS density data from the IMF, we constructed a 78-node maritime transport network with 1,540 port-to-port trade routes and implemented a multi-scenario stress-testing engine.

Three main findings emerge. First, chokepoint disruptions produce large and heterogeneous trade-cost increases: the Panama Canal (mean rerouting cost of 17,840~km-equivalent), Gibraltar (10,624~km), and the Suez Canal (9,949~km) generate the largest rerouting costs when fully closed, while the bypass-dependent Bosporus forces Black Sea ports onto an expensive overland alternative at 5$\times$ normal maritime cost. Second, the marginal value of security provision is increasing in the risk environment: the hegemonic dividend is largest precisely when security conditions are deteriorating. Third, the vulnerability is highly heterogeneous across ports, with bypass-dependent and corridor-dependent developing-region ports benefiting most from the global public good of maritime security. Our cost-per-day estimates---ranging from \$16 million to \$233 million per day depending on the chokepoint---provide a concrete, policy-relevant metric for evaluating the economic value of maritime security provision.

These findings are offered as a data-grounded first step that identifies the data requirements and modeling extensions needed for a full causal assessment. The orders of magnitude---rerouting costs translating to tens of billions of dollars annually, with the most affected port pairs facing cost increases of 10,000--18,000~km-equivalent---suggest that the economic stakes of maritime security provision are substantial, and that the hegemonic dividend from concentrated security at critical chokepoints is a first-order feature of the global trading system.
