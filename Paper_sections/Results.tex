% Results.tex — Revised: minor polish, consistent cross-references

\subsection{Baseline Shipping Density}

Figure~\ref{fig:global_quicklook} displays the global AIS ship-density raster on a log scale. The distribution of shipping activity is extremely right-skewed: the median cell value is zero, the 90th percentile is approximately 1 AIS position, while the 99th percentile reaches approximately 8.6 million and the maximum cell intensity exceeds 50.6 million positions. This extreme skewness reflects the fundamental geographic concentration of maritime trade: the vast majority of the ocean surface carries negligible traffic, while a small fraction of cells---corresponding to established sea lanes, port approaches, and chokepoint corridors---concentrate virtually all shipping activity.

The densest corridors connect East Asia to Europe via the Suez Canal and Strait of Malacca, East Asia to North America via the Pacific, and Northern Europe to the Americas across the North Atlantic. Within each corridor, traffic funnels sharply at chokepoints, producing localized intensity spikes visible even at global scale.

\begin{figure}[H]
\centering
\includegraphics[width=\textwidth]{Figures/generated/global_quicklook.png}
\caption{Global AIS ship-density raster (log scale). Data from the IMF World Seaborne Trade Monitoring System \citep{cerdeiro2020wstms}, aggregating vessel position reports from January 2015 to February 2021. The raster has been downsampled to $3600 \times 1800$ pixels for visualization.}
\label{fig:global_quicklook}
\end{figure}

\subsection{Chokepoint Traffic Intensity}

Rather than presenting the full intensity table in the main text (see Appendix~\ref{sec:appendix_stats} for complete descriptive statistics), we focus on the Suez Canal as a worked example. This focus is motivated by the Suez Canal's central role in global trade and its historical vulnerability to disruption, most recently demonstrated by the 2021 Ever Given grounding \citep{cariou2021ais}.

\begin{figure}[H]
\centering
\includegraphics[width=0.8\textwidth]{Figures/generated/suez_intensity_detail.png}
\caption{AIS ship density at the Suez Canal (log scale). The bounding box ($[32.2, 32.65] \times [29.8, 31.3]$) captures the full navigable corridor including approach channels.}
\label{fig:suez_detail}
\end{figure}

Figure~\ref{fig:suez_stats} provides a four-panel descriptive analysis of the Suez Canal density data. Across all six chokepoints, the Strait of Malacca dominates in total summed intensity, consistent with its role as the primary corridor for Asia--Europe trade \citep{verschuur2023systemic}. An important distinction emerges between total and concentrated intensity: Gibraltar and the Bosporus have among the highest per-cell intensities, indicating extremely concentrated traffic through narrow passages \citep{notteboom2006impact}.

\begin{figure}[H]
\centering
\includegraphics[width=\textwidth]{Figures/generated/suez_density_stats.png}
\caption{Descriptive statistics of AIS density at the Suez Canal. (a)~Spatial intensity map. (b)~Cell-level intensity histogram. (c)~Comparison of mean and P99 intensity across all chokepoints. (d)~Summary statistics. Full chokepoint-by-chokepoint analysis in Appendix~\ref{sec:appendix_stats}.}
\label{fig:suez_stats}
\end{figure}

\subsection{Full Closure Scenario Results}

Table~\ref{tab:scenario_deltas} reports the results of the full closure scenario, in which each chokepoint is removed from the network and shortest-path costs are recomputed for all 1,540 port pairs. The vulnerability ranking by mean rerouting cost reveals a clear hierarchy:

\begin{enumerate}[nosep]
  \item \textbf{Panama Canal}: mean $\Delta$ cost of 17,840~km-equivalent, 298 pairs affected (19.4\%). Panama closure forces Atlantic--Pacific traffic onto extremely long trans-Pacific or Cape Horn alternatives.
  \item \textbf{Strait of Gibraltar}: mean $\Delta$ cost of 10,624~km, 590 pairs affected (38.3\%). Gibraltar is the gateway between the Atlantic and Mediterranean; its closure forces extensive rerouting for Mediterranean-bound traffic.
  \item \textbf{Suez Canal}: mean $\Delta$ cost of 9,949~km, 469 pairs affected (30.5\%). Suez closure breaks the Mediterranean--Indian Ocean corridor, forcing traffic to reroute via the Cape of Good Hope or the Panama Canal.
  \item \textbf{Bab el-Mandeb}: mean $\Delta$ cost of 4,700~km, 454 pairs affected (29.5\%). Closure forces Indian Ocean traffic to bypass the Suez corridor entirely.
  \item \textbf{Bosporus}: mean $\Delta$ cost of 3,275~km, 159 pairs affected (10.3\%). Closure forces Black Sea ports onto the Constanta--Piraeus overland bypass at 5$\times$ normal cost.
  \item \textbf{Strait of Malacca}: mean $\Delta$ cost of 1,754~km, 290 pairs affected (18.8\%). Closure forces traffic through the Lombok Strait alternative.
\end{enumerate}

% Auto-generated by 04b_enhanced_scenarios.py
\begin{table}[htbp]
\centering
\caption{Full closure scenario results: impact of removing each chokepoint on shortest-path costs across all port-to-port pairs. Mean and max $\Delta$ report the rerouting cost for pairs that find alternative routes via bypass edges.}
\label{tab:scenario_deltas}
\begin{tabular}{lrrrrr}
\toprule
Chokepoint Removed & \multicolumn{1}{c}{Mean $\Delta$} & \multicolumn{1}{c}{Max $\Delta$} & \multicolumn{1}{c}{Affected} & \multicolumn{1}{c}{Disconn.} & \multicolumn{1}{c}{Total} \\
& \multicolumn{1}{c}{(km-equiv.)} & \multicolumn{1}{c}{(km-equiv.)} & \multicolumn{1}{c}{Pairs} & \multicolumn{1}{c}{Pairs} & \multicolumn{1}{c}{Pairs} \\
\midrule
  Panama Canal & 17,840 & 45,342 & 298 & 0 & 1540 \\
  Gibraltar & 10,624 & 26,896 & 590 & 0 & 1540 \\
  Suez Canal & 9,949 & 23,317 & 469 & 0 & 1540 \\
  Bab el Mandeb & 4,700 & 9,907 & 454 & 0 & 1540 \\
  Bosporus & 3,275 & 3,540 & 159 & 0 & 1540 \\
  Strait of Malacca & 1,754 & 3,008 & 290 & 0 & 1540 \\
\bottomrule
\end{tabular}
\end{table}


\begin{figure}[H]
\centering
\includegraphics[width=\textwidth]{Figures/generated/scenario_summary.png}
\caption{Full closure scenario results. Left: chokepoints ranked by mean rerouting cost (km-equivalent) across 1,540 port pairs. Right: fraction of port pairs affected by each closure.}
\label{fig:closure_results}
\end{figure}

Figure~\ref{fig:scenario_example_suez} zooms into the Suez Canal closure, showing the 20 most affected port pairs.

\begin{figure}[H]
\centering
\includegraphics[width=\textwidth]{Figures/generated/scenario_example.png}
\caption{Suez Canal closure: 20 most affected port pairs. Mediterranean--Indian Ocean traffic is forced onto dramatically longer alternative routes via the Cape of Good Hope or Panama Canal.}
\label{fig:scenario_example_suez}
\end{figure}

Figure~\ref{fig:closure_map} maps the per-port vulnerability under the three highest-impact closures (Panama, Gibraltar, Suez).

\begin{figure}[H]
\centering
\includegraphics[width=\textwidth]{Figures/generated/closure_impact_map.png}
\caption{Port vulnerability under the three highest-impact closures. Port color intensity reflects mean percentage cost increase. X marks the closed chokepoint.}
\label{fig:closure_map}
\end{figure}

\subsection{Partial Degradation Scenarios}

Figure~\ref{fig:partial_degradation} reports the partial degradation results, in which edge costs at each chokepoint are multiplied by $\alpha \in \{1.25, 1.5, 2.0, 3.0, 5.0, 8.0, 10.0\}$ rather than the chokepoint being fully removed.

The surface plot reveals the severity curve for each chokepoint. At $\alpha = 5$, Gibraltar produces the largest mean cost increase, followed by the Suez Canal and Bab el-Mandeb. Even at $\alpha = 1.5$, the most critical chokepoints impose mean cost increases of 2--5\% across all port pairs. A notable pattern is the contrast between corridor chokepoints (Gibraltar, Suez), where severity curves accelerate monotonically, and chokepoints with viable alternatives (Panama, Malacca), where severity curves are more moderate because alternative routes absorb rerouted traffic at relatively lower cost.

\begin{figure}[H]
\centering
\includegraphics[width=\textwidth]{Figures/generated/partial_degradation_surface_3D_better_angle.png}
\caption{Partial degradation results: mean percentage cost increase across 1,540 port pairs for each chokepoint and degradation multiplier $\alpha$. The surface reveals nonlinear severity curves that accelerate at high $\alpha$ for corridor chokepoints.}
\label{fig:partial_degradation}
\end{figure}

\begin{figure}[H]
	\centering
	\includegraphics[width=\textwidth]{Figures/generated/partial_degradation_surface_lines.png}
	\caption{Partial degradation results a 2D: we split per corridor chokepoints to show which contributes the most when for the costs.}
	\label{fig:partial_degradation2}
\end{figure}


\subsection{Port Vulnerability Analysis}

Figure~\ref{fig:port_vulnerability} presents the full port-by-chokepoint vulnerability matrix: for each of the 56 ports and each of the 6 chokepoints, the cell value shows the mean percentage cost increase when the chokepoint is closed.

\begin{figure}[H]
\centering
\includegraphics[width= 0.8\textwidth]{Figures/generated/port_vulnerability_heatmap.png}
\caption{Port vulnerability heatmap: mean percentage cost increase for each port (rows) when each chokepoint (columns) is fully closed. Ports are sorted by maximum vulnerability; chokepoints by total impact.}
\label{fig:port_vulnerability}
\end{figure}

The vulnerability matrix reveals three distinct profiles:
\begin{itemize}[nosep]
  \item \textbf{Bypass-dependent ports} (behind Bosporus) face the largest per-route cost increases, concentrated on a single chokepoint.
  \item \textbf{Corridor-dependent ports} (e.g., Mediterranean ports for Suez/Gibraltar) face large but more moderate rerouting costs that vary smoothly across chokepoints.
  \item \textbf{Highly connected ports} (e.g., Rotterdam, New York, Los Angeles) with multiple routing options are resilient to any single closure.
\end{itemize}

\subsection{Security Risk Scenario Analysis}

We apply risk premiums of 10\%, 20\%, 50\%, 100\%, and 200\% to each chokepoint individually and measure the resulting mean cost increase across all 1,540 port pairs.

\begin{figure}[H]
\centering
\includegraphics[width=\textwidth]{Figures/generated/security_scenario_comparison.png}
\caption{Security risk scenario: mean percentage cost increase from applying a risk premium to each chokepoint individually. Bars show per-chokepoint impact at five premium levels.}
\label{fig:security_scenarios}
\end{figure}

Gibraltar and the Suez Canal emerge as the most sensitive to risk premiums: a 100\% risk premium on either alone increases mean trade costs by approximately 5\%. The ordering differs from the full-closure ranking because security-risk scenarios capture the \emph{incremental} effect of cost increases rather than the \emph{complete} effect of removal. Chokepoints with high baseline traffic and limited alternative routes are most sensitive to risk premiums.

The security scenario analysis enables a clean estimation of the hegemonic dividend as the \emph{cost that would be avoided if a chokepoint's risk were reduced to baseline}. Aggregating across all six chokepoints under a 50\% risk premium, the total dividend is substantial, consistent with the public-good characterization of maritime security \citep{besley2015welfare, bueger2024securing, oef2010economic}.
