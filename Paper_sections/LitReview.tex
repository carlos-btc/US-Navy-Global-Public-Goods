% LitReview.tex — Revised: shortened from 7 subsections to 4, tangential material removed
% All citations verified against Bibliography/references.bib and Readings/ folder

This section situates the paper within four literatures: maritime security as a public good; trade frictions and maritime disruptions; quantitative spatial models with transportation networks; and AIS-based measurement in maritime economics.

\subsection{Maritime Security as a Public Good}

The concept of maritime security as a global public good is central to both international relations and defense economics. \citet{bueger2019maritime} argue that the ``politics of the global sea'' are fundamentally under-theorized relative to their economic importance, while \citet{bueger2024securing} provide a comprehensive assessment establishing that global trade, energy security, and food security all depend on safe oceans---security that exhibits classic public-good characteristics of non-rivalry and partial non-excludability. The legal foundations are grounded in the international law of the sea: \citet{kraska2015law} details the framework governing freedom of navigation, passage through straits, and the rights of naval forces in international waters.

The economic costs of maritime insecurity have been documented in several contexts. \citet{mbekeani2011piracy} estimate the direct economic impacts of piracy for African development, including higher insurance costs and rerouting expenses. The One Earth Future Foundation report \citep{oef2010economic} provides a comprehensive accounting of piracy-related costs---ransoms, insurance, naval patrols, and supply-chain disruptions---estimating annual global costs in the billions of dollars.

\subsection{Trade Frictions and Maritime Disruptions}

\citet{besley2015welfare} provide a landmark study of the welfare costs of lawlessness in the context of Somali piracy, establishing a direct empirical link between maritime insecurity and shipping costs by exploiting spatial and temporal variation in piracy risk. Disruptions at chokepoints represent a particularly consequential form of trade friction because they affect the entire network structure rather than individual routes. \citet{verschuur2023systemic} analyze the systemic economic impacts of disruptions at 24 major maritime chokepoints, estimating potentially large global losses and providing strong motivation for a scenario-based modeling approach.

The port congestion literature offers complementary evidence on how capacity constraints translate into economic costs. \citet{notteboom2006impact} demonstrates that congestion at key nodes produces cost increases rippling through the shipping industry, while \citet{du2020port} document the strategies that ports employ to manage delays. \citet{morabito2023thesis} provides institutional detail on maritime transport economics that contextualizes network models within the operational realities of the shipping industry.

\subsection{Quantitative Spatial Economics and Transportation Networks}

The methodological backbone of this paper is the quantitative spatial economics framework. \citet{redding2017quantitative} provide the foundational survey, establishing the canonical framework in which locations differ in productivity and amenities, trade costs govern the flow of goods, and equilibrium outcomes depend on the entire spatial distribution of economic activity. \citet{allen2025quantitative} and \citet{redding2025urban} extend this framework to regional and urban analysis, developing the vocabulary of iceberg trade costs, gravity-based flows, and hat-algebra counterfactuals that we adapt to the maritime setting. \citet{proost2019what} emphasize the field's ability to evaluate policies with spatial dimensions---a category that naturally includes maritime security.

Within this tradition, the transportation networks literature provides our core analytical engine. \citet{fajgelbaum2020optimal} develop the foundational framework for optimal transport networks in spatial equilibrium, demonstrating that optimal networks concentrate investment on high-traffic corridors---a finding with direct parallels to maritime chokepoints. \citet{allen2022welfare} develop a quantitative model with endogenous traffic congestion and apply it to the U.S.\ highway network, providing a close methodological parallel to our maritime cost specification. \citet{donaldson2025transport} offers a comprehensive survey of transport infrastructure evaluation, emphasizing the distinction between identified causal effects and model-based counterfactuals that directly motivates our approach. \citet{hierons2024spreading} studies optimal congestion pricing in general equilibrium, providing guidance for the security allocation extension of our model. \citet{bordeu2025commuting} examines infrastructure investment under political fragmentation, highlighting coordination problems analogous to the provision of global maritime security when international cooperation is incomplete. \citet{rossihansberg2020transportation} provides a compact analytical treatment of transportation network models that highlights the key features delivering tractability: sparse network structure, congestion-based costs, and shortest-path routing---features directly incorporated into our maritime network specification.

\subsection{AIS Data in Maritime Economics}

The empirical implementation relies on Automatic Identification System (AIS) data, which have become a cornerstone of modern maritime economics. \citet{kerbl2022ais} provides an overview of AIS applications in economic research, documenting both opportunities and challenges. \citet{cariou2021ais} use AIS data to study the economic impact of the COVID-19 shock on container flows and ship waiting times, illustrating how AIS supports real-time disruption measurement. \citet{cerdeiro2020nowcasting} combine AIS data with machine learning to nowcast world trade, demonstrating the predictive content of shipping movements for macroeconomic outcomes. Together with the IMF World Seaborne Trade Monitoring System \citep{cerdeiro2020wstms} that provides our primary dataset, these papers establish the feasibility of constructing traffic flows, congestion proxies, and intensity measures from AIS data.

\subsection*{Summary}

The existing literature provides a clear foundation: the quantitative spatial economics toolkit \citep{redding2017quantitative, allen2025quantitative} supplies the theoretical structure; transportation network models \citep{fajgelbaum2020optimal, allen2022welfare} provide the analytical engine; and AIS data \citep{kerbl2022ais, cerdeiro2020wstms} enable empirical measurement. Yet a gap remains: existing work rarely combines a global maritime network model with endogenous rerouting and congestion to value security provision under counterfactual disruptions. The chokepoint vulnerability literature \citep{verschuur2023systemic} establishes the stakes but relies on reduced-form methods rather than network-equilibrium counterfactuals. The piracy literature \citep{besley2015welfare, oef2010economic} quantifies costs but treats routes as fixed. Our paper bridges this gap by combining insights from the transportation networks tradition with AIS-based measurement and a flexible scenario framework.
