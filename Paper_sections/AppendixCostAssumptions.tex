% AppendixCostAssumptions.tex — Cost and distance assumptions moved from Data and Methods

Our scenario engine reports rerouting costs in \emph{km-equivalent} units, reflecting weighted shortest-path distances through the maritime network. To translate these into approximate economic magnitudes, Table~\ref{tab:cost_assumptions} documents all assumptions used in the cost mapping.

\begin{longtable}{p{3.8cm}p{2.2cm}p{1.5cm}p{5.5cm}}
\caption{Cost and Distance Assumptions} \label{tab:cost_assumptions} \\
\toprule
Parameter & Value & Units & Justification \& Source \\
\midrule
\endfirsthead
\toprule
Parameter & Value & Units & Justification \& Source \\
\midrule
\endhead
\midrule
\multicolumn{4}{r}{\small\emph{Continued on next page}} \\
\endfoot
\bottomrule
\endlastfoot

\textbf{Haversine distance} & Computed & km & Great-circle distance between node coordinates; standard geodesic formula \citep{allen2022welfare} \\

\textbf{Congestion multiplier range} & $[1.0, 1.5]$ & dimensionless & Calibrated from AIS density via log-linear mapping; upper bound consistent with port congestion estimates \citep{notteboom2006impact} \\

\textbf{Overland bypass: Bosporus (Constanta--Piraeus)} & $5.0 \times$ haversine & km-equiv. & Rail/truck intermodal alternative through southeastern Europe; overland freight typically 3--6$\times$ maritime cost \citep{notteboom2006impact} \\

\textbf{Fuel cost (bunker)} & \$400--600 & per tonne & IFO 380 benchmark price range 2019--2023; IMO 2020 sulphur cap increased costs by 25--40\% for compliant fuels \citep{notteboom2006impact} \\

\textbf{Average fuel consumption} & 150--250 & tonnes/day & Typical for Panamax--post-Panamax container vessels at design speed \citep{du2020port} \\

\textbf{Shipping cost per tonne-km} & \$5--10 & per 1,000 km & Containerized cargo; range reflects vessel size, fuel prices, and route \citep{notteboom2006impact} \\

\textbf{War-risk insurance premium} & 0.1--0.5\% & of hull value & Baseline range; can spike to 1--5\% in conflict zones \citep{besley2015welfare, oef2010economic} \\

\textbf{Canal transit tolls: Suez} & \$0.3--1.0M & per transit & Varies by vessel size and cargo type; represents 2--8\% of total voyage cost for Asia--Europe routes \citep{cariou2021ais} \\

\textbf{Canal transit tolls: Panama} & \$0.2--0.8M & per transit & New Panamax locks increased capacity but raised tolls \citep{notteboom2006impact} \\

\textbf{Average vessel speed} & 12--16 & knots & Slow-steaming range for fuel efficiency \citep{du2020port} \\

\textbf{Daily time charter rate} & \$10,000--50,000 & per day & Container vessel range depending on size and market conditions \citep{notteboom2006impact} \\

\textbf{Crew cost} & \$3,000--8,000 & per day & All-in crew cost for 20--25 person complement \citep{oef2010economic} \\

\textbf{Risk premium scenarios} & 10--200\% & of base cost & Applied to chokepoint-adjacent edges; upper range consistent with war-risk premium spikes observed during Gulf conflicts \citep{besley2015welfare} \\

\textbf{Partial degradation multiplier} & 1.25--10$\times$ & dimensionless & Range captures mild congestion ($1.25\times$) through near-closure ($10\times$) \\

\textbf{Global seaborne trade volume} & $\sim$11 billion & tonnes/year & UNCTAD 2023 estimate; used for order-of-magnitude cost translations \citep{verschuur2023systemic} \\
\end{longtable}

\paragraph{Translating km-equivalent to economic costs.} A mean rerouting cost of $\Delta$ km-equivalent can be approximately converted to annual economic impact as:
\begin{equation}
  \text{Annual cost} \approx \Delta \times \frac{\text{Trade volume (tonnes)}}{\text{Average voyage distance}} \times \text{Cost per tonne-km},
\end{equation}
where we use illustrative values of \$5--10 per tonne per 1,000~km for containerized cargo and approximately 11 billion tonnes of global seaborne trade. These conversions are intended as order-of-magnitude estimates; precise economic costs depend on cargo composition, vessel types, demand elasticities, and market conditions that are beyond the scope of this analysis.

\paragraph{Sensitivity analysis.} We conduct sensitivity analysis along three dimensions: (i)~bounding box size (expanding/contracting chokepoint definitions by 20\%); (ii)~congestion parameter ($\lambda$ varied from 0 to 0.3); and (iii)~cost multiplier range for the congestion normalization. These checks ensure that results are not artifacts of arbitrary parameter choices.
