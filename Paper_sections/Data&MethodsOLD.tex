

% If you want to compile this section only, make sure to include relevant document headers and the \being \end document commands.
% You can make this a bit easier if you use the subfile package

\documentclass{article}
\usepackage{geometry}
\usepackage{amsmath}
\usepackage{amssymb}
\usepackage{hyperref}
\usepackage{natbib}

% Formatting for readability in draft
\geometry{margin=1in}

\begin{document}
	
	\section{Data and Methods}

\subsection{Research Design Overview}
This project employs a \textbf{spatiotemporal causal inference framework} adapted from Imai et al. (2025) to quantify the economic externalities of U.S. Naval presence. Unlike traditional defense economics models that rely on aggregate GDP growth or military spending variables, this design disaggregates analysis to the maritime domain, treating the "Command of the Commons" as a measurable input into global trade stability.

The core analytical objective is to isolate the \textbf{"Security Signal"}—the causal reduction in maritime risk and transaction costs attributable to U.S. forward presence. This effectively measures the "hegemonic dividend," or the extent to which the U.S. Navy acts as a global public goods provider and institutional infrastructure for the global economy.

\subsection{Methodological Framework}
We adapt the spatiotemporal interference model (Imai et al., 2025) to a maritime setting. The model is designed to account for arbitrary spillover effects, crucial for naval operations where a "Freedom of Navigation Operation" (FONOP) in one sector may reduce risk premiums in adjacent sectors.

\subsubsection*{Formal Model Specification}
Let $Y_{it}$ denote the outcome of interest (e.g., shipping density or insurance risk proxy) in spatial unit $i$ at time $t$. Let $Z_{it}$ denote the treatment variable (U.S. Naval presence). The spatiotemporal causal model is defined as:

\begin{equation}
	Y_{it} = \alpha_i + \gamma_t + \beta Z_{it} + \delta \sum_{j \in N(i)} w_{ij} Z_{jt} + \epsilon_{it}
\end{equation}

Where:
\begin{itemize}
	\item $Z_{it}$: Binary or continuous measure of Naval presence in grid $i$ at time $t$.
	\item $\sum_{j \in N(i)} w_{ij} Z_{jt}$: The spillover term, capturing the "Security Signal" from neighboring sectors $N(i)$.
	\item $\alpha_i, \gamma_t$: Unit and time fixed effects to control for invariant geographic features (e.g., depth, distance to shore) and global macroeconomic shocks.
\end{itemize}

\subsection{Data Strategy and Sources}
Data accessibility regarding military deployments is the primary constraint of this project. Consequently, the project adopts a dual-track data strategy: (1) Analysis of available public proxies to demonstrate proof-of-concept, and (2) The development of a "Data Collection Cookbook" for the U.S. Navy, identifying precisely which internal datasets must be declassified or organized to allow for full-scale econometric validation.

\subsubsection{Outcome Variables ($Y$)}
We target indicators of friction in global commerce:
\begin{itemize}
	\item \textbf{Global Shipping Density:} Data derived from Automatic Identification System (AIS) signals, available via the World Bank Data Catalog (Global Shipping Traffic Density). This serves as a proxy for economic activity and route confidence.
	\item \textbf{Maritime Risk / Insurance Premiums:} Aspirational data on hull and machinery (H\&M) or war risk premiums from Lloyd’s List or similar industry bodies. 
	\item \textbf{Connectivity Indices:} The Liner Shipping Connectivity Index (UNCTAD) to measure port-level integration resilience.
\end{itemize}

\subsubsection{Treatment Variables ($D$)}
Treatment is defined as the intensity of U.S. Naval presence:
\begin{itemize}
	\item \textbf{Basing Footprint:} Geocoded locations of U.S. naval facilities (Points + Buffer zones).
	\item \textbf{FONOPs Events:} Event data sourced from the Lowy Institute FONOP tracker and DoD press releases, measuring active assertion of international law.
	\item \textbf{Host Nation Agreements:} Treaty data acting as a proxy for long-term security guarantees.
\end{itemize}

\subsection{Feasibility and Implementation}
\subsubsection{Addressing Data Constraints}
The "Data Deadlock" regarding granular deployment data (e.g., specific submarine locations or carrier group movements) is acknowledged. To mitigate this:
\begin{itemize}
	\item \textbf{Scenario Analysis:} Following feedback from Professor Rossi-Hansberg (January 2026), we will supplement the econometric model with scenario simulations focused on maritime chokepoints (e.g., Strait of Hormuz, Malacca Strait). This models the counterfactual: \textit{"What is the cost to global trade flow if Chokepoint X closes in the absence of a U.S. security guarantee?"}
	\item \textbf{The "Cookbook" Deliverable:} A key output of this professional MA thesis is a technical white paper addressed to the Department of the Navy. This document will specify the data schema required to run the Imai et al. (2025) model internally, effectively arguing that the Navy currently lacks the statistical instrumentation to measure its own economic ROI.
\end{itemize}

\subsubsection{Institutional Feasibility}
Outreach paths have been established to validate the utility of this framework with potential stakeholders:
\begin{itemize}
	\item \textbf{Naval Postgraduate School (Acquisition Research Program):} Validating the cost-benefit analysis framework.
	\item \textbf{Marine Corps Warfighting Lab (MCWL):} Assessing operational relevance.
	\item \textbf{Maritime Industry Experts:} Interviews with ship brokers and insurers (e.g., Hamburg Sud, Maersk) to validate the "Insurance Route" assumptions.
\end{itemize}

\end{document}