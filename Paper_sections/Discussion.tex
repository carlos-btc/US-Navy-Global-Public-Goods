% Discussion.tex — Revised: separated from Conclusion; spatial policy citations integrated

\subsection{The Cost of Chokepoint Disruption}

Table~\ref{tab:cost_per_day} presents the estimated daily rerouting cost of each chokepoint closure, combining our scenario engine outputs with published annual transit counts and standard maritime shipping cost estimates (see Appendix~\ref{sec:appendix_shipcounts} for the underlying ship-count estimation methodology).

% Auto-generated by 04b_enhanced_scenarios.py
%%\begin{table}[htbp]
%\centering
%\caption{Estimated daily rerouting cost of chokepoint closure. Low/Mid/High estimates use \$50/\$75/\$100 per additional km respectively, reflecting fuel, charter time, and crew costs. Daily vessel counts from UNCTAD, Suez Canal Authority, and Panama Canal Authority statistics.}
%\label{tab:cost_per_day}
%\begin{tabular}{lrrrrrr}
%\toprule
%Chokepoint & Daily Vessels & Mean Detour (km) & Cost/Day (Low) & Cost/Day (Mid) & Cost/Day (High) \\
%\midrule
%  Gibraltar & 219 & 10,624 & \$116.4M & \$174.6M & \$232.9M \\
%  Panama Canal & 37 & 17,840 & \$33.0M & \$49.5M & \$66.0M \\
%  Suez Canal & 53 & 9,949 & \$26.6M & \$39.9M & \$53.2M \\
%  Strait of Malacca & 233 & 1,754 & \$20.4M & \$30.6M & \$40.9M \\
%  Bosporus & 118 & 3,275 & \$19.3M & \$28.9M & \$38.6M \\
%  Bab el Mandeb & 68 & 4,700 & \$16.1M & \$24.1M & \$32.2M \\
%\bottomrule
%\end{tabular}
%\end{table}

\begin{table}[H]
	\centering
	\caption{Estimated daily rerouting cost of chokepoint closure. Low/Mid/High estimates use \$50/\$75/\$100 per additional km respectively, reflecting fuel, charter time, and crew costs. Daily vessel counts from UNCTAD, Suez Canal Authority, and Panama Canal Authority statistics.}
	\label{tab:cost_per_day}
	\resizebox{\textwidth}{!}{%
		\begin{tabular}{lrrrrrr}
			\toprule
			Chokepoint & Daily Vessels & Mean Detour (km) & Cost/Day (Low) & Cost/Day (Mid) & Cost/Day (High) \\
			\midrule
			Gibraltar & 219 & 10,624 & \$116.4M & \$174.6M & \$232.9M \\
			Panama Canal & 37 & 17,840 & \$33.0M & \$49.5M & \$66.0M \\
			Suez Canal & 53 & 9,949 & \$26.6M & \$39.9M & \$53.2M \\
			Strait of Malacca & 233 & 1,754 & \$20.4M & \$30.6M & \$40.9M \\
			Bosporus & 118 & 3,275 & \$19.3M & \$28.9M & \$38.6M \\
			Bab el Mandeb & 68 & 4,700 & \$16.1M & \$24.1M & \$32.2M \\
			\bottomrule
		\end{tabular}%
	}
\end{table}

The cost-per-day estimates reveal a clear hierarchy of economic exposure. Gibraltar closure produces the largest daily rerouting cost (\$117--233 million per day), driven by its high daily vessel throughput (219 vessels/day) combined with a substantial mean detour of 10,624~km. The Panama Canal follows (\$33--66M/day): although it handles far fewer vessels (37/day), the extreme detour distance (17,840~km) amplifies the per-vessel rerouting cost. The Suez Canal (\$27--53M/day) and the Strait of Malacca (\$20--41M/day) occupy the middle tier, while the Bosporus (\$19--39M/day) and Bab el-Mandeb (\$16--32M/day) produce the lowest---though still substantial---daily costs.

Aggregating across all six chokepoints, a simultaneous closure would produce daily rerouting costs on the order of \$230--460 million, or approximately \$85--170 billion annually---magnitudes consistent with the order-of-magnitude estimates in \citet{oef2010economic} and \citet{mbekeani2011piracy} for the broader economic costs of maritime insecurity.

\subsection{The Hegemonic Dividend}

The per-chokepoint security risk analysis (Section~\ref{sec:results}) provides a direct estimate of the hegemonic dividend---the trade-cost increase that would materialize if security deteriorated and risk premiums rose. Under a 100\% risk premium, plausible in a scenario of piracy escalation or military confrontation \citep{besley2015welfare}, the per-chokepoint dividend ranges from moderate (Strait of Malacca, approximately 1\%) to substantial (Gibraltar and Suez Canal, approximately 5\% each). Aggregating across all six chokepoints, the total dividend implies trade-cost savings equivalent to thousands of km-equivalent per port pair, translating to tens of billions of dollars annually at current trade volumes.

Crucially, the marginal value of security provision is \emph{increasing in the risk environment}: the dividend is largest precisely when security conditions are deteriorating. As risk levels rise due to piracy, conflict spillovers, or geopolitical tensions, the economic case for sustained security investment strengthens rather than weakens \citep{oef2010economic, mbekeani2011piracy}. This formalizes the intuition from the international relations and defense economics literatures \citep{bueger2019maritime, kraska2015law}.

\subsection{Structural Patterns}

\paragraph{Network topology and vulnerability.} The 78-node network reveals a structural distinction between two types of chokepoints. \emph{Through-corridor chokepoints} (Malacca, Gibraltar, Suez, Bab el-Mandeb, Panama) connect large regions; their closure imposes rerouting costs on many port pairs but does not sever connectivity, because alternative maritime routes absorb rerouted traffic. \emph{Bypass-dependent chokepoints} (Bosporus) are the primary connectors for ports behind them; their closure forces traffic onto expensive overland bypass routes. This distinction maps onto the theoretical framework of \citet{fajgelbaum2020optimal}, who show that optimal transport networks concentrate investment on high-throughput corridors.

Security priorities should therefore differ by chokepoint type. For through-corridor chokepoints, the objective is to minimize rerouting costs by maintaining capacity and reducing risk premiums. For bypass-dependent chokepoints, the objective is to prevent disruption that forces traffic onto dramatically more expensive alternatives. This connects to the congestion externality analysis of \citet{hierons2024spreading}: congestion pricing or capacity investment at through-corridor chokepoints can reduce rerouting costs, whereas bypass-dependent chokepoints require both security provision and investment in alternative infrastructure.

\paragraph{Nonlinear severity.} The partial degradation analysis reveals that the relationship between disruption severity and cost increase is nonlinear. For through-corridor chokepoints, severity curves accelerate at high degradation levels ($\alpha \geq 5$) as rerouting becomes the dominant response. Gibraltar and the Suez Canal exhibit the steepest severity curves, producing mean cost increases exceeding 26\% and 12\% respectively at $\alpha = 5$. This nonlinearity implies that the marginal return to reducing risk is highest at moderate disruption levels, suggesting that maintaining a credible deterrent captures most of the available dividend.

\paragraph{Heterogeneous vulnerability.} The port vulnerability analysis reveals that the dividend is not uniformly distributed. Bypass-dependent ports (Black Sea) benefit most from their chokepoint's security. Corridor-dependent ports (Mediterranean, South Asian, East African) benefit from the security of multiple chokepoints along their primary trade routes. Highly connected hub ports (Rotterdam, New York, Singapore) are relatively insulated. This heterogeneity implies that the political economy of maritime security investment is complex: the ports that benefit most from the public good are often in developing regions with limited capacity to contribute to its provision \citep{gaubert2025place, fajgelbaum2019state}. The geographic sorting dynamics studied by \citet{desmet2015geography} and the agglomeration mechanisms of \citet{duranton2004micro} suggest that security investments may also affect the long-run spatial distribution of economic activity, as firms and workers adjust location decisions in response to changed trade-cost structures.

\subsection{Limitations}

We are explicit about several limitations that constrain interpretation while noting that each suggests a concrete direction for future work.

\paragraph{Static aggregate density.} The AIS data aggregate vessel positions from January 2015 to February 2021 into a single snapshot. We cannot track temporal changes, seasonal variations, or dynamic adjustment to disruptions. Temporal disaggregation would enable analysis of how trade routes evolve in response to shocks \citep{cariou2021ais, du2020port}.

\paragraph{Aggregate density versus vessel-level flows.} The density raster records total AIS positions without distinguishing vessel type, flag, cargo, or voyage. Extending to vessel-level AIS data would enable construction of origin-destination trade matrices and support estimation of the full spatial equilibrium model \citep{fajgelbaum2020optimal, allen2022welfare, kerbl2022ais}.

\paragraph{No welfare computation.} Without bilateral trade volumes and demand elasticities, we cannot compute welfare changes in the general-equilibrium sense of \citet{redding2017quantitative} or \citet{allen2025quantitative}. Our delta-cost measures are informative about direction and relative magnitude but are not welfare estimates.

\paragraph{Exogenous security parameters.} The security risk analysis treats risk premiums as exogenous inputs. In reality, security provision is endogenous to the strategic environment \citep{besley2015welfare, kraska2015law}. Endogenizing security would require a game-theoretic extension in which a security provider allocates resources across chokepoints to minimize expected trade-cost losses \citep{fajgelbaum2020spatial}.

\subsection{Directions for Future Research}

\begin{enumerate}[nosep]
  \item \textbf{Vessel-level AIS trajectories.} Individual vessel tracks would enable construction of origin-destination trade flow matrices and support estimation of route-choice models with probabilistic assignment \citep{fajgelbaum2020optimal, fajgelbaum2020supplement}.

  \item \textbf{Bilateral maritime trade data.} Combining vessel-level AIS with customs or port-authority trade data would enable calibration of a full spatial equilibrium model with endogenous demand \citep{redding2017quantitative, allen2025quantitative, redding2025urban}.

  \item \textbf{Insurance and risk pricing data.} War-risk insurance premiums and piracy incidence records would provide empirically grounded values for the security parameters \citep{besley2015welfare, oef2010economic}.

  \item \textbf{Naval presence proxies.} Open-source data on naval deployments would enable causal estimation of the security provision effect.

  \item \textbf{Dynamic extensions.} A time-varying network with stochastic disruptions and adjustment costs would capture the temporal dimension of chokepoint risk \citep{hierons2024spreading, bordeu2025commuting}.

  \item \textbf{Optimal security allocation.} A normative extension could solve for the optimal spatial allocation of security resources, building on \citet{rossihansberg2004optimal}, \citet{fajgelbaum2020spatial, fajgelbaum2025optimal}, and \citet{gaubert2025place}: given a fixed security budget, which chokepoints should receive the most protection?

  \item \textbf{Political economy of burden-sharing.} The observation that maritime security is a global public good with concentrated provision raises questions about sustainability and equity. An extension incorporating political-economy considerations \citep{fajgelbaum2019state, bordeu2025commuting} could analyze the efficiency and distributional consequences of alternative burden-sharing regimes. The frameworks of \citet{owens2020rethinking} and \citet{rossihansberg2023cognitive} offer tools for evaluating how spatial redistribution interacts with agglomeration and connectivity.
\end{enumerate}
