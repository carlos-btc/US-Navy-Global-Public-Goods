


\subsection{Data}

\subsubsection{AIS Ship Density Raster}

Our primary data source is the global ship density raster from the IMF World Seaborne Trade Monitoring System \citep{cerdeiro2020wstms}. The dataset aggregates AIS (Automatic Identification System) position reports received between January 2015 and February 2021 into a global raster grid at 0.005\textdegree{} $\times$ 0.005\textdegree{} resolution (approximately 500m $\times$ 500m at the equator). Each cell records the total number of AIS positions reported within that cell over the entire period, covering both moving and stationary vessels. The raster thus measures the \emph{intensity of shipping activity} at each location---a proxy for the cumulative traffic load experienced by each point in the global ocean.

The dataset is distributed as a single GeoTIFF file (approximately 9~GB) with pre-computed overview pyramids (approximately 3~GB). The raster covers the full global extent in WGS84 geographic coordinates. Given its size, all processing uses memory-safe techniques: built-in overviews for global visualization, windowed reads for chokepoint-level extraction, and aggressive downsampling (to 3600 $\times$ 1800 pixels) for summary statistics. The AIS data underlying the raster have been validated for use in economic research; see \citet{kerbl2022ais} for a discussion of AIS data quality, coverage, and applications, and \citet{cerdeiro2020nowcasting} for the IMF's methodology for constructing trade-relevant indicators from AIS.

\subsubsection{Chokepoint Definitions}

We define six maritime chokepoints as coarse bounding boxes in geographic coordinates (Table~\ref{tab:chokepoint_definitions}). Each bounding box encloses the navigable channel and approach lanes of a major maritime bottleneck. We extract the AIS density within each bounding box via windowed raster reads and compute aggregate intensity statistics. The six chokepoints are:

\begin{enumerate}[nosep]
  \item \textbf{Suez Canal} --- connecting the Mediterranean and Red Sea;
  \item \textbf{Bab el-Mandeb} --- southern entrance to the Red Sea;
  \item \textbf{Strait of Malacca} --- connecting the Indian Ocean and South China Sea;
  \item \textbf{Panama Canal} --- connecting the Atlantic and Pacific Oceans;
  \item \textbf{Bosporus} --- connecting the Black Sea and Mediterranean;
  \item \textbf{Strait of Gibraltar} --- entrance to the Mediterranean from the Atlantic.
\end{enumerate}

These chokepoints were selected based on their prominence in the maritime security literature \citep{verschuur2023systemic, bueger2024securing} and their role as physical bottlenecks that concentrate traffic and create vulnerability. The Strait of Hormuz, the Cape of Good Hope, and the Danish Straits are included in the network as waypoint nodes (enabling realistic routing) but are not classified as chokepoints for the closure analysis: the Strait of Hormuz functions as a critical but open passage without the canal-like physical constraints of Suez or Panama; the Cape of Good Hope is an open-ocean alternative route; and the Danish Straits do not constitute a narrow mandatory-passage bottleneck for modern shipping. The bounding boxes are deliberately coarse to ensure computational tractability and to capture approach-lane traffic in addition to the narrowest passage.

\begin{table}[H]
\centering
\caption{Chokepoint Bounding Box Definitions}
\label{tab:chokepoint_definitions}
\small
\begin{tabular}{lcccc}
\toprule
Chokepoint & Lon Min & Lon Max & Lat Min & Lat Max \\
\midrule
Suez Canal        & 32.20 & 32.65 & 29.80 & 31.30 \\
Bab el-Mandeb     & 42.25 & 43.55 & 12.15 & 13.25 \\
Strait of Malacca & 99.00 & 104.80 & 1.00 & 6.50 \\
Panama Canal      & $-$80.20 & $-$79.40 & 8.70 & 9.60 \\
Bosporus          & 28.90 & 29.30 & 41.05 & 41.25 \\
Gibraltar         & $-$6.20 & $-$5.20 & 35.80 & 36.30 \\
\bottomrule
\end{tabular}
\end{table}

Figure~\ref{fig:bbox_example} illustrates the bounding box extraction procedure using the Suez Canal as an example. The bounding box captures the canal and its approach lanes on the Mediterranean and Red Sea sides, where vessel traffic concentrates into a high-density corridor. The full set of six bounding boxes is shown on a world map in Appendix~A (Figure~\ref{fig:appendix_all_bboxes}).

\begin{figure}[H]
\centering
\includegraphics[width=0.85\textwidth]{Figures/bounding_box_example.png}
\caption{AIS ship density at the Suez Canal with the chokepoint bounding box overlaid (dashed cyan). The heatmap shows log-transformed AIS position counts; brighter areas indicate higher traffic intensity.}
\label{fig:bbox_example}
\end{figure}

\subsection{Methods}

\subsubsection{Empirical Strategy: Bottleneck Scenario Analysis on a Maritime Transport Network}

The central challenge in quantifying maritime security as a global public good is that the most policy-relevant outcomes---trade reliability, routing resilience, and avoided trade-cost spikes---are \emph{equilibrium objects} of a transportation network \citep{fajgelbaum2020optimal}. When a chokepoint is disrupted through conflict risk, congestion, or physical closure, global shipments reroute endogenously. Any credible evaluation must therefore model \emph{trade-flow substitution across routes} rather than treat observed routes as fixed.

Following the transportation networks framework \citep{allen2022welfare, fajgelbaum2020optimal}, we build a quantitative spatial model of global maritime shipping on a graph in which: (i) shipments choose routes through the network; (ii) link-level costs depend on distance, traffic (congestion), and link quality/capacity; and (iii) counterfactuals are conducted by shocking bottleneck links and recomputing equilibrium flows and delivered trade costs. The Navy's role as a global public good provider is introduced as a cost-reducing and risk-reducing shifter on bottleneck-adjacent edges: higher security provision lowers effective iceberg trade costs and improves reliability \citep{besley2015welfare, bueger2024securing}.

\subsubsection{Network Construction}

We represent the global maritime system as an undirected weighted graph $G = (\mathcal{J}, \mathcal{E})$ with 78 nodes and approximately 155 edges:

\begin{itemize}[nosep]
  \item \textbf{Port nodes} (56): Major container and bulk ports worldwide, including Shanghai, Singapore, Rotterdam, New York, Dubai, Santos, and 50 others, selected to represent the major origin-destination endpoints of global maritime trade. Each port is placed at its real-world geographic coordinates.
  \item \textbf{Chokepoint nodes} (6): The six maritime bottlenecks defined in Section~3.1.2, placed at the center of their respective bounding boxes.
  \item \textbf{Waypoint nodes} (16): Open-ocean routing waypoints---including the Strait of Hormuz, North Pacific, North Atlantic, Central Pacific, Caribbean Sea, Arabian Sea, Bay of Bengal, South China Sea, East Mediterranean, Norwegian Sea, Cape of Good Hope, Lombok Strait, Aden Junction, Mozambique Channel, West Africa, and South Atlantic---that ensure realistic path geometry through intermediate waters. Trans-ocean and inter-regional edges pass through these waypoints so that computed routes follow realistic maritime arcs rather than great-circle paths across continents.
  \item \textbf{Edges} $\mathcal{E}$ ($\sim$155): Feasible maritime connections following known shipping corridors. Edge costs are proportional to great-circle distances between endpoint coordinates, with optional cost multipliers for overland bypass routes.
\end{itemize}

A critical design feature is the \emph{bypass-dependent topology}: Black Sea ports (Constanta, Odessa, Novorossiysk) connect primarily through the \textbf{Bosporus}, with an overland bypass via southeastern European rail/road corridors (Constanta--Piraeus) at approximately 5 times the normal cost. This design ensures that Bosporus closure produces \emph{large but finite} rerouting costs---reflecting the existence of extremely costly real-world alternatives---rather than complete disconnection. Persian Gulf ports connect through the Strait of Hormuz waypoint to the Arabian Sea, reflecting their geographic position but without a mandatory-passage constraint in the closure analysis. Baltic ports (St.~Petersburg, Gdansk, Stockholm, Helsinki) connect directly to Northwestern European ports. Indian Ocean--Pacific traffic must transit through either the Strait of Malacca or the Lombok Strait alternative south of Indonesia, consistent with real-world shipping patterns \citep{verschuur2023systemic}.

The 56 ports generate $\binom{56}{2} = 1{,}540$ unique origin-destination pairs, enabling granular analysis of how chokepoint disruptions propagate across specific trade routes. Figure~\ref{fig:network_map} shows the full network overlaid on a world map, with overland bypass edges shown as dashed lines.

\begin{figure}[H]
\centering
\includegraphics[width=\textwidth]{Figures/network_world_map.png}
\caption{Maritime transport network: 78 nodes (56 ports, 6 chokepoints, 16 waypoints) connected by approximately 155 edges following major shipping corridors. Red diamonds mark chokepoints; blue circles mark ports; gray squares mark ocean waypoints. Dashed brown lines indicate overland bypass routes with elevated cost multipliers.}
\label{fig:network_map}
\end{figure}

\subsubsection{Edge Cost Specification}

Let $t_e \geq 1$ denote the generalized cost of traversing edge $e$. Inspired by congestion-based network models \citep{allen2022welfare, hierons2024spreading}, we parameterize:
\begin{equation}
  t_e = \bar{t}_e \cdot \left(\Xi_e\right)^{\lambda},
  \label{eq:congestion}
\end{equation}
where $\Xi_e$ is a traffic proxy on edge $e$ (derived from AIS density at the adjacent chokepoint) and $\lambda \geq 0$ captures congestion---the degree to which costs are increasing in traffic intensity. The baseline component $\bar{t}_e$ depends on great-circle distance between the endpoints of edge $e$:
\begin{equation}
  \bar{t}_e = \text{dist}_e \cdot m_e,
  \label{eq:baseline_cost}
\end{equation}
where $m_e$ is an edge-specific multiplier that captures physical constraints (narrow channels, mandatory traffic separation) and security/risk conditions.

\paragraph{Security as a cost shifter.} We model security provision as a reduction in the effective cost multiplier on bottleneck-adjacent edges:
\begin{equation}
  m_e = m_e^{\text{base}} \cdot (1 + \delta_{\text{risk}} \cdot \text{Risk}_e) \cdot (1 - \delta_{\text{sec}} \cdot S_e),
  \label{eq:security_cost}
\end{equation}
where $\text{Risk}_e$ is a risk indicator (e.g., piracy incidence, conflict proximity), $S_e$ is security intensity (naval presence, patrol frequency), and $\delta_{\text{risk}}, \delta_{\text{sec}} > 0$ are parameters governing the sensitivity of costs to risk and security. This formulation follows \citet{besley2015welfare} in treating insecurity as a trade-cost shifter, and extends it by making security endogenous to the cost structure. In the absence of direct naval presence data, we treat the security parameters as sensitivity analysis inputs and report results across a range of plausible values.

Figure~\ref{fig:edge_cost} illustrates the edge cost structure along the Suez--Red Sea corridor, showing how great-circle distances between nodes determine baseline costs.

\begin{figure}[H]
\centering
\includegraphics[width=0.85\textwidth]{Figures/edge_cost_diagram.png}
\caption{Edge cost illustration: Suez--Red Sea corridor. Numbers on edges show great-circle distances (km). The cost formula $t_e = \text{dist}_e \times m_e \times \Xi_e^{\lambda}$ scales these distances by congestion and security multipliers.}
\label{fig:edge_cost}
\end{figure}

Figure~\ref{fig:security_shifter} shows a side-by-side comparison of the network under baseline conditions and under a risk premium scenario, illustrating how the security cost shifter amplifies edge costs at the affected chokepoint.

\begin{figure}[H]
\centering
\includegraphics[width=\textwidth]{Figures/security_shifter_comparison.png}
\caption{Security cost shifter: baseline (left) versus risk premium scenario (right). The risk premium doubles the effective cost of edges adjacent to Bab el-Mandeb, increasing the cost of transiting the Red Sea corridor.}
\label{fig:security_shifter}
\end{figure}

\subsubsection{Cost and Distance Assumptions}
\label{sec:cost_assumptions}

Our scenario engine reports rerouting costs in \emph{km-equivalent} units, reflecting weighted shortest-path distances through the maritime network. To translate these into approximate economic magnitudes, we document all assumptions used in the cost mapping. Table~\ref{tab:cost_assumptions} lists each parameter, its value, units, and justification.

%This Table was regenerated by ChatGPT because I could not for the life of me get the dang table to work.

\begingroup
\setlength{\LTleft}{0pt}
\setlength{\LTright}{0pt}
\setlength{\tabcolsep}{4pt}        % tighter column padding
\renewcommand{\arraystretch}{1.20} % more row height so text doesn't overlap
\small
\sloppy                           % avoids overfull boxes from citations/long tokens

\begin{longtable}{@{}L{0.30\textwidth}C{0.14\textwidth}C{0.12\textwidth}L{0.44\textwidth}@{}}
\caption{Cost and Distance Assumptions}\label{tab:cost_assumptions}\\
\toprule
Parameter & Value & Units & Justification \& Source \\
\midrule
\endfirsthead

\toprule
Parameter & Value & Units & Justification \& Source \\
\midrule
\endhead

\midrule
\multicolumn{4}{r}{\small\emph{Continued on next page}} \\
\endfoot

\bottomrule
\endlastfoot

\textbf{Haversine distance} & Computed & km &
Great-circle distance between node coordinates; standard geodesic formula \citep{allen2022welfare} \\

\textbf{Congestion multiplier range} & $[1.0, 1.5]$ & dimensionless &
Calibrated from AIS density via log-linear mapping; upper bound consistent with port congestion estimates \citep{notteboom2006impact} \\

\textbf{Overland bypass: Bosporus (Constanta--Piraeus)} & $5.0\times$ haversine & km-equiv. &
Rail/truck intermodal alternative through southeastern Europe; overland freight typically 3--6$\times$ maritime cost \citep{notteboom2006impact} \\

\textbf{Fuel cost (bunker)} & \$400--600 & per tonne &
IFO 380 benchmark price range 2019--2023; IMO 2020 sulphur cap increased costs by 25--40\% for compliant fuels \citep{notteboom2006impact} \\

\textbf{Average fuel consumption} & 150--250 & tonnes/day &
Typical for Panamax--post-Panamax container vessels at design speed \citep{du2020port} \\

\textbf{Shipping cost per tonne-km} & \$5--10 & per 1{,}000 km &
Containerized cargo; range reflects vessel size, fuel prices, and route \citep{notteboom2006impact} \\

\textbf{War-risk insurance premium} & 0.1--0.5\% & of hull value &
Baseline range; can spike to 1--5\% in conflict zones \citep{besley2015welfare, oef2010economic} \\

\textbf{Canal transit tolls: Suez} & \$0.3--1.0M & per transit &
Varies by vessel size and cargo type; represents 2--8\% of total voyage cost for Asia--Europe routes \citep{cariou2021ais} \\

\textbf{Canal transit tolls: Panama} & \$0.2--0.8M & per transit &
New Panamax locks increased capacity but raised tolls \citep{notteboom2006impact} \\

\textbf{Average vessel speed} & 12--16 & knots &
Slow-steaming range for fuel efficiency \citep{du2020port} \\

\textbf{Daily time charter rate} & \$10{,}000--50{,}000 & per day &
Container vessel range depending on size and market conditions \citep{notteboom2006impact} \\

\textbf{Crew cost} & \$3{,}000--8{,}000 & per day &
All-in crew cost for 20--25 person complement \citep{oef2010economic} \\

\textbf{Risk premium scenarios} & 10--200\% & of base cost &
Applied to chokepoint-adjacent edges; upper range consistent with war-risk premium spikes observed during Gulf conflicts \citep{besley2015welfare} \\

\textbf{Partial degradation multiplier} & 1.25--10$\times$ & dimensionless &
Range captures mild congestion ($1.25\times$) through near-closure ($10\times$) \\

\textbf{Global seaborne trade volume} & $\sim$11 billion & tonnes/year &
UNCTAD 2023 estimate; used for order-of-magnitude cost translations \citep{verschuur2023systemic} \\

\end{longtable}
\endgroup



% \begin{longtable}{p{3.8cm}p{2.2cm}p{1.5cm}p{5.5cm}}
% \caption{Cost and Distance Assumptions} \label{tab:cost_assumptions} \\
% \toprule
% Parameter & Value & Units & Justification \& Source \\
% \midrule
% \endfirsthead
% \toprule
% Parameter & Value & Units & Justification \& Source \\
% \midrule
% \endhead
% \midrule
% \multicolumn{4}{r}{\small\emph{Continued on next page}} \\
% \endfoot
% \bottomrule
% \endlastfoot

% \textbf{Haversine distance} & Computed & km & Great-circle distance between node coordinates; standard geodesic formula \citep{allen2022welfare} \\

% \textbf{Congestion multiplier range} & $[1.0, 1.5]$ & dimensionless & Calibrated from AIS density via log-linear mapping; upper bound consistent with port congestion estimates \citep{notteboom2006impact} \\

% \textbf{Overland bypass: Bosporus (Constanta--Piraeus)} & $5.0 \times$ haversine & km-equiv. & Rail/truck intermodal alternative through southeastern Europe; overland freight typically 3--6$\times$ maritime cost \citep{notteboom2006impact} \\

% \textbf{Fuel cost (bunker)} & \$400--600 & per tonne & IFO 380 benchmark price range 2019--2023; IMO 2020 sulphur cap increased costs by 25--40\% for compliant fuels \citep{notteboom2006impact} \\

% \textbf{Average fuel consumption} & 150--250 & tonnes/day & Typical for Panamax--post-Panamax container vessels at design speed \citep{du2020port} \\

% \textbf{Shipping cost per tonne-km} & \$5--10 & per 1,000 km & Containerized cargo; range reflects vessel size, fuel prices, and route \citep{notteboom2006impact} \\

% \textbf{War-risk insurance premium} & 0.1--0.5\% & of hull value & Baseline range; can spike to 1--5\% in conflict zones \citep{besley2015welfare, oef2010economic} \\

% \textbf{Canal transit tolls: Suez} & \$0.3--1.0M & per transit & Varies by vessel size and cargo type; represents 2--8\% of total voyage cost for Asia--Europe routes \citep{cariou2021ais} \\

% \textbf{Canal transit tolls: Panama} & \$0.2--0.8M & per transit & New Panamax locks increased capacity but raised tolls \citep{notteboom2006impact} \\

% \textbf{Average vessel speed} & 12--16 & knots & Slow-steaming range for fuel efficiency \citep{du2020port} \\

% \textbf{Daily time charter rate} & \$10,000--50,000 & per day & Container vessel range depending on size and market conditions \citep{notteboom2006impact} \\

% \textbf{Crew cost} & \$3,000--8,000 & per day & All-in crew cost for 20--25 person complement \citep{oef2010economic} \\

% \textbf{Risk premium scenarios} & 10--200\% & of base cost & Applied to chokepoint-adjacent edges; upper range consistent with war-risk premium spikes observed during Gulf conflicts \citep{besley2015welfare} \\

% \textbf{Partial degradation multiplier} & 1.25--10$\times$ & dimensionless & Range captures mild congestion ($1.25\times$) through near-closure ($10\times$) \\

% \textbf{Global seaborne trade volume} & $\sim$11 billion & tonnes/year & UNCTAD 2023 estimate; used for order-of-magnitude cost translations \citep{verschuur2023systemic} \\
% \end{longtable}

\paragraph{Translating km-equivalent to economic costs.} A mean rerouting cost of $\Delta$ km-equivalent can be approximately converted to annual economic impact as:
\begin{equation}
  \text{Annual cost} \approx \Delta \times \frac{\text{Trade volume (tonnes)}}{\text{Average voyage distance}} \times \text{Cost per tonne-km},
\end{equation}
where we use illustrative values of \$5--10 per tonne per 1,000~km for containerized cargo and approximately 11 billion tonnes of global seaborne trade. These conversions are intended as order-of-magnitude estimates; precise economic costs depend on cargo composition, vessel types, demand elasticities, and market conditions that are beyond the scope of this analysis.

\subsubsection{Congestion Calibration from AIS Density}

We calibrate the congestion proxy $\Xi_e$ using the AIS density data. For each chokepoint $b$, we extract the total AIS position count within the bounding box (Section~3.1.2) and compute:
\begin{equation}
  \Xi_b = \sum_{c \in \text{BBox}(b)} \text{density}_c,
\end{equation}
where the sum is over all raster cells $c$ within the bounding box of chokepoint $b$. We then normalize these values to a $[1, 1.5]$ multiplier range using log-scaling:
\begin{equation}
  m_b^{\text{congestion}} = 1 + 0.5 \cdot \frac{\ln(1 + \Xi_b) - \min_b \ln(1 + \Xi_b)}{\max_b \ln(1 + \Xi_b) - \min_b \ln(1 + \Xi_b)}.
\end{equation}
This produces a mild congestion penalty (up to 50\%) for the most heavily trafficked chokepoints relative to the least trafficked, consistent with empirical estimates of congestion costs in port settings \citep{notteboom2006impact, du2020port}. Figure~\ref{fig:congestion_calibration} illustrates the calibration mapping from AIS intensity to the congestion cost multiplier.

\begin{figure}[H]
\centering
\includegraphics[width=\textwidth]{Figures/congestion_calibration.png}
\caption{Congestion calibration. Left: total AIS intensity (log scale) by chokepoint. Right: mapping from AIS intensity to the congestion multiplier $m_b^{\text{congestion}} \in [1, 1.5]$.}
\label{fig:congestion_calibration}
\end{figure}

\subsubsection{Route Choice and OD Trade Costs}

For each origin port $o$ and destination port $d$, the baseline trade cost is the shortest weighted path through the network:
\begin{equation}
  \tau_{od} = \min_{r \in \mathcal{R}_{od}} \sum_{e \in r} t_e,
  \label{eq:shortest_path}
\end{equation}
where $\mathcal{R}_{od}$ is the set of feasible routes (edge sequences) from $o$ to $d$. This corresponds to the $\theta \to \infty$ limit of the log-sum aggregator used in probabilistic route-choice models \citep{fajgelbaum2020optimal}. We use the Dijkstra shortest-path algorithm implemented in NetworkX to solve (\ref{eq:shortest_path}) for all 1,540 port pairs.

Figure~\ref{fig:route_choice} illustrates the route-choice mechanism by comparing the baseline optimal route from Shanghai to Rotterdam with the rerouted path under Suez Canal closure. Under normal conditions, traffic transits via the Strait of Malacca, Indian Ocean, Bab el-Mandeb, and Suez Canal. When Suez closes, the optimal route shifts to a trans-Pacific alternative via the North Pacific, Panama Canal, and North Atlantic---a dramatically longer but feasible alternative that the scenario engine quantifies.

\begin{figure}[H]
\centering
\includegraphics[width=\textwidth]{Figures/route_choice_before_after.png}
\caption{Route choice: Shanghai to Rotterdam. Left: baseline route via Malacca--Suez corridor. Right: rerouted path via North Pacific--Panama--North Atlantic after Suez Canal closure. Routes pass through ocean waypoints, following realistic maritime arcs.}
\label{fig:route_choice}
\end{figure}

\subsubsection{Scenario Design: Chokepoint Stress Tests}

The central empirical object is the distribution of counterfactual outcomes under disruptions to bottlenecks. We implement three classes of scenarios, using the Suez Canal as a worked example to illustrate each (Figure~\ref{fig:suez_scenarios}):

\paragraph{Scenario 1: Full closure.} For each chokepoint $b$, we remove it from the graph (all edges incident to $b$ are deleted) and recompute shortest paths for all 1,540 port pairs. The delta cost $\Delta \tau_{od}^{(b)} = \tau_{od}^{\text{shocked}} - \tau_{od}^{\text{baseline}}$ measures the rerouting penalty imposed by closure of $b$. For corridor chokepoints like the Suez Canal, closure forces traffic onto dramatically longer alternative routes (e.g., via the Cape of Good Hope or the Panama Canal), producing large rerouting costs. For bypass-dependent chokepoints like the Bosporus, closure forces traffic onto expensive overland alternatives, producing very large per-route costs. This scenario corresponds to a complete physical blockage (e.g., the 2021 Ever Given incident in the Suez Canal).

\paragraph{Scenario 2: Partial capacity degradation.} Rather than full closure, we increase the cost multiplier on chokepoint-adjacent edges by a factor $\alpha > 1$:
\begin{equation}
  t_e^{\text{shocked}} = \alpha \cdot t_e^{\text{baseline}}, \quad e \in \mathcal{E}(b),
\end{equation}
where $\mathcal{E}(b)$ is the set of edges incident to chokepoint $b$. We vary $\alpha \in \{1.25, 1.5, 2.0, 3.0, 5.0, 8.0, 10.0\}$ to trace out a ``severity curve'' for each chokepoint. This scenario captures congestion spikes, speed restrictions, or capacity reductions short of full closure.

\paragraph{Scenario 3: Security risk premium.} We model the effect of reduced security by adding a risk premium to a specific chokepoint's edges:
\begin{equation}
  m_e^{\text{risk}} = m_e^{\text{base}} \cdot (1 + \delta_{\text{risk}}),
\end{equation}
where $\delta_{\text{risk}} \in \{0.10, 0.20, 0.50, 1.00, 2.00\}$ represents an exogenous increase in risk (piracy premium, conflict escalation, insurance surcharge). We apply this premium to \emph{one chokepoint at a time}, producing a pure per-chokepoint vulnerability measure that avoids arbitrary categorization of chokepoints as ``high'' or ``low'' security.

\begin{figure}[H]
\centering
\includegraphics[width=\textwidth]{Figures/suez_scenario_3panel.png}
\caption{Suez Canal: three disruption scenarios. Left: full closure (Mediterranean--Indian Ocean traffic rerouted via Cape of Good Hope or Panama). Center: partial degradation ($\alpha = 3$, costs triple on Suez-adjacent edges). Right: risk premium (+100\%, costs double). The scenario illustrates the three classes of stress tests applied to all six chokepoints.}
\label{fig:suez_scenarios}
\end{figure}

\subsubsection{Output Metrics}

For each scenario, we report:
\begin{enumerate}[nosep]
  \item \textbf{Delta cost}: $\Delta \tau_{od}$ for all 1,540 port pairs, in absolute (km-equivalent) and percentage terms.
  \item \textbf{Mean delta cost}: averaged across port pairs, as a summary vulnerability index for each chokepoint.
  \item \textbf{Fraction affected}: the share of port pairs whose optimal route changes under the shock.
  \item \textbf{Per-port vulnerability}: for each port, the mean cost increase across all its trade routes when a given chokepoint is disrupted.
\end{enumerate}

\subsubsection{Sensitivity Analysis}

We conduct sensitivity analysis along three dimensions: (i) bounding box size (expanding/contracting chokepoint definitions by 20\%); (ii) congestion parameter ($\lambda$ varied from 0 to 0.3); and (iii) cost multiplier range for the congestion normalization. These checks ensure that results are not artifacts of arbitrary parameter choices.
