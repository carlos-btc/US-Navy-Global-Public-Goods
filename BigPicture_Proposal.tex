\documentclass{article}

\title{Big Picture Draft:}
\author{Carlos Carpi}
\date{2026-01-22}

\begin{document}


\maketitle
 

\section*{Proposed Title}
US Navy as Provider of Global Public Goods

*NOT MUCH TO UPDATE HERE PLEASE FEEL FREE TO IGNORE*

\section{Introduction \& Literature Review}

\subsection*{Context}
The US Navy provides global public goods and institutional infrastructure. This project seeks to help them with a suggestion on what data to collect to write the perfect paper that can estimate these global public goods provision.


\subsection*{Your Research Question}
Does U.S. Naval forward presence — via overseas basing, port calls, and freedom of navigation operations (FONOPs) — causally reduce maritime risk and lower transaction costs for global commerce through security spillovers?
\\

{\leftskip=2cm This project will develop a defensible, replicable, and Navy-usable framework for quantifying the “hegemonic dividend” — i.e., the extent to which U.S. naval power functions as a global public good that subsidizes host nation economies and the maritime system. \par}


\subsection*{What does the existing literature say}
The recent literature review by von Boemcken and Bolaños Suárez (2025) \cite{bibid} analyzes 140 studies to conclude that while military spending frequently impairs economic growth and increases public debt, these effects are highly context-dependent. Specifically, in high-income and arms-exporting countries, defense investments can generate positive externalities through technological spillovers, whereas in developing nations, they often crowd out welfare expenditure. Crucially for this project, the review identifies a theoretical "security channel"—where military spending creates a stable environment for development—but notes that this mechanism is rarely modeled empirically. This project directly addresses this gap by isolating the "security signal" of U.S. Naval presence, testing the review's hypothesis that military resources can function as an investment in security that benefits sustainable development.

\subsection*{Significance with respect to existing knowledge}
This project contributes to the literature by addressing the "maritime gap" identified in defense economics, where global commerce and naval presence remain largely underexplored despite their strategic importance. By "unbundling" the Imai et al. (2025) \cite{bibid} spatiotemporal framework, this research offers a novel methodological approach to quantifying the "hegemonic dividend"—the economic subsidy provided by U.S. security guarantees—moving beyond standard GDP-growth models to measure specific reductions in transaction costs and risk. Furthermore, it reframes the "guns versus butter" debate by treating naval power as a global public good that acts as institutional infrastructure for the global economy.


\section{Methods and Data (i.e. Research / Project Design Overview)}

\subsection*{Core Design Idea}
Use Imai et al \cite[]{} paper that used USAF data and move it to Sea / US Navy “Unbundle” the Imai et al. (2025) spatiotemporal causal inference framework (originally designed for insurgent violence and airstrikes) and redeploy it in a maritime setting with arbitrary security spillovers across sea lanes.
\\
Put simply, apply the same mathematical rigor to Arbitrary Spillover in the Global Commons

\subsection*{State analytical method and justify}
Estimate whether Naval presence produces statistically detectable reductions in maritime risk, shipping delays, insurance risk, and trade friction.

\subsubsection*{Modeling Approach:}
•	Unit = maritime raster grid OR chokepoint polygon
•	Treatment = Naval presence intensity
•	Y = shipping density / connectivity / insurance proxies
•	Spillovers = global public good externalities
•	Interference = non-SUTVA by design (key feature)
\subsubsection*{Secondary Designs (if data constraints persist):}
•	Event studies (FONOPs, basing agreements)
•	DiD with host nations
•	Synthetic control for chokepoint disruptions
•	Interrupted time series (COVID precedent in synthesis) 


\subsection*{State data/approach and justify}
The data is the main constraint for this potential paper, so this proposal limits itself to mapping / schematizing some of the main sources of data that would need to be collected / accessed / integrated to create a model. There a potential simulation study to prove the feasibility of the concept should the data constraints prove not too restricting. \\
\subsubsection*{\indent Existing Data (Public or Semi-Public):}

There are many data sources and datasets described below, bundled by the nature of their usage in this project. 

\subsubsection*{\indent \indent \textit{ Outcome Variables (Y): }}

\begin{itemize}
	\item Global Shipping Density (AIS-based; World Bank Data Catalog) 
	\item Liner Shipping Connectivity Index (UNCTAD/World Bank) 
	\item Port throughput / maritime transport indicators (World Bank) 
	\item Potential: Maritime insurance premiums (Lloyd’s/industry — not yet accessed)
	\item Global Insurance premiums (TBD, but there are papers on this \cite{bibid})

\end{itemize}

\subsubsection*{\indent \indent \textit{ Treatment Variables (D):}}
\begin{itemize}
	\item U.S. Naval Basing Footprint (spatial point + buffering)
	\item Freedom of Navigation Operations (FONOP event data)
	\item Port Calls / Exercises (if open)
	\item Host Nation Security Agreements
	\item US Navy Base locations \\


\end{itemize}
\subsubsection*{\indent \indent \textit{ Potential Moderators / Covariates:}}
\begin{itemize}
	\item Threat environment (piracy incidents, chokepoints, geopolitical disputes)
	\item Local economic indicators (FDI, trade openness, port ranking)
	\item Naval force posture differences (blue-water vs littoral
\\
\end{itemize}
\subsubsection*{\indent \indent \textit{ Classified / Not Directly Accessible (but relevant):}}
\begin{itemize}
	\item Deployment timing granularity
	\item Mission-level logistics
	\item ISR data (Maritime Domain Awareness)
	\item Operational costs (likely a VAMOSC integration) \\
\end{itemize}

\subsubsection*{\indent Data Scraping Strategy:}
\subsubsection*{\indent \indent \textit{ Targets:}}
\begin{itemize}
	\item AIS Maritime Traffic (2012→present)
	\item Naval event reporting 
	\subitem DoD / DoW Press releases, 
	\subitem Congressional CRS reports, 
	\subitem Lowy FONOP tracker
	\item Maritime Insurance / Underwriting (Lloyd’s)
	\item Satellite data on route chokepoints based on trade intensity (Most important after conversation with Professor Rossi-Hansberg on Friday January 23rd)
\end{itemize}

\subsubsection*{\indent \indent \textit{ Scraping / API Access Tools:}}
\begin{itemize}
	\item Python + AIS APIs (MarineTraffic / Spire / Orbcomm)
	\item Web scraping from defense press releases
	\item Lowy Institute FONOP archive (already identified)
	\item World Bank Data API
\end{itemize}
\subsubsection*{\indent \indent \textit{ Broad data meta characteristics: }}
\begin{itemize}
	\item Temporal Cadence: Monthly
	\subitem Minimum workable: monthly
	\subitem Ideal: daily/weekly (drives spatiotemporal inference)
	\item Spatial Cadence:
	\subitem Rasterized grids (Imai et al. style)
	\subitem OR maritime chokepoint polygons
\end{itemize}


\section{Feasibility}
There are many concerns, most of them related to data. Broadly, this project is a proposal that acknowledges data deadlock for deployments and proposes pivot to data collection proposal for the US Navy rather than extraction. Insurance market route appears feasible and has strong historical depth (proposal track argues trillions-scale sensitivity). After conversation with Professor Rossi-Hansberg on Friday, January 23rd, it seems the main route might be to interview market participants and maybe navy personnel to map the main global bottlenecks and move even further towards understanding commercial traffic flow patterns above the direct effect of Us Navy intervention in any given bottleneck. Most importantly, AIS + World Bank/IMF data confirmed accessible. 
\\ \\
That said, there are Technical Feasibility Questions (AIS scale computationally heavy but manageable and	Spatiotemporal models compute in R (geocausal library) and Political/Institutional Feasibility ( given the sensitive nature of military deployment data, this is obvious, but 3 outreach paths are already identified to Navy/Marines.
\\ 
\begin{itemize}
	\item Path 1: Reaching out to folks I previously worked with/under who happen to be in the US Navy \\
	\item Path 2: Reaching out to the folks at the Marine War Fighting Lab \\ https://www.mcwl.marines.mil/ \\
	\item Path 3: Gemini suggested contacting the Acquisition Research Program (ARP) at the Naval Postgraduate School. Contact Info: arp@nps.edu | (831) 656-3793. They appear to be seeking external PhD-level research to justify Theater Security Cooperation budgets.
\end{itemize}
\vspace{10pt}

For the next meeting with Professor Rossi-Hansberg, ask the professor if any scholars in the space of spatial economics and defense space, and ask about PAPERS he looked at recently that he liked in space.  

\vspace{10pt}

\subsection*{Evaluation of approach w.r.t. RQ/project goal}
Could not identify any major conceptual mismatch. Method fit: Spillovers + interference → perfect match for Imai approach.
Checked other methods with professor Rossi-hansberg, and he suggests simpler model that uses scenarios and simulations around maritine trade bottlenecks that could prompt / force alternative routing. Reaching out to shipping companies contacts might be important here. \cite[add. text]{keylist}
\\

After meeting with Professor Rossi-Hansberg, it seems current route or his proposed route both have Substantive fit: Global trade + shipping = spatial systems with exogenous chokepoints.There seems to also be novelty fit: Synthesis matrix confirms maritime security is a missing domain in literature. 

Finally, there is also professional fit: Path to sponsor + Navy usage = clearly articulated. Future notes should indentify connections with maritime industry players (ship brokers in Germany, some folks on the global logistics side: 
\begin{itemize}
	\item LATAM: CSAV
	\item North EUROPE and Mediterranean: Hamburg Sud
	\item Asia: TBC, likely Mersk Sealand
	\item Oceania: TBD
	\item Africa an: TBD
\end{itemize}). 
\\

\subsection*{Initial Results (or Mock-up)}
Will likely map gross traffic flows from-to regional pars to create broad-strokes idea of what is worth what. Map US Navy Fleets / download their own subsections to understand different sub-commands and who to reach-out to for what. Ideally, map sub-space enough to understand exactly what are the important questions to get from each player. 

\subsection*{Proposed timeline}
Will follow timeline as proposed below. Ultimately, expect to have first draft out by end of January, and Final draft is done by end of February so that interviews with different market/military thought leaders can proceed smoothly and provide concrete feedback for re-writes until mid-March bingo. \\
\subsection*{Securing an Advisor/sponsor}
\subsubsection*{\indent UCh Professor Paul Poast (Poli Sci / IR):}

The next meeting is soon to be scheduled, but as mentioned in Office Hours, it is an ongoing issue. However, Daniel has agreed to help.

\subsubsection*{\indent Previous Notes for Reference:}


Agreed to mentor the project and had productive first meeting. Will use his cost-benefit analysis of US Alliances as basis for theory. Need to schedule second meeting in mid-February. Will email for a time in February this week, but must reach out to Hagel Lecture personnel to support bringing World Bank President to campus (might be only 2027 - TBD) \\  \\
He works on many related fields. 
•	Military strategy
•	IR + hegemony
•	Defense policy and basing rights angle
•	Fit due to Posen-style framing of “command of the commons” (synthesis) 
•	Seemed interested in the idea of using insurance premiums data for not just this paper. \\ \\ Must review NDS 2026 asap \cite[add. text]{keylist} \\

\subsubsection*{\indent UCh Professor Rossi-Hansberg (Econ, Spatial Economics):}

The pivot is on-going, and a meeting is scheduled for Friday to follow up.

\subsubsection*{\indent Previous Notes for Reference:}


First meeting went well. Suggested a pivot / focus on bottle necks of maritime traffic with a scenario analysis. 
\\
Professor Rossi-Hansberg is available for future meetings. Send update asap and schedule meeting for February.
\\ \\
Some relevant notes in no particular order: •	At core, this needs to be an ECONOMICS paper, NOT a defense/government piece
•	Deep knowledge of spatial economics 
•	Seemed open to the ideas, and was very patient on the project so far. Great meeting January 23rd, Friday. 
\\
\subsubsection*{\indent Kosuke Imai (Poli Sci methods / Stats):}
Postponed until pivot is complete.
\\

\subsubsection*{\indent External Institutional Sponsors (Professional Thesis Option):}
•	Naval Postgraduate School ARP (already identified) 
•	Marine Warfighting Lab (identified)
•	US Navy @RI - US Naval War College (identified, original TTL)
•	Navy economist-planners (TBD)


\subsection*{Cost and funding}
Minimal cost expected; However, funding, if it comes will likely be from:
\begin{itemize}
	\item DoD Grants
	\item World Bank / IMF grant for data collection of maritime data already there
	\item Governments (state and local inside US ONLY)
	\item Within UCh (CIR, among others)
	\subitem CIR
	\subitem UCh corp
	\subitem Hagel Lecture folks
	\item NGOs
	\subitem The Chicago Council on Global Affairs
	\subitem Smith Richardson Foundation
	\subitem RAND
	\subitem Alfred P. Sloan Foundation / MIT folks
	\subitem MacArthur Foundation
	\subitem IDA (defense not WBG)
\end{itemize}


\section*{Overall structure and alignment}
The theoretical framework and methodological design are well-aligned, specifically the application of spatiotemporal causal inference to capture the arbitrary spillovers of maritime security. However, the project will need to refine the data strategy following the pivot from data extraction to a "data collection proposal," particularly regarding classified deployment granularity. In the coming weeks, the goal is to examine additional literature on maritime insurance markets to validate risk proxies and integrate Professor Rossi-Hansberg’s feedback on using scenario analyses of maritime bottlenecks to robustly model trade flow alternatives.


\end{document}