\documentclass[12pt]{article}
%\usepackage{graphicx} % Required for inserting images
%\usepackage[style=apa]{biblatex} % Feel free to change the style to another one you prefer
%\addbibresource{Bibliography/references.bib}  
% ── Packages ──────────────────────────────────────────────
\usepackage[margin=1in]{geometry}
\usepackage{graphicx}
\usepackage{booktabs}
\usepackage{caption}
\usepackage{subcaption}
\usepackage{amsmath,amssymb}
\usepackage[breaklinks=true,colorlinks=true,linkcolor=blue,citecolor=blue,urlcolor=blue]{hyperref}
\usepackage{natbib}
\usepackage{setspace}
\usepackage{float}
\usepackage{longtable}
\usepackage{enumitem}
\usepackage{booktabs,longtable,array,ragged2e}

% Better p-columns: wrap + ragged right + allow \\ at end of cells
\newcolumntype{L}[1]{>{\RaggedRight\arraybackslash}p{#1}}
\newcolumntype{C}[1]{>{\Centering\arraybackslash}p{#1}}

% ── Bibliography ──────────────────────────────────────────
\bibliographystyle{apalike}


% ── Title ─────────────────────────────────────────────────
\title{Maritime Security as a Global Public Good:\\
       Trade Frictions, Chokepoint Vulnerability, and a\\
       Scenario Engine Using AIS Shipping Density}
\author{Carlos}
\date{\today}

\begin{document}

\maketitle

\begin{abstract}
\noindent
This paper frames maritime security as a global public good and evaluates how disruptions at maritime bottlenecks exacerbate trade frictions by rerouting and causing congestion. Using high-resolution AIS ship-density rasters from the IMF World Seaborne Trade Monitoring System \citep{cerdeiro2020wstms}, we construct a 78-node maritime transport network---56 major ports, 6 chokepoints, and 16 ocean waypoints connected by approximately 155 edges following real-world shipping corridors---and implement a scenario engine that simulates the consequences of bottleneck closures, capacity reductions, and risk spikes across 1,540 port-to-port trade routes. Under full closure scenarios, the Panama Canal (mean rerouting cost of 17,840 km-equivalent), Gibraltar (10,624 km), and the Suez Canal (9,949 km) generate the largest rerouting costs. Bab el-Mandeb (4,700 km) and the Bosporus (3,275 km) force traffic onto longer maritime alternatives or expensive overland bypasses at 5$\times$ normal cost. A partial-capacity degradation exercise and a per-chokepoint security risk analysis demonstrate that even modest disruptions produce economically meaningful increases in trade costs, and that the marginal value of security provision increases with risk. We estimate daily rerouting costs of chokepoint closure ranging from \$16 million (Bab el-Mandeb) to \$233 million (Gibraltar), implying aggregate annual costs on the order of tens of billions of dollars. These findings provide a replicable, data-grounded framework for quantifying the ``hegemonic dividend'' — the economic value of the provision of systemic global security in the maritime commons of the status-quo.

\end{abstract} \hspace{10pt}

\onehalfspacing

\section{Introduction}
% Introduction.tex — Revised for narrative cohesion

Global commerce is overwhelmingly maritime. Approximately 80 percent of world trade by volume and over 70 percent by value traverses the oceans, channeled through a small number of physical bottlenecks---straits, canals, and narrow sea lanes---that concentrate traffic and vulnerability \citep{verschuur2023systemic, bueger2024securing}. The reliability of these corridors is not a natural given; it depends on the provision of security, navigational infrastructure, and rules-based governance that together constitute the ``global maritime commons'' \citep{bueger2019maritime, kraska2015law}. When this security breaks down---through piracy, conflict-related closures, or geopolitical risk---the consequences propagate rapidly through rerouting, congestion, and elevated trade costs \citep{besley2015welfare, notteboom2006impact}.

The U.S.\ Navy has historically been the predominant provider of security in the global commons, maintaining forward presence through overseas basing, freedom of navigation operations, and patrol activities that deter threats and ensure the openness of critical sea lanes. This role can be framed as the provision of a \emph{global public good}: security that is non-rival and largely non-excludable \citep{bueger2024securing, mbekeani2011piracy}. The economic value of this public good---what we term the ``hegemonic dividend''---remains poorly quantified, in part because the most policy-relevant outcomes (trade reliability, routing resilience, avoided cost spikes) are \emph{equilibrium objects} of a transportation network that require explicit modeling of route substitution.

This paper asks: \emph{How sensitive is the global maritime trading system to disruptions at critical chokepoints, and what is the economic value of security provision that prevents or mitigates such disruptions?} Figure~\ref{fig:intro_quicklook} provides the motivating empirical context: global AIS ship-density data reveal that maritime traffic is extraordinarily concentrated along a small number of corridors, funneling through physical bottlenecks where disruption risk is highest.

\begin{figure}[H]
\centering
\includegraphics[width=\textwidth]{Figures/generated/global_quicklook.png}
\caption{Global AIS ship-density raster (log scale), aggregating vessel position reports from January 2015 to February 2021. Traffic concentrates along established corridors and funnels through a small number of physical chokepoints. Data from the IMF World Seaborne Trade Monitoring System \citep{cerdeiro2020wstms}.}
\label{fig:intro_quicklook}
\end{figure}

We address this question through a scenario-based quantitative spatial framework rather than through direct causal estimation of naval operations. This choice is motivated by two considerations. First, granular data on naval deployments and operational details are largely classified and unavailable for academic research. Second, even with such data, the most informative exercise for policymakers is counterfactual: \emph{what would happen if a chokepoint were disrupted, and how much would enhanced security reduce the losses?} This framing aligns with the ``transportation networks'' approach in quantitative spatial economics \citep{fajgelbaum2020optimal, allen2022welfare}, which represents trade flows on a graph with endogenous route choice and congestion, and evaluates counterfactuals by shocking network elements and recomputing equilibrium.

Our empirical strategy proceeds in three stages. First, we use high-resolution AIS ship-density rasters from the IMF World Seaborne Trade Monitoring System \citep{cerdeiro2020wstms} to construct a baseline portrait of global shipping intensity and to measure the concentration of maritime activity at six major chokepoints. Second, we build a stylized maritime transport network---a weighted graph of 78 nodes (56~ports, 6~chokepoints, 16~ocean waypoints) connected by approximately 155 edges---in which edge costs depend on distance and a congestion proxy calibrated from the AIS density data. Third, we implement a scenario engine that imposes exogenous shocks to chokepoint edges (closure, capacity reduction, risk spike) and computes the resulting changes in least-cost path lengths, rerouting patterns, and aggregate trade-cost indices across 1,540 port-to-port routes. Security enters the model as a cost-reducing and risk-reducing shifter on bottleneck-adjacent edges.

The paper makes three contributions. First, it provides a replicable, data-grounded measurement framework for evaluating chokepoint vulnerability using publicly available AIS data. Second, the scenario engine translates the qualitative intuition that ``chokepoints matter'' into quantitative cost-change estimates that can be compared across bottlenecks and shock magnitudes---directly useful for security resource allocation. Third, by situating the analysis within the quantitative spatial economics toolkit \citep{redding2017quantitative, donaldson2025transport}, we connect the maritime security literature to a rigorous modeling tradition that emphasizes general-equilibrium effects, welfare measurement, and network structure.

We are explicit about the limitations of the current exercise. With only an aggregate density raster (not vessel-level trajectories or bilateral trade flows), we cannot estimate a full spatial equilibrium with origin-destination demand or compute welfare in the general-equilibrium sense of \citet{fajgelbaum2020optimal}. The network model is therefore stylized: it uses great-circle distances for edge weights, a congestion proxy derived from raster intensity, and shortest-path routing. We label all scenario results as partial-exercise estimates and interpret them as informative bounds on the sensitivity of the maritime system to disruption. A key output of the paper is a concrete specification of what additional data would be needed to move from scenario simulation to causal inference.

The remainder of the paper is organized as follows. Section~\ref{sec:literature} reviews the related literature. Section~\ref{sec:data_methods} describes the data and methods. Section~\ref{sec:results} presents the results. Section~\ref{sec:discussion} discusses the implications and limitations. Section~\ref{sec:conclusion} concludes.


\section{Literature Review}
% LitReview.tex — Revised: shortened from 7 subsections to 4, tangential material removed
% All citations verified against Bibliography/references.bib and Readings/ folder

This section situates the paper within four literatures: maritime security as a public good; trade frictions and maritime disruptions; quantitative spatial models with transportation networks; and AIS-based measurement in maritime economics.

\subsection{Maritime Security as a Public Good}

The concept of maritime security as a global public good is central to both international relations and defense economics. \citet{bueger2019maritime} argue that the ``politics of the global sea'' are fundamentally under-theorized relative to their economic importance, while \citet{bueger2024securing} provide a comprehensive assessment establishing that global trade, energy security, and food security all depend on safe oceans---security that exhibits classic public-good characteristics of non-rivalry and partial non-excludability. The legal foundations are grounded in the international law of the sea: \citet{kraska2015law} details the framework governing freedom of navigation, passage through straits, and the rights of naval forces in international waters.

The economic costs of maritime insecurity have been documented in several contexts. \citet{mbekeani2011piracy} estimate the direct economic impacts of piracy for African development, including higher insurance costs and rerouting expenses. The One Earth Future Foundation report \citep{oef2010economic} provides a comprehensive accounting of piracy-related costs---ransoms, insurance, naval patrols, and supply-chain disruptions---estimating annual global costs in the billions of dollars.

\subsection{Trade Frictions and Maritime Disruptions}

\citet{besley2015welfare} provide a landmark study of the welfare costs of lawlessness in the context of Somali piracy, establishing a direct empirical link between maritime insecurity and shipping costs by exploiting spatial and temporal variation in piracy risk. Disruptions at chokepoints represent a particularly consequential form of trade friction because they affect the entire network structure rather than individual routes. \citet{verschuur2023systemic} analyze the systemic economic impacts of disruptions at 24 major maritime chokepoints, estimating potentially large global losses and providing strong motivation for a scenario-based modeling approach.

The port congestion literature offers complementary evidence on how capacity constraints translate into economic costs. \citet{notteboom2006impact} demonstrates that congestion at key nodes produces cost increases rippling through the shipping industry, while \citet{du2020port} document the strategies that ports employ to manage delays. \citet{morabito2023thesis} provides institutional detail on maritime transport economics that contextualizes network models within the operational realities of the shipping industry.

\subsection{Quantitative Spatial Economics and Transportation Networks}

The methodological backbone of this paper is the quantitative spatial economics framework. \citet{redding2017quantitative} provide the foundational survey, establishing the canonical framework in which locations differ in productivity and amenities, trade costs govern the flow of goods, and equilibrium outcomes depend on the entire spatial distribution of economic activity. \citet{allen2025quantitative} and \citet{redding2025urban} extend this framework to regional and urban analysis, developing the vocabulary of iceberg trade costs, gravity-based flows, and hat-algebra counterfactuals that we adapt to the maritime setting. \citet{proost2019what} emphasize the field's ability to evaluate policies with spatial dimensions---a category that naturally includes maritime security.

Within this tradition, the transportation networks literature provides our core analytical engine. \citet{fajgelbaum2020optimal} develop the foundational framework for optimal transport networks in spatial equilibrium, demonstrating that optimal networks concentrate investment on high-traffic corridors---a finding with direct parallels to maritime chokepoints. \citet{allen2022welfare} develop a quantitative model with endogenous traffic congestion and apply it to the U.S.\ highway network, providing a close methodological parallel to our maritime cost specification. \citet{donaldson2025transport} offers a comprehensive survey of transport infrastructure evaluation, emphasizing the distinction between identified causal effects and model-based counterfactuals that directly motivates our approach. \citet{hierons2024spreading} studies optimal congestion pricing in general equilibrium, providing guidance for the security allocation extension of our model. \citet{bordeu2025commuting} examines infrastructure investment under political fragmentation, highlighting coordination problems analogous to the provision of global maritime security when international cooperation is incomplete. \citet{rossihansberg2020transportation} provides a compact analytical treatment of transportation network models that highlights the key features delivering tractability: sparse network structure, congestion-based costs, and shortest-path routing---features directly incorporated into our maritime network specification.

\subsection{AIS Data in Maritime Economics}

The empirical implementation relies on Automatic Identification System (AIS) data, which have become a cornerstone of modern maritime economics. \citet{kerbl2022ais} provides an overview of AIS applications in economic research, documenting both opportunities and challenges. \citet{cariou2021ais} use AIS data to study the economic impact of the COVID-19 shock on container flows and ship waiting times, illustrating how AIS supports real-time disruption measurement. \citet{cerdeiro2020nowcasting} combine AIS data with machine learning to nowcast world trade, demonstrating the predictive content of shipping movements for macroeconomic outcomes. Together with the IMF World Seaborne Trade Monitoring System \citep{cerdeiro2020wstms} that provides our primary dataset, these papers establish the feasibility of constructing traffic flows, congestion proxies, and intensity measures from AIS data.

\subsection*{Summary}

The existing literature provides a clear foundation: the quantitative spatial economics toolkit \citep{redding2017quantitative, allen2025quantitative} supplies the theoretical structure; transportation network models \citep{fajgelbaum2020optimal, allen2022welfare} provide the analytical engine; and AIS data \citep{kerbl2022ais, cerdeiro2020wstms} enable empirical measurement. Yet a gap remains: existing work rarely combines a global maritime network model with endogenous rerouting and congestion to value security provision under counterfactual disruptions. The chokepoint vulnerability literature \citep{verschuur2023systemic} establishes the stakes but relies on reduced-form methods rather than network-equilibrium counterfactuals. The piracy literature \citep{besley2015welfare, oef2010economic} quantifies costs but treats routes as fixed. Our paper bridges this gap by combining insights from the transportation networks tradition with AIS-based measurement and a flexible scenario framework.


\section{Data and Methods}



\subsection{Data}

\subsubsection{AIS Ship Density Raster}

Our primary data source is the global ship density raster from the IMF World Seaborne Trade Monitoring System \citep{cerdeiro2020wstms}. The dataset aggregates AIS (Automatic Identification System) position reports received between January 2015 and February 2021 into a global raster grid at 0.005\textdegree{} $\times$ 0.005\textdegree{} resolution (approximately 500m $\times$ 500m at the equator). Each cell records the total number of AIS positions reported within that cell over the entire period, covering both moving and stationary vessels. The raster thus measures the \emph{intensity of shipping activity} at each location---a proxy for the cumulative traffic load experienced by each point in the global ocean.

The dataset is distributed as a single GeoTIFF file (approximately 9~GB) with pre-computed overview pyramids (approximately 3~GB). The raster covers the full global extent in WGS84 geographic coordinates. Given its size, all processing uses memory-safe techniques: built-in overviews for global visualization, windowed reads for chokepoint-level extraction, and aggressive downsampling (to 3600 $\times$ 1800 pixels) for summary statistics. The AIS data underlying the raster have been validated for use in economic research; see \citet{kerbl2022ais} for a discussion of AIS data quality, coverage, and applications, and \citet{cerdeiro2020nowcasting} for the IMF's methodology for constructing trade-relevant indicators from AIS.

\subsubsection{Chokepoint Definitions}

We define six maritime chokepoints as coarse bounding boxes in geographic coordinates (Table~\ref{tab:chokepoint_definitions}). Each bounding box encloses the navigable channel and approach lanes of a major maritime bottleneck. We extract the AIS density within each bounding box via windowed raster reads and compute aggregate intensity statistics. The six chokepoints are:

\begin{enumerate}[nosep]
  \item \textbf{Suez Canal} --- connecting the Mediterranean and Red Sea;
  \item \textbf{Bab el-Mandeb} --- southern entrance to the Red Sea;
  \item \textbf{Strait of Malacca} --- connecting the Indian Ocean and South China Sea;
  \item \textbf{Panama Canal} --- connecting the Atlantic and Pacific Oceans;
  \item \textbf{Bosporus} --- connecting the Black Sea and Mediterranean;
  \item \textbf{Strait of Gibraltar} --- entrance to the Mediterranean from the Atlantic.
\end{enumerate}

These chokepoints were selected based on their prominence in the maritime security literature \citep{verschuur2023systemic, bueger2024securing} and their role as physical bottlenecks that concentrate traffic and create vulnerability. The Strait of Hormuz, the Cape of Good Hope, and the Danish Straits are included in the network as waypoint nodes (enabling realistic routing) but are not classified as chokepoints for the closure analysis: the Strait of Hormuz functions as a critical but open passage without the canal-like physical constraints of Suez or Panama; the Cape of Good Hope is an open-ocean alternative route; and the Danish Straits do not constitute a narrow mandatory-passage bottleneck for modern shipping. The bounding boxes are deliberately coarse to ensure computational tractability and to capture approach-lane traffic in addition to the narrowest passage.

\begin{table}[H]
\centering
\caption{Chokepoint Bounding Box Definitions}
\label{tab:chokepoint_definitions}
\small
\begin{tabular}{lcccc}
\toprule
Chokepoint & Lon Min & Lon Max & Lat Min & Lat Max \\
\midrule
Suez Canal        & 32.20 & 32.65 & 29.80 & 31.30 \\
Bab el-Mandeb     & 42.25 & 43.55 & 12.15 & 13.25 \\
Strait of Malacca & 99.00 & 104.80 & 1.00 & 6.50 \\
Panama Canal      & $-$80.20 & $-$79.40 & 8.70 & 9.60 \\
Bosporus          & 28.90 & 29.30 & 41.05 & 41.25 \\
Gibraltar         & $-$6.20 & $-$5.20 & 35.80 & 36.30 \\
\bottomrule
\end{tabular}
\end{table}

Figure~\ref{fig:bbox_example} illustrates the bounding box extraction procedure using the Suez Canal as an example. The bounding box captures the canal and its approach lanes on the Mediterranean and Red Sea sides, where vessel traffic concentrates into a high-density corridor. The full set of six bounding boxes is shown on a world map in Appendix~A (Figure~\ref{fig:appendix_all_bboxes}).

\begin{figure}[H]
\centering
\includegraphics[width=0.85\textwidth]{Figures/bounding_box_example.png}
\caption{AIS ship density at the Suez Canal with the chokepoint bounding box overlaid (dashed cyan). The heatmap shows log-transformed AIS position counts; brighter areas indicate higher traffic intensity.}
\label{fig:bbox_example}
\end{figure}

\subsection{Methods}

\subsubsection{Empirical Strategy: Bottleneck Scenario Analysis on a Maritime Transport Network}

The central challenge in quantifying maritime security as a global public good is that the most policy-relevant outcomes---trade reliability, routing resilience, and avoided trade-cost spikes---are \emph{equilibrium objects} of a transportation network \citep{fajgelbaum2020optimal}. When a chokepoint is disrupted through conflict risk, congestion, or physical closure, global shipments reroute endogenously. Any credible evaluation must therefore model \emph{trade-flow substitution across routes} rather than treat observed routes as fixed.

Following the transportation networks framework \citep{allen2022welfare, fajgelbaum2020optimal}, we build a quantitative spatial model of global maritime shipping on a graph in which: (i) shipments choose routes through the network; (ii) link-level costs depend on distance, traffic (congestion), and link quality/capacity; and (iii) counterfactuals are conducted by shocking bottleneck links and recomputing equilibrium flows and delivered trade costs. The Navy's role as a global public good provider is introduced as a cost-reducing and risk-reducing shifter on bottleneck-adjacent edges: higher security provision lowers effective iceberg trade costs and improves reliability \citep{besley2015welfare, bueger2024securing}.

\subsubsection{Network Construction}

We represent the global maritime system as an undirected weighted graph $G = (\mathcal{J}, \mathcal{E})$ with 78 nodes and approximately 155 edges:

\begin{itemize}[nosep]
  \item \textbf{Port nodes} (56): Major container and bulk ports worldwide, including Shanghai, Singapore, Rotterdam, New York, Dubai, Santos, and 50 others, selected to represent the major origin-destination endpoints of global maritime trade. Each port is placed at its real-world geographic coordinates.
  \item \textbf{Chokepoint nodes} (6): The six maritime bottlenecks defined in Section~3.1.2, placed at the center of their respective bounding boxes.
  \item \textbf{Waypoint nodes} (16): Open-ocean routing waypoints---including the Strait of Hormuz, North Pacific, North Atlantic, Central Pacific, Caribbean Sea, Arabian Sea, Bay of Bengal, South China Sea, East Mediterranean, Norwegian Sea, Cape of Good Hope, Lombok Strait, Aden Junction, Mozambique Channel, West Africa, and South Atlantic---that ensure realistic path geometry through intermediate waters. Trans-ocean and inter-regional edges pass through these waypoints so that computed routes follow realistic maritime arcs rather than great-circle paths across continents.
  \item \textbf{Edges} $\mathcal{E}$ ($\sim$155): Feasible maritime connections following known shipping corridors. Edge costs are proportional to great-circle distances between endpoint coordinates, with optional cost multipliers for overland bypass routes.
\end{itemize}

A critical design feature is the \emph{bypass-dependent topology}: Black Sea ports (Constanta, Odessa, Novorossiysk) connect primarily through the \textbf{Bosporus}, with an overland bypass via southeastern European rail/road corridors (Constanta--Piraeus) at approximately 5 times the normal cost. This design ensures that Bosporus closure produces \emph{large but finite} rerouting costs---reflecting the existence of extremely costly real-world alternatives---rather than complete disconnection. Persian Gulf ports connect through the Strait of Hormuz waypoint to the Arabian Sea, reflecting their geographic position but without a mandatory-passage constraint in the closure analysis. Baltic ports (St.~Petersburg, Gdansk, Stockholm, Helsinki) connect directly to Northwestern European ports. Indian Ocean--Pacific traffic must transit through either the Strait of Malacca or the Lombok Strait alternative south of Indonesia, consistent with real-world shipping patterns \citep{verschuur2023systemic}.

The 56 ports generate $\binom{56}{2} = 1{,}540$ unique origin-destination pairs, enabling granular analysis of how chokepoint disruptions propagate across specific trade routes. Figure~\ref{fig:network_map} shows the full network overlaid on a world map, with overland bypass edges shown as dashed lines.

\begin{figure}[H]
\centering
\includegraphics[width=\textwidth]{Figures/network_world_map.png}
\caption{Maritime transport network: 78 nodes (56 ports, 6 chokepoints, 16 waypoints) connected by approximately 155 edges following major shipping corridors. Red diamonds mark chokepoints; blue circles mark ports; gray squares mark ocean waypoints. Dashed brown lines indicate overland bypass routes with elevated cost multipliers.}
\label{fig:network_map}
\end{figure}

\subsubsection{Edge Cost Specification}

Let $t_e \geq 1$ denote the generalized cost of traversing edge $e$. Inspired by congestion-based network models \citep{allen2022welfare, hierons2024spreading}, we parameterize:
\begin{equation}
  t_e = \bar{t}_e \cdot \left(\Xi_e\right)^{\lambda},
  \label{eq:congestion}
\end{equation}
where $\Xi_e$ is a traffic proxy on edge $e$ (derived from AIS density at the adjacent chokepoint) and $\lambda \geq 0$ captures congestion---the degree to which costs are increasing in traffic intensity. The baseline component $\bar{t}_e$ depends on great-circle distance between the endpoints of edge $e$:
\begin{equation}
  \bar{t}_e = \text{dist}_e \cdot m_e,
  \label{eq:baseline_cost}
\end{equation}
where $m_e$ is an edge-specific multiplier that captures physical constraints (narrow channels, mandatory traffic separation) and security/risk conditions.

\paragraph{Security as a cost shifter.} We model security provision as a reduction in the effective cost multiplier on bottleneck-adjacent edges:
\begin{equation}
  m_e = m_e^{\text{base}} \cdot (1 + \delta_{\text{risk}} \cdot \text{Risk}_e) \cdot (1 - \delta_{\text{sec}} \cdot S_e),
  \label{eq:security_cost}
\end{equation}
where $\text{Risk}_e$ is a risk indicator (e.g., piracy incidence, conflict proximity), $S_e$ is security intensity (naval presence, patrol frequency), and $\delta_{\text{risk}}, \delta_{\text{sec}} > 0$ are parameters governing the sensitivity of costs to risk and security. This formulation follows \citet{besley2015welfare} in treating insecurity as a trade-cost shifter, and extends it by making security endogenous to the cost structure. In the absence of direct naval presence data, we treat the security parameters as sensitivity analysis inputs and report results across a range of plausible values.

Figure~\ref{fig:edge_cost} illustrates the edge cost structure along the Suez--Red Sea corridor, showing how great-circle distances between nodes determine baseline costs.

\begin{figure}[H]
\centering
\includegraphics[width=0.85\textwidth]{Figures/edge_cost_diagram.png}
\caption{Edge cost illustration: Suez--Red Sea corridor. Numbers on edges show great-circle distances (km). The cost formula $t_e = \text{dist}_e \times m_e \times \Xi_e^{\lambda}$ scales these distances by congestion and security multipliers.}
\label{fig:edge_cost}
\end{figure}

Figure~\ref{fig:security_shifter} shows a side-by-side comparison of the network under baseline conditions and under a risk premium scenario, illustrating how the security cost shifter amplifies edge costs at the affected chokepoint.

\begin{figure}[H]
\centering
\includegraphics[width=\textwidth]{Figures/security_shifter_comparison.png}
\caption{Security cost shifter: baseline (left) versus risk premium scenario (right). The risk premium doubles the effective cost of edges adjacent to Bab el-Mandeb, increasing the cost of transiting the Red Sea corridor.}
\label{fig:security_shifter}
\end{figure}

\subsubsection{Cost and Distance Assumptions}
\label{sec:cost_assumptions}

Our scenario engine reports rerouting costs in \emph{km-equivalent} units, reflecting weighted shortest-path distances through the maritime network. To translate these into approximate economic magnitudes, we document all assumptions used in the cost mapping. Table~\ref{tab:cost_assumptions} lists each parameter, its value, units, and justification.

%This Table was regenerated by ChatGPT because I could not for the life of me get the dang table to work.

\begingroup
\setlength{\LTleft}{0pt}
\setlength{\LTright}{0pt}
\setlength{\tabcolsep}{4pt}        % tighter column padding
\renewcommand{\arraystretch}{1.20} % more row height so text doesn't overlap
\small
\sloppy                           % avoids overfull boxes from citations/long tokens

\begin{longtable}{@{}L{0.30\textwidth}C{0.14\textwidth}C{0.12\textwidth}L{0.44\textwidth}@{}}
\caption{Cost and Distance Assumptions}\label{tab:cost_assumptions}\\
\toprule
Parameter & Value & Units & Justification \& Source \\
\midrule
\endfirsthead

\toprule
Parameter & Value & Units & Justification \& Source \\
\midrule
\endhead

\midrule
\multicolumn{4}{r}{\small\emph{Continued on next page}} \\
\endfoot

\bottomrule
\endlastfoot

\textbf{Haversine distance} & Computed & km &
Great-circle distance between node coordinates; standard geodesic formula \citep{allen2022welfare} \\

\textbf{Congestion multiplier range} & $[1.0, 1.5]$ & dimensionless &
Calibrated from AIS density via log-linear mapping; upper bound consistent with port congestion estimates \citep{notteboom2006impact} \\

\textbf{Overland bypass: Bosporus (Constanta--Piraeus)} & $5.0\times$ haversine & km-equiv. &
Rail/truck intermodal alternative through southeastern Europe; overland freight typically 3--6$\times$ maritime cost \citep{notteboom2006impact} \\

\textbf{Fuel cost (bunker)} & \$400--600 & per tonne &
IFO 380 benchmark price range 2019--2023; IMO 2020 sulphur cap increased costs by 25--40\% for compliant fuels \citep{notteboom2006impact} \\

\textbf{Average fuel consumption} & 150--250 & tonnes/day &
Typical for Panamax--post-Panamax container vessels at design speed \citep{du2020port} \\

\textbf{Shipping cost per tonne-km} & \$5--10 & per 1{,}000 km &
Containerized cargo; range reflects vessel size, fuel prices, and route \citep{notteboom2006impact} \\

\textbf{War-risk insurance premium} & 0.1--0.5\% & of hull value &
Baseline range; can spike to 1--5\% in conflict zones \citep{besley2015welfare, oef2010economic} \\

\textbf{Canal transit tolls: Suez} & \$0.3--1.0M & per transit &
Varies by vessel size and cargo type; represents 2--8\% of total voyage cost for Asia--Europe routes \citep{cariou2021ais} \\

\textbf{Canal transit tolls: Panama} & \$0.2--0.8M & per transit &
New Panamax locks increased capacity but raised tolls \citep{notteboom2006impact} \\

\textbf{Average vessel speed} & 12--16 & knots &
Slow-steaming range for fuel efficiency \citep{du2020port} \\

\textbf{Daily time charter rate} & \$10{,}000--50{,}000 & per day &
Container vessel range depending on size and market conditions \citep{notteboom2006impact} \\

\textbf{Crew cost} & \$3{,}000--8{,}000 & per day &
All-in crew cost for 20--25 person complement \citep{oef2010economic} \\

\textbf{Risk premium scenarios} & 10--200\% & of base cost &
Applied to chokepoint-adjacent edges; upper range consistent with war-risk premium spikes observed during Gulf conflicts \citep{besley2015welfare} \\

\textbf{Partial degradation multiplier} & 1.25--10$\times$ & dimensionless &
Range captures mild congestion ($1.25\times$) through near-closure ($10\times$) \\

\textbf{Global seaborne trade volume} & $\sim$11 billion & tonnes/year &
UNCTAD 2023 estimate; used for order-of-magnitude cost translations \citep{verschuur2023systemic} \\

\end{longtable}
\endgroup



% \begin{longtable}{p{3.8cm}p{2.2cm}p{1.5cm}p{5.5cm}}
% \caption{Cost and Distance Assumptions} \label{tab:cost_assumptions} \\
% \toprule
% Parameter & Value & Units & Justification \& Source \\
% \midrule
% \endfirsthead
% \toprule
% Parameter & Value & Units & Justification \& Source \\
% \midrule
% \endhead
% \midrule
% \multicolumn{4}{r}{\small\emph{Continued on next page}} \\
% \endfoot
% \bottomrule
% \endlastfoot

% \textbf{Haversine distance} & Computed & km & Great-circle distance between node coordinates; standard geodesic formula \citep{allen2022welfare} \\

% \textbf{Congestion multiplier range} & $[1.0, 1.5]$ & dimensionless & Calibrated from AIS density via log-linear mapping; upper bound consistent with port congestion estimates \citep{notteboom2006impact} \\

% \textbf{Overland bypass: Bosporus (Constanta--Piraeus)} & $5.0 \times$ haversine & km-equiv. & Rail/truck intermodal alternative through southeastern Europe; overland freight typically 3--6$\times$ maritime cost \citep{notteboom2006impact} \\

% \textbf{Fuel cost (bunker)} & \$400--600 & per tonne & IFO 380 benchmark price range 2019--2023; IMO 2020 sulphur cap increased costs by 25--40\% for compliant fuels \citep{notteboom2006impact} \\

% \textbf{Average fuel consumption} & 150--250 & tonnes/day & Typical for Panamax--post-Panamax container vessels at design speed \citep{du2020port} \\

% \textbf{Shipping cost per tonne-km} & \$5--10 & per 1,000 km & Containerized cargo; range reflects vessel size, fuel prices, and route \citep{notteboom2006impact} \\

% \textbf{War-risk insurance premium} & 0.1--0.5\% & of hull value & Baseline range; can spike to 1--5\% in conflict zones \citep{besley2015welfare, oef2010economic} \\

% \textbf{Canal transit tolls: Suez} & \$0.3--1.0M & per transit & Varies by vessel size and cargo type; represents 2--8\% of total voyage cost for Asia--Europe routes \citep{cariou2021ais} \\

% \textbf{Canal transit tolls: Panama} & \$0.2--0.8M & per transit & New Panamax locks increased capacity but raised tolls \citep{notteboom2006impact} \\

% \textbf{Average vessel speed} & 12--16 & knots & Slow-steaming range for fuel efficiency \citep{du2020port} \\

% \textbf{Daily time charter rate} & \$10,000--50,000 & per day & Container vessel range depending on size and market conditions \citep{notteboom2006impact} \\

% \textbf{Crew cost} & \$3,000--8,000 & per day & All-in crew cost for 20--25 person complement \citep{oef2010economic} \\

% \textbf{Risk premium scenarios} & 10--200\% & of base cost & Applied to chokepoint-adjacent edges; upper range consistent with war-risk premium spikes observed during Gulf conflicts \citep{besley2015welfare} \\

% \textbf{Partial degradation multiplier} & 1.25--10$\times$ & dimensionless & Range captures mild congestion ($1.25\times$) through near-closure ($10\times$) \\

% \textbf{Global seaborne trade volume} & $\sim$11 billion & tonnes/year & UNCTAD 2023 estimate; used for order-of-magnitude cost translations \citep{verschuur2023systemic} \\
% \end{longtable}

\paragraph{Translating km-equivalent to economic costs.} A mean rerouting cost of $\Delta$ km-equivalent can be approximately converted to annual economic impact as:
\begin{equation}
  \text{Annual cost} \approx \Delta \times \frac{\text{Trade volume (tonnes)}}{\text{Average voyage distance}} \times \text{Cost per tonne-km},
\end{equation}
where we use illustrative values of \$5--10 per tonne per 1,000~km for containerized cargo and approximately 11 billion tonnes of global seaborne trade. These conversions are intended as order-of-magnitude estimates; precise economic costs depend on cargo composition, vessel types, demand elasticities, and market conditions that are beyond the scope of this analysis.

\subsubsection{Congestion Calibration from AIS Density}

We calibrate the congestion proxy $\Xi_e$ using the AIS density data. For each chokepoint $b$, we extract the total AIS position count within the bounding box (Section~3.1.2) and compute:
\begin{equation}
  \Xi_b = \sum_{c \in \text{BBox}(b)} \text{density}_c,
\end{equation}
where the sum is over all raster cells $c$ within the bounding box of chokepoint $b$. We then normalize these values to a $[1, 1.5]$ multiplier range using log-scaling:
\begin{equation}
  m_b^{\text{congestion}} = 1 + 0.5 \cdot \frac{\ln(1 + \Xi_b) - \min_b \ln(1 + \Xi_b)}{\max_b \ln(1 + \Xi_b) - \min_b \ln(1 + \Xi_b)}.
\end{equation}
This produces a mild congestion penalty (up to 50\%) for the most heavily trafficked chokepoints relative to the least trafficked, consistent with empirical estimates of congestion costs in port settings \citep{notteboom2006impact, du2020port}. Figure~\ref{fig:congestion_calibration} illustrates the calibration mapping from AIS intensity to the congestion cost multiplier.

\begin{figure}[H]
\centering
\includegraphics[width=\textwidth]{Figures/congestion_calibration.png}
\caption{Congestion calibration. Left: total AIS intensity (log scale) by chokepoint. Right: mapping from AIS intensity to the congestion multiplier $m_b^{\text{congestion}} \in [1, 1.5]$.}
\label{fig:congestion_calibration}
\end{figure}

\subsubsection{Route Choice and OD Trade Costs}

For each origin port $o$ and destination port $d$, the baseline trade cost is the shortest weighted path through the network:
\begin{equation}
  \tau_{od} = \min_{r \in \mathcal{R}_{od}} \sum_{e \in r} t_e,
  \label{eq:shortest_path}
\end{equation}
where $\mathcal{R}_{od}$ is the set of feasible routes (edge sequences) from $o$ to $d$. This corresponds to the $\theta \to \infty$ limit of the log-sum aggregator used in probabilistic route-choice models \citep{fajgelbaum2020optimal}. We use the Dijkstra shortest-path algorithm implemented in NetworkX to solve (\ref{eq:shortest_path}) for all 1,540 port pairs.

Figure~\ref{fig:route_choice} illustrates the route-choice mechanism by comparing the baseline optimal route from Shanghai to Rotterdam with the rerouted path under Suez Canal closure. Under normal conditions, traffic transits via the Strait of Malacca, Indian Ocean, Bab el-Mandeb, and Suez Canal. When Suez closes, the optimal route shifts to a trans-Pacific alternative via the North Pacific, Panama Canal, and North Atlantic---a dramatically longer but feasible alternative that the scenario engine quantifies.

\begin{figure}[H]
\centering
\includegraphics[width=\textwidth]{Figures/route_choice_before_after.png}
\caption{Route choice: Shanghai to Rotterdam. Left: baseline route via Malacca--Suez corridor. Right: rerouted path via North Pacific--Panama--North Atlantic after Suez Canal closure. Routes pass through ocean waypoints, following realistic maritime arcs.}
\label{fig:route_choice}
\end{figure}

\subsubsection{Scenario Design: Chokepoint Stress Tests}

The central empirical object is the distribution of counterfactual outcomes under disruptions to bottlenecks. We implement three classes of scenarios, using the Suez Canal as a worked example to illustrate each (Figure~\ref{fig:suez_scenarios}):

\paragraph{Scenario 1: Full closure.} For each chokepoint $b$, we remove it from the graph (all edges incident to $b$ are deleted) and recompute shortest paths for all 1,540 port pairs. The delta cost $\Delta \tau_{od}^{(b)} = \tau_{od}^{\text{shocked}} - \tau_{od}^{\text{baseline}}$ measures the rerouting penalty imposed by closure of $b$. For corridor chokepoints like the Suez Canal, closure forces traffic onto dramatically longer alternative routes (e.g., via the Cape of Good Hope or the Panama Canal), producing large rerouting costs. For bypass-dependent chokepoints like the Bosporus, closure forces traffic onto expensive overland alternatives, producing very large per-route costs. This scenario corresponds to a complete physical blockage (e.g., the 2021 Ever Given incident in the Suez Canal).

\paragraph{Scenario 2: Partial capacity degradation.} Rather than full closure, we increase the cost multiplier on chokepoint-adjacent edges by a factor $\alpha > 1$:
\begin{equation}
  t_e^{\text{shocked}} = \alpha \cdot t_e^{\text{baseline}}, \quad e \in \mathcal{E}(b),
\end{equation}
where $\mathcal{E}(b)$ is the set of edges incident to chokepoint $b$. We vary $\alpha \in \{1.25, 1.5, 2.0, 3.0, 5.0, 8.0, 10.0\}$ to trace out a ``severity curve'' for each chokepoint. This scenario captures congestion spikes, speed restrictions, or capacity reductions short of full closure.

\paragraph{Scenario 3: Security risk premium.} We model the effect of reduced security by adding a risk premium to a specific chokepoint's edges:
\begin{equation}
  m_e^{\text{risk}} = m_e^{\text{base}} \cdot (1 + \delta_{\text{risk}}),
\end{equation}
where $\delta_{\text{risk}} \in \{0.10, 0.20, 0.50, 1.00, 2.00\}$ represents an exogenous increase in risk (piracy premium, conflict escalation, insurance surcharge). We apply this premium to \emph{one chokepoint at a time}, producing a pure per-chokepoint vulnerability measure that avoids arbitrary categorization of chokepoints as ``high'' or ``low'' security.

\begin{figure}[H]
\centering
\includegraphics[width=\textwidth]{Figures/suez_scenario_3panel.png}
\caption{Suez Canal: three disruption scenarios. Left: full closure (Mediterranean--Indian Ocean traffic rerouted via Cape of Good Hope or Panama). Center: partial degradation ($\alpha = 3$, costs triple on Suez-adjacent edges). Right: risk premium (+100\%, costs double). The scenario illustrates the three classes of stress tests applied to all six chokepoints.}
\label{fig:suez_scenarios}
\end{figure}

\subsubsection{Output Metrics}

For each scenario, we report:
\begin{enumerate}[nosep]
  \item \textbf{Delta cost}: $\Delta \tau_{od}$ for all 1,540 port pairs, in absolute (km-equivalent) and percentage terms.
  \item \textbf{Mean delta cost}: averaged across port pairs, as a summary vulnerability index for each chokepoint.
  \item \textbf{Fraction affected}: the share of port pairs whose optimal route changes under the shock.
  \item \textbf{Per-port vulnerability}: for each port, the mean cost increase across all its trade routes when a given chokepoint is disrupted.
\end{enumerate}

\subsubsection{Sensitivity Analysis}

We conduct sensitivity analysis along three dimensions: (i) bounding box size (expanding/contracting chokepoint definitions by 20\%); (ii) congestion parameter ($\lambda$ varied from 0 to 0.3); and (iii) cost multiplier range for the congestion normalization. These checks ensure that results are not artifacts of arbitrary parameter choices.


\section{Results}
% Results.tex — Revised: minor polish, consistent cross-references

\subsection{Baseline Shipping Density}

Figure~\ref{fig:global_quicklook} displays the global AIS ship-density raster on a log scale. The distribution of shipping activity is extremely right-skewed: the median cell value is zero, the 90th percentile is approximately 1 AIS position, while the 99th percentile reaches approximately 8.6 million and the maximum cell intensity exceeds 50.6 million positions. This extreme skewness reflects the fundamental geographic concentration of maritime trade: the vast majority of the ocean surface carries negligible traffic, while a small fraction of cells---corresponding to established sea lanes, port approaches, and chokepoint corridors---concentrate virtually all shipping activity.

The densest corridors connect East Asia to Europe via the Suez Canal and Strait of Malacca, East Asia to North America via the Pacific, and Northern Europe to the Americas across the North Atlantic. Within each corridor, traffic funnels sharply at chokepoints, producing localized intensity spikes visible even at global scale.

\begin{figure}[H]
\centering
\includegraphics[width=\textwidth]{Figures/generated/global_quicklook.png}
\caption{Global AIS ship-density raster (log scale). Data from the IMF World Seaborne Trade Monitoring System \citep{cerdeiro2020wstms}, aggregating vessel position reports from January 2015 to February 2021. The raster has been downsampled to $3600 \times 1800$ pixels for visualization.}
\label{fig:global_quicklook}
\end{figure}

\subsection{Chokepoint Traffic Intensity}

Rather than presenting the full intensity table in the main text (see Appendix~\ref{sec:appendix_stats} for complete descriptive statistics), we focus on the Suez Canal as a worked example. This focus is motivated by the Suez Canal's central role in global trade and its historical vulnerability to disruption, most recently demonstrated by the 2021 Ever Given grounding \citep{cariou2021ais}.

\begin{figure}[H]
\centering
\includegraphics[width=0.8\textwidth]{Figures/generated/suez_intensity_detail.png}
\caption{AIS ship density at the Suez Canal (log scale). The bounding box ($[32.2, 32.65] \times [29.8, 31.3]$) captures the full navigable corridor including approach channels.}
\label{fig:suez_detail}
\end{figure}

Figure~\ref{fig:suez_stats} provides a four-panel descriptive analysis of the Suez Canal density data. Across all six chokepoints, the Strait of Malacca dominates in total summed intensity, consistent with its role as the primary corridor for Asia--Europe trade \citep{verschuur2023systemic}. An important distinction emerges between total and concentrated intensity: Gibraltar and the Bosporus have among the highest per-cell intensities, indicating extremely concentrated traffic through narrow passages \citep{notteboom2006impact}.

\begin{figure}[H]
\centering
\includegraphics[width=\textwidth]{Figures/generated/suez_density_stats.png}
\caption{Descriptive statistics of AIS density at the Suez Canal. (a)~Spatial intensity map. (b)~Cell-level intensity histogram. (c)~Comparison of mean and P99 intensity across all chokepoints. (d)~Summary statistics. Full chokepoint-by-chokepoint analysis in Appendix~\ref{sec:appendix_stats}.}
\label{fig:suez_stats}
\end{figure}

\subsection{Full Closure Scenario Results}

Table~\ref{tab:scenario_deltas} reports the results of the full closure scenario, in which each chokepoint is removed from the network and shortest-path costs are recomputed for all 1,540 port pairs. The vulnerability ranking by mean rerouting cost reveals a clear hierarchy:

\begin{enumerate}[nosep]
  \item \textbf{Panama Canal}: mean $\Delta$ cost of 17,840~km-equivalent, 298 pairs affected (19.4\%). Panama closure forces Atlantic--Pacific traffic onto extremely long trans-Pacific or Cape Horn alternatives.
  \item \textbf{Strait of Gibraltar}: mean $\Delta$ cost of 10,624~km, 590 pairs affected (38.3\%). Gibraltar is the gateway between the Atlantic and Mediterranean; its closure forces extensive rerouting for Mediterranean-bound traffic.
  \item \textbf{Suez Canal}: mean $\Delta$ cost of 9,949~km, 469 pairs affected (30.5\%). Suez closure breaks the Mediterranean--Indian Ocean corridor, forcing traffic to reroute via the Cape of Good Hope or the Panama Canal.
  \item \textbf{Bab el-Mandeb}: mean $\Delta$ cost of 4,700~km, 454 pairs affected (29.5\%). Closure forces Indian Ocean traffic to bypass the Suez corridor entirely.
  \item \textbf{Bosporus}: mean $\Delta$ cost of 3,275~km, 159 pairs affected (10.3\%). Closure forces Black Sea ports onto the Constanta--Piraeus overland bypass at 5$\times$ normal cost.
  \item \textbf{Strait of Malacca}: mean $\Delta$ cost of 1,754~km, 290 pairs affected (18.8\%). Closure forces traffic through the Lombok Strait alternative.
\end{enumerate}

% Auto-generated by 04b_enhanced_scenarios.py
\begin{table}[htbp]
\centering
\caption{Full closure scenario results: impact of removing each chokepoint on shortest-path costs across all port-to-port pairs. Mean and max $\Delta$ report the rerouting cost for pairs that find alternative routes via bypass edges.}
\label{tab:scenario_deltas}
\begin{tabular}{lrrrrr}
\toprule
Chokepoint Removed & \multicolumn{1}{c}{Mean $\Delta$} & \multicolumn{1}{c}{Max $\Delta$} & \multicolumn{1}{c}{Affected} & \multicolumn{1}{c}{Disconn.} & \multicolumn{1}{c}{Total} \\
& \multicolumn{1}{c}{(km-equiv.)} & \multicolumn{1}{c}{(km-equiv.)} & \multicolumn{1}{c}{Pairs} & \multicolumn{1}{c}{Pairs} & \multicolumn{1}{c}{Pairs} \\
\midrule
  Panama Canal & 17,840 & 45,342 & 298 & 0 & 1540 \\
  Gibraltar & 10,624 & 26,896 & 590 & 0 & 1540 \\
  Suez Canal & 9,949 & 23,317 & 469 & 0 & 1540 \\
  Bab el Mandeb & 4,700 & 9,907 & 454 & 0 & 1540 \\
  Bosporus & 3,275 & 3,540 & 159 & 0 & 1540 \\
  Strait of Malacca & 1,754 & 3,008 & 290 & 0 & 1540 \\
\bottomrule
\end{tabular}
\end{table}


\begin{figure}[H]
\centering
\includegraphics[width=\textwidth]{Figures/generated/scenario_summary.png}
\caption{Full closure scenario results. Left: chokepoints ranked by mean rerouting cost (km-equivalent) across 1,540 port pairs. Right: fraction of port pairs affected by each closure.}
\label{fig:closure_results}
\end{figure}

Figure~\ref{fig:scenario_example_suez} zooms into the Suez Canal closure, showing the 20 most affected port pairs.

\begin{figure}[H]
\centering
\includegraphics[width=\textwidth]{Figures/generated/scenario_example.png}
\caption{Suez Canal closure: 20 most affected port pairs. Mediterranean--Indian Ocean traffic is forced onto dramatically longer alternative routes via the Cape of Good Hope or Panama Canal.}
\label{fig:scenario_example_suez}
\end{figure}

Figure~\ref{fig:closure_map} maps the per-port vulnerability under the three highest-impact closures (Panama, Gibraltar, Suez).

\begin{figure}[H]
\centering
\includegraphics[width=\textwidth]{Figures/generated/closure_impact_map.png}
\caption{Port vulnerability under the three highest-impact closures. Port color intensity reflects mean percentage cost increase. X marks the closed chokepoint.}
\label{fig:closure_map}
\end{figure}

\subsection{Partial Degradation Scenarios}

Figure~\ref{fig:partial_degradation} reports the partial degradation results, in which edge costs at each chokepoint are multiplied by $\alpha \in \{1.25, 1.5, 2.0, 3.0, 5.0, 8.0, 10.0\}$ rather than the chokepoint being fully removed.

The surface plot reveals the severity curve for each chokepoint. At $\alpha = 5$, Gibraltar produces the largest mean cost increase, followed by the Suez Canal and Bab el-Mandeb. Even at $\alpha = 1.5$, the most critical chokepoints impose mean cost increases of 2--5\% across all port pairs. A notable pattern is the contrast between corridor chokepoints (Gibraltar, Suez), where severity curves accelerate monotonically, and chokepoints with viable alternatives (Panama, Malacca), where severity curves are more moderate because alternative routes absorb rerouted traffic at relatively lower cost.

\begin{figure}[H]
\centering
\includegraphics[width=\textwidth]{Figures/generated/partial_degradation_surface_3D_better_angle.png}
\caption{Partial degradation results: mean percentage cost increase across 1,540 port pairs for each chokepoint and degradation multiplier $\alpha$. The surface reveals nonlinear severity curves that accelerate at high $\alpha$ for corridor chokepoints.}
\label{fig:partial_degradation}
\end{figure}

\begin{figure}[H]
	\centering
	\includegraphics[width=\textwidth]{Figures/generated/partial_degradation_surface_lines.png}
	\caption{Partial degradation results a 2D: we split per corridor chokepoints to show which contributes the most when for the costs.}
	\label{fig:partial_degradation2}
\end{figure}


\subsection{Port Vulnerability Analysis}

Figure~\ref{fig:port_vulnerability} presents the full port-by-chokepoint vulnerability matrix: for each of the 56 ports and each of the 6 chokepoints, the cell value shows the mean percentage cost increase when the chokepoint is closed.

\begin{figure}[H]
\centering
\includegraphics[width= 0.8\textwidth]{Figures/generated/port_vulnerability_heatmap.png}
\caption{Port vulnerability heatmap: mean percentage cost increase for each port (rows) when each chokepoint (columns) is fully closed. Ports are sorted by maximum vulnerability; chokepoints by total impact.}
\label{fig:port_vulnerability}
\end{figure}

The vulnerability matrix reveals three distinct profiles:
\begin{itemize}[nosep]
  \item \textbf{Bypass-dependent ports} (behind Bosporus) face the largest per-route cost increases, concentrated on a single chokepoint.
  \item \textbf{Corridor-dependent ports} (e.g., Mediterranean ports for Suez/Gibraltar) face large but more moderate rerouting costs that vary smoothly across chokepoints.
  \item \textbf{Highly connected ports} (e.g., Rotterdam, New York, Los Angeles) with multiple routing options are resilient to any single closure.
\end{itemize}

\subsection{Security Risk Scenario Analysis}

We apply risk premiums of 10\%, 20\%, 50\%, 100\%, and 200\% to each chokepoint individually and measure the resulting mean cost increase across all 1,540 port pairs.

\begin{figure}[H]
\centering
\includegraphics[width=\textwidth]{Figures/generated/security_scenario_comparison.png}
\caption{Security risk scenario: mean percentage cost increase from applying a risk premium to each chokepoint individually. Bars show per-chokepoint impact at five premium levels.}
\label{fig:security_scenarios}
\end{figure}

Gibraltar and the Suez Canal emerge as the most sensitive to risk premiums: a 100\% risk premium on either alone increases mean trade costs by approximately 5\%. The ordering differs from the full-closure ranking because security-risk scenarios capture the \emph{incremental} effect of cost increases rather than the \emph{complete} effect of removal. Chokepoints with high baseline traffic and limited alternative routes are most sensitive to risk premiums.

The security scenario analysis enables a clean estimation of the hegemonic dividend as the \emph{cost that would be avoided if a chokepoint's risk were reduced to baseline}. Aggregating across all six chokepoints under a 50\% risk premium, the total dividend is substantial, consistent with the public-good characterization of maritime security \citep{besley2015welfare, bueger2024securing, oef2010economic}.


\section{Discussion and Conclusion}
% Discussion.tex — Revised: separated from Conclusion; spatial policy citations integrated

\subsection{The Cost of Chokepoint Disruption}

Table~\ref{tab:cost_per_day} presents the estimated daily rerouting cost of each chokepoint closure, combining our scenario engine outputs with published annual transit counts and standard maritime shipping cost estimates (see Appendix~\ref{sec:appendix_shipcounts} for the underlying ship-count estimation methodology).

% Auto-generated by 04b_enhanced_scenarios.py
%%\begin{table}[htbp]
%\centering
%\caption{Estimated daily rerouting cost of chokepoint closure. Low/Mid/High estimates use \$50/\$75/\$100 per additional km respectively, reflecting fuel, charter time, and crew costs. Daily vessel counts from UNCTAD, Suez Canal Authority, and Panama Canal Authority statistics.}
%\label{tab:cost_per_day}
%\begin{tabular}{lrrrrrr}
%\toprule
%Chokepoint & Daily Vessels & Mean Detour (km) & Cost/Day (Low) & Cost/Day (Mid) & Cost/Day (High) \\
%\midrule
%  Gibraltar & 219 & 10,624 & \$116.4M & \$174.6M & \$232.9M \\
%  Panama Canal & 37 & 17,840 & \$33.0M & \$49.5M & \$66.0M \\
%  Suez Canal & 53 & 9,949 & \$26.6M & \$39.9M & \$53.2M \\
%  Strait of Malacca & 233 & 1,754 & \$20.4M & \$30.6M & \$40.9M \\
%  Bosporus & 118 & 3,275 & \$19.3M & \$28.9M & \$38.6M \\
%  Bab el Mandeb & 68 & 4,700 & \$16.1M & \$24.1M & \$32.2M \\
%\bottomrule
%\end{tabular}
%\end{table}

\begin{table}[H]
	\centering
	\caption{Estimated daily rerouting cost of chokepoint closure. Low/Mid/High estimates use \$50/\$75/\$100 per additional km respectively, reflecting fuel, charter time, and crew costs. Daily vessel counts from UNCTAD, Suez Canal Authority, and Panama Canal Authority statistics.}
	\label{tab:cost_per_day}
	\resizebox{\textwidth}{!}{%
		\begin{tabular}{lrrrrrr}
			\toprule
			Chokepoint & Daily Vessels & Mean Detour (km) & Cost/Day (Low) & Cost/Day (Mid) & Cost/Day (High) \\
			\midrule
			Gibraltar & 219 & 10,624 & \$116.4M & \$174.6M & \$232.9M \\
			Panama Canal & 37 & 17,840 & \$33.0M & \$49.5M & \$66.0M \\
			Suez Canal & 53 & 9,949 & \$26.6M & \$39.9M & \$53.2M \\
			Strait of Malacca & 233 & 1,754 & \$20.4M & \$30.6M & \$40.9M \\
			Bosporus & 118 & 3,275 & \$19.3M & \$28.9M & \$38.6M \\
			Bab el Mandeb & 68 & 4,700 & \$16.1M & \$24.1M & \$32.2M \\
			\bottomrule
		\end{tabular}%
	}
\end{table}

The cost-per-day estimates reveal a clear hierarchy of economic exposure. Gibraltar closure produces the largest daily rerouting cost (\$117--233 million per day), driven by its high daily vessel throughput (219 vessels/day) combined with a substantial mean detour of 10,624~km. The Panama Canal follows (\$33--66M/day): although it handles far fewer vessels (37/day), the extreme detour distance (17,840~km) amplifies the per-vessel rerouting cost. The Suez Canal (\$27--53M/day) and the Strait of Malacca (\$20--41M/day) occupy the middle tier, while the Bosporus (\$19--39M/day) and Bab el-Mandeb (\$16--32M/day) produce the lowest---though still substantial---daily costs.

Aggregating across all six chokepoints, a simultaneous closure would produce daily rerouting costs on the order of \$230--460 million, or approximately \$85--170 billion annually---magnitudes consistent with the order-of-magnitude estimates in \citet{oef2010economic} and \citet{mbekeani2011piracy} for the broader economic costs of maritime insecurity.

\subsection{The Hegemonic Dividend}

The per-chokepoint security risk analysis (Section~\ref{sec:results}) provides a direct estimate of the hegemonic dividend---the trade-cost increase that would materialize if security deteriorated and risk premiums rose. Under a 100\% risk premium, plausible in a scenario of piracy escalation or military confrontation \citep{besley2015welfare}, the per-chokepoint dividend ranges from moderate (Strait of Malacca, approximately 1\%) to substantial (Gibraltar and Suez Canal, approximately 5\% each). Aggregating across all six chokepoints, the total dividend implies trade-cost savings equivalent to thousands of km-equivalent per port pair, translating to tens of billions of dollars annually at current trade volumes.

Crucially, the marginal value of security provision is \emph{increasing in the risk environment}: the dividend is largest precisely when security conditions are deteriorating. As risk levels rise due to piracy, conflict spillovers, or geopolitical tensions, the economic case for sustained security investment strengthens rather than weakens \citep{oef2010economic, mbekeani2011piracy}. This formalizes the intuition from the international relations and defense economics literatures \citep{bueger2019maritime, kraska2015law}.

\subsection{Structural Patterns}

\paragraph{Network topology and vulnerability.} The 78-node network reveals a structural distinction between two types of chokepoints. \emph{Through-corridor chokepoints} (Malacca, Gibraltar, Suez, Bab el-Mandeb, Panama) connect large regions; their closure imposes rerouting costs on many port pairs but does not sever connectivity, because alternative maritime routes absorb rerouted traffic. \emph{Bypass-dependent chokepoints} (Bosporus) are the primary connectors for ports behind them; their closure forces traffic onto expensive overland bypass routes. This distinction maps onto the theoretical framework of \citet{fajgelbaum2020optimal}, who show that optimal transport networks concentrate investment on high-throughput corridors.

Security priorities should therefore differ by chokepoint type. For through-corridor chokepoints, the objective is to minimize rerouting costs by maintaining capacity and reducing risk premiums. For bypass-dependent chokepoints, the objective is to prevent disruption that forces traffic onto dramatically more expensive alternatives. This connects to the congestion externality analysis of \citet{hierons2024spreading}: congestion pricing or capacity investment at through-corridor chokepoints can reduce rerouting costs, whereas bypass-dependent chokepoints require both security provision and investment in alternative infrastructure.

\paragraph{Nonlinear severity.} The partial degradation analysis reveals that the relationship between disruption severity and cost increase is nonlinear. For through-corridor chokepoints, severity curves accelerate at high degradation levels ($\alpha \geq 5$) as rerouting becomes the dominant response. Gibraltar and the Suez Canal exhibit the steepest severity curves, producing mean cost increases exceeding 26\% and 12\% respectively at $\alpha = 5$. This nonlinearity implies that the marginal return to reducing risk is highest at moderate disruption levels, suggesting that maintaining a credible deterrent captures most of the available dividend.

\paragraph{Heterogeneous vulnerability.} The port vulnerability analysis reveals that the dividend is not uniformly distributed. Bypass-dependent ports (Black Sea) benefit most from their chokepoint's security. Corridor-dependent ports (Mediterranean, South Asian, East African) benefit from the security of multiple chokepoints along their primary trade routes. Highly connected hub ports (Rotterdam, New York, Singapore) are relatively insulated. This heterogeneity implies that the political economy of maritime security investment is complex: the ports that benefit most from the public good are often in developing regions with limited capacity to contribute to its provision \citep{gaubert2025place, fajgelbaum2019state}. The geographic sorting dynamics studied by \citet{desmet2015geography} and the agglomeration mechanisms of \citet{duranton2004micro} suggest that security investments may also affect the long-run spatial distribution of economic activity, as firms and workers adjust location decisions in response to changed trade-cost structures.

\subsection{Limitations}

We are explicit about several limitations that constrain interpretation while noting that each suggests a concrete direction for future work.

\paragraph{Static aggregate density.} The AIS data aggregate vessel positions from January 2015 to February 2021 into a single snapshot. We cannot track temporal changes, seasonal variations, or dynamic adjustment to disruptions. Temporal disaggregation would enable analysis of how trade routes evolve in response to shocks \citep{cariou2021ais, du2020port}.

\paragraph{Aggregate density versus vessel-level flows.} The density raster records total AIS positions without distinguishing vessel type, flag, cargo, or voyage. Extending to vessel-level AIS data would enable construction of origin-destination trade matrices and support estimation of the full spatial equilibrium model \citep{fajgelbaum2020optimal, allen2022welfare, kerbl2022ais}.

\paragraph{No welfare computation.} Without bilateral trade volumes and demand elasticities, we cannot compute welfare changes in the general-equilibrium sense of \citet{redding2017quantitative} or \citet{allen2025quantitative}. Our delta-cost measures are informative about direction and relative magnitude but are not welfare estimates.

\paragraph{Exogenous security parameters.} The security risk analysis treats risk premiums as exogenous inputs. In reality, security provision is endogenous to the strategic environment \citep{besley2015welfare, kraska2015law}. Endogenizing security would require a game-theoretic extension in which a security provider allocates resources across chokepoints to minimize expected trade-cost losses \citep{fajgelbaum2020spatial}.

\subsection{Directions for Future Research}

\begin{enumerate}[nosep]
  \item \textbf{Vessel-level AIS trajectories.} Individual vessel tracks would enable construction of origin-destination trade flow matrices and support estimation of route-choice models with probabilistic assignment \citep{fajgelbaum2020optimal, fajgelbaum2020supplement}.

  \item \textbf{Bilateral maritime trade data.} Combining vessel-level AIS with customs or port-authority trade data would enable calibration of a full spatial equilibrium model with endogenous demand \citep{redding2017quantitative, allen2025quantitative, redding2025urban}.

  \item \textbf{Insurance and risk pricing data.} War-risk insurance premiums and piracy incidence records would provide empirically grounded values for the security parameters \citep{besley2015welfare, oef2010economic}.

  \item \textbf{Naval presence proxies.} Open-source data on naval deployments would enable causal estimation of the security provision effect.

  \item \textbf{Dynamic extensions.} A time-varying network with stochastic disruptions and adjustment costs would capture the temporal dimension of chokepoint risk \citep{hierons2024spreading, bordeu2025commuting}.

  \item \textbf{Optimal security allocation.} A normative extension could solve for the optimal spatial allocation of security resources, building on \citet{rossihansberg2004optimal}, \citet{fajgelbaum2020spatial, fajgelbaum2025optimal}, and \citet{gaubert2025place}: given a fixed security budget, which chokepoints should receive the most protection?

  \item \textbf{Political economy of burden-sharing.} The observation that maritime security is a global public good with concentrated provision raises questions about sustainability and equity. An extension incorporating political-economy considerations \citep{fajgelbaum2019state, bordeu2025commuting} could analyze the efficiency and distributional consequences of alternative burden-sharing regimes. The frameworks of \citet{owens2020rethinking} and \citet{rossihansberg2023cognitive} offer tools for evaluating how spatial redistribution interacts with agglomeration and connectivity.
\end{enumerate}


%\input{Bibliography/references}
\newpage
\bibliography{Bibliography/references}

% ══════════════════════════════════════════════════════════
% APPENDICES
% ══════════════════════════════════════════════════════════
\appendix
\newpage
\section{Appendix}
\section{Chokepoint Bounding Box Maps}
\label{sec:appendix_bboxes}

Figure~\ref{fig:appendix_all_bboxes} shows all six chokepoint bounding boxes on a single world map. Figures~\ref{fig:appendix_bbox_suez}--\ref{fig:appendix_bbox_gibraltar} show individual zoomed maps for each chokepoint, with the AIS density visible in the background and the bounding box coordinates annotated.

\begin{figure}[H]
\centering
\includegraphics[width=\textwidth]{Figures/appendix_all_bboxes_world.png}
\caption{All six chokepoint bounding boxes overlaid on a world map.}
\label{fig:appendix_all_bboxes}
\end{figure}

\begin{figure}[H]
\centering
\includegraphics[width=0.75\textwidth]{Figures/appendix_bbox_Suez_Canal.png}
\caption{Bounding box: Suez Canal.}
\label{fig:appendix_bbox_suez}
\end{figure}

\begin{figure}[H]
\centering
\includegraphics[width=0.75\textwidth]{Figures/appendix_bbox_Bab_el_Mandeb.png}
\caption{Bounding box: Bab el-Mandeb.}
\end{figure}

\begin{figure}[H]
\centering
\includegraphics[width=0.75\textwidth]{Figures/appendix_bbox_Strait_of_Malacca.png}
\caption{Bounding box: Strait of Malacca.}
\end{figure}

\begin{figure}[H]
\centering
\includegraphics[width=0.75\textwidth]{Figures/appendix_bbox_Panama_Canal.png}
\caption{Bounding box: Panama Canal.}
\end{figure}

\begin{figure}[H]
\centering
\includegraphics[width=0.75\textwidth]{Figures/appendix_bbox_Bosporus.png}
\caption{Bounding box: Bosporus.}
\end{figure}

\begin{figure}[H]
\centering
\includegraphics[width=0.75\textwidth]{Figures/appendix_bbox_Gibraltar.png}
\caption{Bounding box: Strait of Gibraltar.}
\label{fig:appendix_bbox_gibraltar}
\end{figure}

\section{Chokepoint Traffic Density Close-ups}
\label{sec:appendix_density}

Figures~\ref{fig:appendix_dens_suez}--\ref{fig:appendix_dens_gibraltar} show zoomed AIS density overlays for each chokepoint bounding box, with intensity statistics annotated.

\begin{figure}[H]
\centering
\includegraphics[width=0.75\textwidth]{Figures/appendix_density_Suez_Canal.png}
\caption{AIS density: Suez Canal.}
\label{fig:appendix_dens_suez}
\end{figure}

\begin{figure}[H]
\centering
\includegraphics[width=0.75\textwidth]{Figures/appendix_density_Bab_el_Mandeb.png}
\caption{AIS density: Bab el-Mandeb.}
\end{figure}

\begin{figure}[H]
\centering
\includegraphics[width=0.75\textwidth]{Figures/appendix_density_Strait_of_Malacca.png}
\caption{AIS density: Strait of Malacca.}
\end{figure}

\begin{figure}[H]
\centering
\includegraphics[width=0.75\textwidth]{Figures/appendix_density_Panama_Canal.png}
\caption{AIS density: Panama Canal.}
\end{figure}

\begin{figure}[H]
\centering
\includegraphics[width=0.75\textwidth]{Figures/appendix_density_Bosporus.png}
\caption{AIS density: Bosporus.}
\end{figure}

\begin{figure}[H]
\centering
\includegraphics[width=0.75\textwidth]{Figures/appendix_density_Gibraltar.png}
\caption{AIS density: Strait of Gibraltar.}
\label{fig:appendix_dens_gibraltar}
\end{figure}

\section{Chokepoint Descriptive Statistics}
\label{sec:appendix_stats}

The full chokepoint intensity table is reported below, followed by four-panel descriptive statistics figures for each chokepoint.

% Auto-generated by 04b_enhanced_scenarios.py
\begin{table}[htbp]
\centering
\caption{Chokepoint traffic intensity from AIS raster data. Sum intensity reflects total vessel traffic density within each chokepoint bounding box.}
\label{tab:chokepoint_intensity}
\begin{tabular}{lrrrrr}
\toprule
Chokepoint & Sum Intensity & Mean Intensity & P99 Cell & Grid Cells & Rank \\
\midrule
  Strait of Malacca & 1.21T & 945.8K & 12.61M & 1,276,000 & 1 \\
  Gibraltar & 249.01B & 12.45M & 40.31M & 20,000 & 2 \\
  Bab el Mandeb & 46.04B & 804.9K & 15.90M & 57,200 & 3 \\
  Panama Canal & 40.36B & 1.40M & 15.39M & 28,800 & 4 \\
  Suez Canal & 35.79B & 1.33M & 32.09M & 27,000 & 5 \\
  Bosporus & 11.36B & 3.55M & 48.42M & 3,200 & 6 \\
\bottomrule
\end{tabular}
\end{table}


\begin{figure}[H]
\centering
\includegraphics[width=\textwidth]{Figures/appendix_stats_Suez_Canal.png}
\caption{Descriptive statistics: Suez Canal.}
\end{figure}

\begin{figure}[H]
\centering
\includegraphics[width=\textwidth]{Figures/appendix_stats_Bab_el_Mandeb.png}
\caption{Descriptive statistics: Bab el-Mandeb.}
\end{figure}

\begin{figure}[H]
\centering
\includegraphics[width=\textwidth]{Figures/appendix_stats_Strait_of_Malacca.png}
\caption{Descriptive statistics: Strait of Malacca.}
\end{figure}

\begin{figure}[H]
\centering
\includegraphics[width=\textwidth]{Figures/appendix_stats_Panama_Canal.png}
\caption{Descriptive statistics: Panama Canal.}
\end{figure}

\begin{figure}[H]
\centering
\includegraphics[width=\textwidth]{Figures/appendix_stats_Bosporus.png}
\caption{Descriptive statistics: Bosporus.}
\end{figure}

\begin{figure}[H]
\centering
\includegraphics[width=\textwidth]{Figures/appendix_stats_Gibraltar.png}
\caption{Descriptive statistics: Strait of Gibraltar.}
\end{figure}

\section{Ship-Count Estimation Under Closure Scenarios}
\label{sec:appendix_shipcounts}

\input{Paper_sections/Appendix6}

\end{document}

