\subsection{Motivation}

Global commerce is overwhelmingly maritime. Approximately 80 percent of world trade by volume and over 70 percent by value traverses the oceans, channeled through a small number of physical bottlenecks---straits, canals, and narrow sea lanes---that concentrate traffic and vulnerability \citep{verschuur2023systemic, bueger2024securing}. The reliability of these corridors is not a natural given; it depends on the provision of security, navigational infrastructure, and rules-based governance that together constitute what international relations scholars term the ``global maritime commons'' \citep{bueger2019maritime, kraska2015law}. When this security breaks down---through piracy, conflict-related closures, or geopolitical risk---the consequences propagate rapidly through rerouting, congestion, and elevated trade costs \citep{besley2015welfare, notteboom2006impact}.

The U.S. Navy has historically been the predominant provider of security in the global commons, maintaining forward presence through overseas basing, freedom of navigation operations (FONOPs), and patrol activities that deter threats and ensure the openness of critical sea lanes. This role can be framed as the provision of a \emph{global public good}: security that is non-rival (one country's benefit from safe sea lanes does not diminish another's) and largely non-excludable (all trading nations benefit from open chokepoints regardless of contribution) \citep{bueger2024securing, mbekeani2011piracy}. The economic value of this public good---what we term the ``hegemonic dividend''---remains poorly quantified, in part because the most policy-relevant outcomes (trade reliability, routing resilience, avoided cost spikes) are \emph{equilibrium objects} of a transportation network that require explicit modeling of route substitution.

\subsection{Research Question and Approach}

This paper asks: \emph{How sensitive is the global maritime trading system to disruptions at critical chokepoints, and what is the economic value of security provision that prevents or mitigates such disruptions?}

Figure~\ref{fig:intro_quicklook} provides the motivating empirical context: global AIS ship-density data reveal that maritime traffic is extraordinarily concentrated along a small number of corridors, funneling through physical bottlenecks where disruption risk is highest.

\begin{figure}[H]
\centering
\includegraphics[width=\textwidth]{Figures/global_quicklook.png}
\caption{Global AIS ship-density raster (log scale), aggregating vessel position reports from January 2015 to February 2021. Traffic concentrates along established corridors and funnels through a small number of physical chokepoints. Data from the IMF World Seaborne Trade Monitoring System \citep{cerdeiro2020wstms}.}
\label{fig:intro_quicklook}
\end{figure}

We address this question through a scenario-based quantitative spatial framework rather than through direct causal estimation of naval operations. This choice is motivated by two considerations. First, granular data on naval deployments, patrol schedules, and operational details are largely classified and unavailable for academic research. Second, even with such data, the most informative exercise for policymakers is counterfactual: \emph{what would happen if a chokepoint were disrupted, and how much would enhanced security reduce the losses?} This framing aligns with the ``transportation networks'' approach in quantitative spatial economics \citep{fajgelbaum2020optimal, allen2022welfare}, which represents trade flows on a graph with endogenous route choice and congestion, and evaluates counterfactuals by shocking network elements and recomputing equilibrium.

Our empirical strategy proceeds in three stages. First, we use high-resolution AIS (Automatic Identification System) ship-density rasters from the IMF World Seaborne Trade Monitoring System \citep{cerdeiro2020wstms} to construct a baseline portrait of global shipping intensity and to measure the concentration of maritime activity at six major chokepoints. Second, we build a stylized maritime transport network---a weighted graph of ocean basins and chokepoints---in which edge costs depend on distance and a congestion proxy calibrated from the AIS density data. Third, we implement a ``scenario engine'' that imposes exogenous shocks to chokepoint edges (closure, capacity reduction, risk spike) and computes the resulting changes in least-cost path lengths, rerouting patterns, and aggregate trade-cost indices across basin pairs. Security enters the model as a cost-reducing and risk-reducing shifter on bottleneck-adjacent edges: higher security provision lowers the effective cost of transit and improves reliability, which becomes especially valuable under disruption scenarios.

\subsection{Contributions}

This paper makes three contributions. First, it provides a replicable, data-grounded measurement framework for evaluating chokepoint vulnerability using publicly available AIS data. The global density map and chokepoint intensity rankings we construct offer a transparent baseline for any subsequent analysis of maritime disruptions. Second, the scenario engine translates the qualitative intuition that ``chokepoints matter'' into quantitative cost-change estimates that can be compared across bottlenecks and shock magnitudes. This ranking of vulnerability---which bottleneck closure generates the largest rerouting cost---is directly useful for security resource allocation. Third, by situating the analysis within the quantitative spatial economics toolkit \citep{redding2017quantitative, donaldson2025transport}, we connect the maritime security literature to a rigorous modeling tradition that emphasizes general-equilibrium effects, welfare measurement, and network structure.

We are explicit about the limitations of the current exercise. With only an aggregate density raster (not vessel-level trajectories or bilateral trade flows), we cannot estimate a full spatial equilibrium with origin-destination demand or compute welfare in the general-equilibrium sense of \citet{fajgelbaum2020optimal}. The network model is therefore stylized: it uses great-circle distances for edge weights, a congestion proxy derived from raster intensity, and shortest-path routing. We label all scenario results as ``partial-exercise'' estimates and interpret them as informative bounds on the sensitivity of the maritime system to disruption. A key output of the paper is a concrete specification of what additional data---vessel-level AIS trajectories, bilateral maritime trade matrices, insurance risk indicators, and naval presence measures---would be needed to move from scenario simulation to causal inference.

\subsection{Roadmap}

Section~2 reviews the literature on maritime security as a public good, trade frictions and risk, quantitative spatial models of transportation, and the use of AIS data in economic research. Section~3 describes the data (AIS density raster, chokepoint definitions) and methods (network construction, cost specification, scenario design). Section~4 presents results: baseline density patterns, chokepoint intensity rankings, and scenario outputs under multiple shock types. Section~5 discusses implications, limitations, and directions for future work.
