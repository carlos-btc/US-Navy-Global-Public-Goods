
\subsection{The Cost of Chokepoint Disruption and the Hegemonic Dividend}

Table~\ref{tab:cost_per_day} presents the estimated daily rerouting cost of each chokepoint closure, combining our scenario engine outputs with published annual transit counts and standard maritime shipping cost estimates.

% Auto-generated by 04b_enhanced_scenarios.py
%%\begin{table}[htbp]
%\centering
%\caption{Estimated daily rerouting cost of chokepoint closure. Low/Mid/High estimates use \$50/\$75/\$100 per additional km respectively, reflecting fuel, charter time, and crew costs. Daily vessel counts from UNCTAD, Suez Canal Authority, and Panama Canal Authority statistics.}
%\label{tab:cost_per_day}
%\begin{tabular}{lrrrrrr}
%\toprule
%Chokepoint & Daily Vessels & Mean Detour (km) & Cost/Day (Low) & Cost/Day (Mid) & Cost/Day (High) \\
%\midrule
%  Gibraltar & 219 & 10,624 & \$116.4M & \$174.6M & \$232.9M \\
%  Panama Canal & 37 & 17,840 & \$33.0M & \$49.5M & \$66.0M \\
%  Suez Canal & 53 & 9,949 & \$26.6M & \$39.9M & \$53.2M \\
%  Strait of Malacca & 233 & 1,754 & \$20.4M & \$30.6M & \$40.9M \\
%  Bosporus & 118 & 3,275 & \$19.3M & \$28.9M & \$38.6M \\
%  Bab el Mandeb & 68 & 4,700 & \$16.1M & \$24.1M & \$32.2M \\
%\bottomrule
%\end{tabular}
%\end{table}

\begin{table}[H]
	\centering
	\caption{Estimated daily rerouting cost of chokepoint closure. Low/Mid/High estimates use \$50/\$75/\$100 per additional km respectively, reflecting fuel, charter time, and crew costs. Daily vessel counts from UNCTAD, Suez Canal Authority, and Panama Canal Authority statistics.}
	\label{tab:cost_per_day}
	\resizebox{\textwidth}{!}{%
		\begin{tabular}{lrrrrrr}
			\toprule
			Chokepoint & Daily Vessels & Mean Detour (km) & Cost/Day (Low) & Cost/Day (Mid) & Cost/Day (High) \\
			\midrule
			Gibraltar & 219 & 10,624 & \$116.4M & \$174.6M & \$232.9M \\
			Panama Canal & 37 & 17,840 & \$33.0M & \$49.5M & \$66.0M \\
			Suez Canal & 53 & 9,949 & \$26.6M & \$39.9M & \$53.2M \\
			Strait of Malacca & 233 & 1,754 & \$20.4M & \$30.6M & \$40.9M \\
			Bosporus & 118 & 3,275 & \$19.3M & \$28.9M & \$38.6M \\
			Bab el Mandeb & 68 & 4,700 & \$16.1M & \$24.1M & \$32.2M \\
			\bottomrule
		\end{tabular}%
	}
\end{table}

The cost-per-day estimates reveal a clear hierarchy of economic exposure. Gibraltar closure produces the largest daily rerouting cost (\$117--233 million per day), driven by its exceptionally high daily vessel throughput (219 vessels/day) combined with a substantial mean detour of 10,624~km. The Panama Canal follows (\$33--66M/day): although it handles far fewer vessels (37/day), the extreme detour distance (17,840~km) amplifies the per-vessel rerouting cost. The Suez Canal (\$27--53M/day) and the Strait of Malacca (\$20--41M/day) occupy the middle tier, while the Bosporus (\$19--39M/day) and Bab el-Mandeb (\$16--32M/day) produce the lowest---though still substantial---daily costs.

Aggregating across all six chokepoints, a simultaneous closure would produce daily rerouting costs on the order of \$230--460 million, or approximately \$85--170 billion annually. Even a single closure sustained for one month would generate rerouting costs of \$0.5--7 billion, depending on the chokepoint---magnitudes consistent with the order-of-magnitude estimates in \citet{oef2010economic} and \citet{mbekeani2011piracy} for the broader economic costs of maritime insecurity.

\paragraph{The hegemonic dividend as avoided cost.} The per-chokepoint security risk analysis (Section~4.6) provides a direct estimate of the hegemonic dividend. For each chokepoint, the dividend is the trade-cost increase that would materialize if security deteriorated and risk premiums rose. Under a 100\% risk premium---plausible in a scenario of piracy escalation or military confrontation \citep{besley2015welfare}---the per-chokepoint dividend ranges from moderate (Strait of Malacca, approximately 1\%) to substantial (Gibraltar and Suez Canal, approximately 5\% each). Aggregating across all six chokepoints, the total dividend under a 100\% risk premium implies trade-cost savings equivalent to thousands of km-equivalent per port pair, translating to tens of billions of dollars annually at current trade volumes.

This finding formalizes the intuition from the international relations and defense economics literatures \citep{bueger2019maritime, kraska2015law}: the U.S. Navy's forward presence at major chokepoints generates substantial positive externalities for the global trading system. Crucially, the marginal value of security provision is \emph{increasing in the risk environment}: the dividend is largest precisely when security conditions are deteriorating. This has implications for burden-sharing debates: as risk levels rise (due to piracy, conflict spillovers, or geopolitical tensions), the economic case for sustained security investment strengthens rather than weakens \citep{oef2010economic, mbekeani2011piracy}.

\paragraph{Heterogeneity across ports.} The port vulnerability analysis (Section~4.5) reveals that the dividend is not uniformly distributed. Bypass-dependent ports (Black Sea) benefit most from their chokepoint's security, as they face the largest per-route rerouting costs when forced onto the expensive Constanta--Piraeus overland bypass at 5$\times$ normal cost. Corridor-dependent ports (Mediterranean, South Asian, East African) benefit from the security of multiple chokepoints along their primary trade routes. Highly connected hub ports (Rotterdam, New York, Singapore) are relatively insulated, benefiting primarily from the general reduction in risk premiums. This heterogeneity implies that the political economy of maritime security investment is complex: the ports that benefit most from the public good are often in developing regions with limited capacity to contribute to its provision \citep{gaubert2025place, fajgelbaum2019state}.

\subsection{Structural Patterns and Downstream Implications}

\paragraph{Network topology and vulnerability.} The expanded 78-node network reveals a structural distinction between two types of chokepoints that is obscured in simpler models. \emph{Through-corridor chokepoints} (Malacca, Gibraltar, Suez, Bab el-Mandeb, Panama) connect large regions of the network; their closure imposes rerouting costs on many port pairs but does not sever connectivity, because alternative maritime routes (e.g., the Lombok Strait as an alternative to Malacca, or the Cape of Good Hope as an alternative to Suez) absorb rerouted traffic. \emph{Bypass-dependent chokepoints} (Bosporus) are the primary connectors for ports behind them; their closure forces traffic onto expensive overland bypass routes, producing very large per-route cost increases even though the network remains technically connected. This distinction maps onto the theoretical framework of \citet{fajgelbaum2020optimal}, who show that optimal transport networks concentrate investment on high-throughput corridors: through-corridor chokepoints carry the highest traffic precisely because they offer the shortest routes between major economic regions.

The practical implication is that security priorities should differ by chokepoint type. For through-corridor chokepoints, the objective is to minimize rerouting costs by maintaining capacity and reducing risk premiums. For bypass-dependent chokepoints, the objective is to prevent disruption that forces traffic onto dramatically more expensive alternatives, because even with bypass routes available, the cost multiplier (5$\times$ normal) produces severe economic impacts for dependent ports. This distinction connects to the congestion externality analysis of \citet{hierons2024spreading}: congestion pricing or capacity investment at through-corridor chokepoints can reduce rerouting costs, whereas bypass-dependent chokepoints require both security provision and continued investment in alternative infrastructure (e.g., overland rail) to mitigate vulnerability.

\paragraph{Nonlinear severity and saturation.} The partial degradation analysis reveals that the relationship between disruption severity and cost increase is nonlinear for most chokepoints. For through-corridor chokepoints, the severity curve accelerates at high degradation levels ($\alpha \geq 5$) as rerouting becomes the dominant response. Gibraltar and the Suez Canal exhibit the steepest severity curves, producing mean cost increases exceeding 26\% and 12\% respectively at $\alpha = 5$, reflecting their role as bottlenecks for Mediterranean trade. For the Strait of Malacca, the severity curve is more moderate because the Lombok Strait absorbs rerouted traffic at relatively lower additional cost. This nonlinearity has implications for security investment: the marginal return to reducing risk is highest at moderate disruption levels, suggesting that maintaining a credible deterrent (even short of complete security) captures most of the available dividend.

\paragraph{Supply chain resilience.} Our analysis identifies specific port pairs and corridors that are most vulnerable to disruption, providing a basis for supply-chain risk assessment. The Panama Canal produces the largest mean rerouting cost (17,840 km-equivalent) when closed, reflecting the enormous detour required for Atlantic--Pacific traffic. The Suez Canal, as the primary corridor for Asia--Europe trade, produces a mean rerouting cost of 9,949~km when closed, forcing traffic via the Cape of Good Hope or through the Panama Canal across the Pacific. The Bosporus produces an analogous pattern for Black Sea ports, which must shift to the Constanta--Piraeus overland rail bypass at five times normal cost, with a mean rerouting cost of 3,275~km-equivalent. These findings are consistent with the real-world disruptions documented by \citet{cariou2021ais} during the Ever Given incident and by \citet{verschuur2023systemic} for broader chokepoint disruption scenarios.

\paragraph{Infrastructure and policy responses.} The structural patterns suggest several policy responses, building on the optimal spatial policy framework of \citet{fajgelbaum2020spatial, fajgelbaum2025optimal}. For through-corridor chokepoints, investments that expand capacity or reduce transit times (canal widening, traffic management systems) can lower baseline costs and reduce the severity of partial disruptions. For bypass-dependent chokepoints, investments in alternative infrastructure---overland rail connections, alternative port development---can reduce the cost multiplier of bypass routes, lowering the vulnerability of dependent ports. The geographic sorting dynamics studied by \citet{desmet2015geography} and the agglomeration mechanisms of \citet{duranton2004micro} suggest that such investments may also affect the long-run spatial distribution of economic activity, as firms and workers adjust their location decisions in response to changed trade-cost structures.

\subsection{Limitations}

We are explicit about several limitations that constrain the interpretation of our results, while noting that each suggests a concrete direction for future work.

\paragraph{Static aggregate density.} The AIS data aggregate vessel positions from January 2015 to February 2021 into a single snapshot. We cannot track changes in shipping patterns over time, seasonal variations, or the dynamic adjustment to disruptions. A dataset with temporal disaggregation would enable analysis of how trade routes evolve in response to shocks \citep{cariou2021ais, du2020port}.

\paragraph{Aggregate density versus vessel-level flows.} The density raster records total AIS positions without distinguishing vessel type, flag, cargo, or voyage. We cannot weight traffic by economic value or estimate bilateral trade flows. Extending the analysis to vessel-level AIS data would enable construction of origin-destination trade matrices and support estimation of the full spatial equilibrium model \citep{fajgelbaum2020optimal, allen2022welfare, kerbl2022ais}.

\paragraph{No welfare computation.} Without bilateral trade volumes and demand elasticities, we cannot compute welfare changes in the general-equilibrium sense of \citet{redding2017quantitative} or \citet{allen2025quantitative}. Our delta-cost measures are informative about the direction and relative magnitude of disruption impacts but are not welfare estimates. Connecting our scenario engine to a gravity-based trade model would close this gap.

\paragraph{Exogenous security parameters.} The security risk analysis treats risk premiums as exogenous inputs. In reality, security provision is endogenous to the strategic environment: naval deployment decisions respond to threat assessments, which in turn depend on shipping patterns and insurance pricing \citep{besley2015welfare, kraska2015law}. Endogenizing security provision would require a game-theoretic extension in which a security provider allocates resources across chokepoints to minimize expected trade-cost losses \citep{rossihansberg2004optimal, fajgelbaum2020spatial}.

\subsection{Directions for Future Research}

\begin{enumerate}[nosep]
  \item \textbf{Vessel-level AIS trajectories.} Access to individual vessel tracks would enable construction of origin-destination trade flow matrices and support estimation of route-choice models with probabilistic assignment \citep{fajgelbaum2020optimal, fajgelbaum2020supplement}.

  \item \textbf{Bilateral maritime trade data.} Combining vessel-level AIS with customs or port-authority trade data would enable calibration of a full spatial equilibrium model with endogenous demand \citep{redding2017quantitative, allen2025quantitative, redding2025urban}.

  \item \textbf{Insurance and risk pricing data.} War-risk insurance premiums and piracy incidence records would provide empirically grounded values for the security parameters \citep{besley2015welfare, oef2010economic}.

  \item \textbf{Naval presence proxies.} Open-source data on naval deployments and patrol schedules would enable causal estimation of the security provision effect \citep{proost2019what}.

  \item \textbf{Dynamic extensions.} A time-varying network with stochastic disruptions and adjustment costs would capture the temporal dimension of chokepoint risk, including the option value of maintaining alternative routes \citep{hierons2024spreading, bordeu2025commuting}. Such a model could incorporate the geographic sorting dynamics of \citet{desmet2015geography} and the agglomeration mechanisms of \citet{duranton2004micro}.

  \item \textbf{Optimal security allocation.} A normative extension could solve for the optimal spatial allocation of security resources, building on \citet{fajgelbaum2020spatial, fajgelbaum2025optimal} and \citet{gaubert2025place}. This would address: given a fixed security budget, which chokepoints should receive the most protection?

  \item \textbf{Political economy of burden-sharing.} The observation that maritime security is a global public good with concentrated provision raises questions about sustainability and equity. An extension incorporating political-economy considerations \citep{fajgelbaum2019state, bordeu2025commuting} could analyze the efficiency and distributional consequences of alternative burden-sharing regimes.
\end{enumerate}

\subsection{Conclusion}

Maritime security is a global public good whose economic value has been under-studied relative to its importance. This paper contributes a transparent, replicable framework for measuring chokepoint vulnerability and quantifying the hegemonic dividend from security provision. Using AIS density data from the IMF, we constructed a 78-node maritime transport network with 1,540 port-to-port trade routes and implemented a multi-scenario stress-testing engine.

Three main findings emerge. First, chokepoint disruptions produce large and heterogeneous trade-cost increases: the Panama Canal (mean rerouting cost of 17,840~km-equivalent), Gibraltar (10,624~km), and the Suez Canal (9,949~km) generate the largest rerouting costs when fully closed, while the bypass-dependent Bosporus forces Black Sea ports onto an expensive overland alternative at 5$\times$ normal maritime cost. Second, the marginal value of security provision is increasing in the risk environment: the hegemonic dividend is largest precisely when security conditions are deteriorating. Third, the vulnerability is highly heterogeneous across ports, with bypass-dependent ports and corridor-dependent developing-region ports benefiting most from the global public good of maritime security. Our cost-per-day estimates---ranging from \$16 million to \$233 million per day depending on the chokepoint---provide a concrete, policy-relevant metric for evaluating the economic value of maritime security provision.

These findings are offered as a data-grounded first step that identifies the data requirements and modeling extensions needed for a full causal assessment of the economic value of maritime security. The orders of magnitude---rerouting costs translating to tens of billions of dollars annually, with the most affected port pairs facing cost increases of 10,000--18,000~km-equivalent---suggest that the economic stakes of maritime security provision are substantial, and that the hegemonic dividend from concentrated security at critical chokepoints is a first-order feature of the global trading system \citep{bueger2024securing, rossihansberg2023cognitive, owens2020rethinking}.
