

This literature review synthesizes scholarly work across five domains that jointly motivate and inform the present analysis: maritime security as a global public good; the economics of trade frictions, risk, and piracy; quantitative spatial models of trade and geography; transportation network models with congestion; and emerging data sources in maritime economics. We also draw on the broader spatial economics literature on optimal policy, agglomeration, and redistribution to frame the welfare implications of security provision.

\subsection{Maritime Security as a Global Public Good}

The concept of maritime security as a global public good is central to international relations and security studies. \citet{bueger2019maritime} argue for a holistic understanding of maritime security that links national, environmental, economic, and human dimensions, establishing that the ``politics of the global sea'' are fundamentally under-theorized relative to their economic importance. In a comprehensive assessment, \citet{bueger2024securing} emphasize that global trade, energy security, and food security all depend on safe and secure oceans, and that the provision of this security exhibits classic public-good characteristics: non-rivalry and partial non-excludability.

The legal foundations of maritime security are grounded in the international law of the sea. \citet{kraska2015law} provides a detailed treatment of the legal framework governing freedom of navigation, passage through straits, and the rights and obligations of naval forces in international waters. This framework underpins the legitimacy of forward naval presence as a security provider and defines the legal space within which chokepoint governance operates.

From a defense economics perspective, military spending on maritime security can be understood as an investment in institutional infrastructure for the global economy. The economic cost of maritime insecurity has been documented in several contexts. \citet{mbekeani2011piracy} estimate the direct economic impacts of piracy for African development, including higher insurance costs, rerouting expenses, and deterrence of investment. The One Earth Future Foundation's report on the economic cost of piracy \citep{oef2010economic} provides a comprehensive accounting of piracy-related costs including ransoms, insurance, naval patrols, and supply-chain disruptions, estimating annual global costs in the billions of dollars.

\subsection{Trade Frictions, Risk, and Maritime Disruptions}

The project's focus on trade frictions arising from security risk connects to a broad literature on the determinants and consequences of trade costs. \citet{besley2015welfare} provide a landmark study of the welfare costs of lawlessness in the context of Somali piracy, showing that piracy acts as a significant cost shifter for international shipping routes. Their identification strategy exploits the spatial and temporal variation in piracy risk to estimate the impact on shipping costs and trade flows, establishing a direct empirical link between maritime insecurity and economic outcomes.

Disruptions at maritime chokepoints represent a particularly consequential form of trade friction because they affect not individual routes but the entire network structure. \citet{verschuur2023systemic} analyze the systemic economic impacts of disruptions at 24 major maritime chokepoints and estimate potentially large global losses, with some scenarios involving GDP impacts of several percentage points for vulnerable economies. Their findings provide strong motivation for a scenario-based modeling approach and establish that chokepoint disruptions propagate through the global economy via complex network effects.

The port congestion literature provides relevant evidence on how capacity constraints translate into economic costs. \citet{notteboom2006impact} analyzes the relationship between port congestion and freight rates, demonstrating that congestion at key nodes produces cost increases that ripple through the shipping industry. \citet{du2020port} offer a broader assessment of port congestion's economic impacts and policy responses, documenting the strategies that ports employ to manage delays and their effectiveness.

\subsection{Quantitative Spatial Economics}

The methodological backbone of this paper is the quantitative spatial economics framework, which combines theoretical structure with empirical tractability to evaluate the general-equilibrium effects of infrastructure, policy, and shocks. \citet{redding2017quantitative} provide the foundational survey of this field, establishing the canonical framework in which locations differ in productivity and amenities, trade costs govern the flow of goods across space, and equilibrium outcomes depend on the entire spatial distribution of economic activity. This framework has become the workhorse for evaluating transportation, trade, and place-based policies.

\citet{allen2025quantitative} extend this framework to regional analysis, developing tools for studying how productivity and policy changes propagate across space through trade and migration linkages. \citet{redding2025urban} focuses on the urban dimension, showing how the same quantitative tools can be applied to understand within-city spatial structure and commuting patterns. Together, these surveys establish the methodological vocabulary---iceberg trade costs, gravity-based trade flows, hat-algebra counterfactuals---that we adapt to the maritime setting.

The broader spatial economics literature provides essential context for understanding why location, agglomeration, and connectivity matter. \citet{duranton2004micro} develop the micro-foundations of urban agglomeration economies, identifying the sharing, matching, and learning mechanisms that generate productivity gains from spatial concentration. \citet{desmet2015geography} study the geography of development within countries, showing that spatial variation in productivity and access to markets drives large differences in living standards even within national borders. \citet{proost2019what} offer a synthetic assessment of what can be learned from spatial economics, emphasizing the field's ability to evaluate policies with spatial dimensions---a category that naturally includes maritime security.

\subsection{Transportation Networks and Congestion}

The core analytical engine of our paper draws on the transportation networks literature within spatial economics. \citet{fajgelbaum2020optimal} develop the foundational framework for analyzing optimal transport networks in spatial equilibrium, in which a planner allocates infrastructure investment across edges of a network to maximize welfare, subject to the constraint that trade flows and route choices are endogenous to the resulting cost structure. Their model demonstrates that optimal networks concentrate investment on high-traffic corridors---a finding with direct parallels to the concentration of maritime activity at chokepoints. The online supplement \citep{fajgelbaum2020supplement} provides the computational details that inform our own implementation.

\citet{allen2022welfare} develop a quantitative spatial model with endogenous traffic congestion and apply it to the U.S. highway network. Their framework---in which edge costs are increasing in traffic through a congestion function---provides a close methodological parallel to our maritime cost specification, where chokepoint edge costs depend on shipping intensity. The welfare effects they estimate from highway improvements establish the empirical relevance of congestion-based cost functions for infrastructure evaluation.

\citet{donaldson2025transport} offers a comprehensive survey of transport infrastructure evaluation methods, providing a template for applied policy analysis that our paper follows. His emphasis on distinguishing between identified causal effects and model-based counterfactuals is directly relevant to our approach, which is transparent about operating primarily in the counterfactual simulation domain.

\citet{hierons2024spreading} studies optimal congestion pricing in general equilibrium, developing a framework in which a planner sets tolls on congested edges to internalize externalities. This work provides guidance for the ``optimal security allocation'' extension of our model, in which security resources are allocated across bottlenecks to maximize the reduction in expected trade costs. \citet{bordeu2025commuting} examines infrastructure investment under political fragmentation, highlighting coordination problems that arise when multiple jurisdictions share a common network---an analogy to the provision of global maritime security when international cooperation is incomplete and a single hegemon bears a disproportionate share of the cost.

\subsection{Spatial Policy, Redistribution, and Welfare}

The welfare implications of maritime security connect to the broader literature on optimal spatial policies. \citet{fajgelbaum2020spatial} develop the theory of optimal spatial policies in the presence of geographic sorting, showing how place-based interventions interact with spatial equilibrium to produce welfare gains or losses. \citet{fajgelbaum2025optimal} extend this analysis in a handbook chapter that surveys the frontier of optimal spatial policy design. \citet{fajgelbaum2019state} study how state-level tax differences create spatial misallocation in the United States, establishing that policy-induced distortions to location choices can have first-order welfare effects---a finding that extends naturally to maritime corridors where security differentials affect route choice.

\citet{gaubert2025place} analyze place-based redistribution, developing tools for evaluating the equity and efficiency implications of spatially targeted policies. \citet{owens2020rethinking} apply spatial economic tools to the case of Detroit, demonstrating how place-based interventions can be evaluated within a quantitative spatial framework. \citet{rossihansberg2004optimal} studies optimal urban land use and zoning, providing foundational theory for how spatial planners should allocate resources across locations---an intellectual ancestor of the ``optimal security allocation'' problem we pose. \citet{rossihansberg2023cognitive} examine how the spatial distribution of cognitive activity (``cognitive hubs'') affects the case for spatial redistribution, establishing that agglomeration externalities create a role for policy even in market economies.

\citet{rossihansberg2020transportation} provides a compact analytical treatment of transportation network models that highlights the key features delivering tractability: sparse network structure, congestion-based costs, and shortest-path or logit routing. These features are directly incorporated into our maritime network specification.

\subsection{AIS Data and Measurement in Maritime Economics}

The empirical implementation of this paper relies on Automatic Identification System (AIS) data, which have become a cornerstone of modern maritime economics research. \citet{kerbl2022ais} provides an overview of the use of AIS data in economic research, documenting the range of applications from trade measurement to congestion analysis and highlighting both the opportunities and the challenges (coverage gaps, vessel identification, large data volumes) that researchers face.

\citet{cariou2021ais} use AIS data to study the economic impact of the COVID-19 shock on container flows and ship waiting times in major ports, illustrating how AIS supports real-time measurement of disruptions and their propagation through the maritime network. \citet{cerdeiro2020nowcasting} combine AIS data with machine learning techniques to nowcast world trade, demonstrating the predictive content of shipping movements for macroeconomic outcomes. Together with the IMF World Seaborne Trade Monitoring System \citep{cerdeiro2020wstms} that provides our primary dataset, these papers establish the feasibility of constructing key model inputs---traffic flows, congestion proxies, and activity intensity measures---from AIS data.

\citet{morabito2023thesis} provides a comprehensive treatment of maritime transport economics in a doctoral thesis that covers market structure, pricing, and the economics of port competition---offering institutional detail that contextualizes our network model within the operational realities of the shipping industry.

\subsection{Summary and Gap}

This literature review reveals a clear foundation for modeling maritime security as a public good with measurable economic value. The quantitative spatial economics toolkit \citep{redding2017quantitative, allen2025quantitative} provides the theoretical structure; transportation network models \citep{fajgelbaum2020optimal, allen2022welfare} supply the analytical engine; and AIS data \citep{kerbl2022ais, cerdeiro2020wstms} enable empirical measurement. Yet a gap remains: existing work rarely combines a global maritime network model with endogenous rerouting and congestion to value security provision under counterfactual disruptions. The chokepoint vulnerability literature \citep{verschuur2023systemic} establishes the stakes but typically relies on reduced-form or input-output methods rather than network-equilibrium counterfactuals. The piracy literature \citep{besley2015welfare, oef2010economic} quantifies costs but treats routes as fixed rather than endogenous. Our paper aims to bridge this gap by combining insights from the transportation networks tradition with AIS-based measurement and a flexible scenario framework that accommodates the full range of bottleneck disruptions relevant to maritime security policy.