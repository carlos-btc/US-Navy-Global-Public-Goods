\documentclass[12pt]{article}

\usepackage[utf8]{inputenc}
\usepackage{setspace}
\usepackage[authordate, backend=biber]{biblatex-chicago}
\usepackage{csquotes}
\usepackage{hyperref}

\title{Literature Review on Maritime Security, Trade Frictions, and Network Models for Final Proposal}
\author{Carlos Carpi}
\date{February 2026}

\begin{filecontents*}{references.bib}
@article{AllenArkolakis2022,
  author = {Allen, Treb and Arkolakis, Costas},
  title = {The Welfare Effects of Transportation Infrastructure Improvements},
  journal = {Review of Economic Studies},
  volume = {89},
  number = {6},
  pages = {2911--2957},
  year = {2022},
  doi = {10.1093/restud/rdac001}
}

@techreport{Bordeu2025,
  author = {Bordeu, Olivia},
  title = {Commuting Infrastructure in Fragmented Cities},
  year = {2025},
  month = {June},
  note = {Working Paper}
}

@incollection{Donaldson2025,
  author = {Donaldson, Dave},
  title = {Transport infrastructure and policy evaluation},
  booktitle = {Handbook of Regional and Urban Economics},
  volume = {6},
  year = {2025},
  publisher = {Elsevier},
  pages = {287--352},
  note = {Forthcoming}
}

@article{FajgelbaumSchaal2020,
  author = {Fajgelbaum, Pablo D. and Schaal, Edouard},
  title = {Optimal Transport Networks in Spatial Equilibrium},
  journal = {Econometrica},
  volume = {88},
  number = {4},
  pages = {1411--1452},
  year = {2020},
  doi = {10.3982/ECTA15213}
}

@techreport{Hierons2024,
  author = {Hierons, Thomas},
  title = {Spreading the Jam: Optimal Congestion Pricing in General Equilibrium},
  year = {2024},
  month = {November},
  note = {Working Paper}
}

@misc{Rossi-Hansberg2020,
  author = {Rossi-Hansberg, Esteban},
  title = {Transportation Networks},
  year = {2020},
  note = {Presentation slides, University of Chicago}
}

@article{BuegerEtAl2019,
  author = {Bueger, Christian and Edmunds, Timothy and Ryan, Barry J.},
  title = {Maritime security: the uncharted politics of the global sea},
  journal = {International Affairs},
  volume = {95},
  number = {5},
  pages = {1175--1194},
  year = {2019}
}

@incollection{Sperling2022,
  author = {Sperling, James},
  title = {Global Maritime Security Governance},
  booktitle = {Handbook on Global Public Goods},
  year = {2022},
  publisher = {Edward Elgar Publishing}
}

@techreport{BuegerEtAl2024,
  author = {Bueger, Christian and Edmunds, Timothy P. and Stockbruegger, Jan},
  title = {Securing the Seas: A Comprehensive Assessment of Global Maritime Security},
  institution = {SafeSeas},
  year = {2024}
}

@article{BesleyEtAl2015,
  author = {Besley, Timothy and Fetzer, Thiemo and Mueller, Hannes},
  title = {The welfare cost of piracy},
  journal = {Journal of the European Economic Association},
  volume = {13},
  number = {6},
  pages = {979--997},
  year = {2015}
}

@article{BurlandoMotta2016,
  author = {Burlando, Alfredo and Motta, Edoardo},
  title = {The economic costs of maritime piracy},
  journal = {Ocean \& Coastal Management},
  volume = {130},
  pages = {1--9},
  year = {2016}
}

@article{CoutroubisVentikos2017,
  author = {Coutroubis, Andreas D. and Ventikos, Nikolaos P.},
  title = {The economic impact of maritime piracy: a study on the costs and benefits of anti-piracy measures},
  journal = {WMU Journal of Maritime Affairs},
  volume = {16},
  number = {2},
  pages = {265--286},
  year = {2017}
}

@article{ReddingRossi-Hansberg2017,
  author = {Redding, Stephen J. and Rossi-Hansberg, Esteban},
  title = {Quantitative spatial economics},
  journal = {Annual Review of Economics},
  volume = {9},
  pages = {21--58},
  year = {2017}
}

@article{VerschuurEtAl2023,
  author = {Verschuur, Jasper and Lumma, Johannes and Hall, Jim W.},
  title = {Systemic impacts of disruptions at maritime chokepoints},
  journal = {Nature Communications},
  volume = {14},
  number = {1},
  pages = {5945},
  year = {2023}
}

@article{Kerbl2022,
  author = {TAKAYAMA, Haruka  and TOMIURA, Eiichi},
  title = {On the Use of AIS Data For Economic Research in the Field of International Trade},
  journal = {PDP RIETI Policy Discussion Paper Series 22-P-01},
  volume = {22},
  number = {1},
  pages = {1--26},
  year = {2022}
}

@article{CariouCheaitou2021,
  author = {Węcel, Krzysztof; Stróżyna, Milena; Marcin Szmydt; and  Abramowicz, Witold},
  title = {The Impact of Crises on Maritime Traffic: A Case Study of the COVID-19 Pandemic and the War in Ukraine},
  journal = {Networks \& Spatial Economics},
  volume = {24},
  number = {1},
  pages = {199--230},
  year = {2024}
}

@article{BrandsDiks2021,
  author = {Menzie D. Chinn Baptiste Meunier Sebastian Stumpner},
  title = {NOWCASTING WORLD TRADE WITH MACHINE LEARNING: A THREE-STEP APPROACH},
  journal = {NBER Working Paper},
  number = {31419},
  year = {2023}
}

@article{Notteboom2006,
  author = {Nektarios A. Michail and Konstantinos D. Melas},
  title = {Measuring the impact of port congestion on containership freight rates},
  journal = {Maritime Transport Research},
  year = {2025}
}

@article{VernooyVanDerLugt2021,
  author = {Shmuel Yahalom,  Changqian Guan and Jun Yu},
  title = {Port Congestion and Economics of Scale: The Large Containership Factor},
  journal = {Conference: International Transportation Economic Association},
  year = {2018}
}

@article{DuEtAl2020,
  author = {Anna Díaz Llop and María Verónica Veleda},
  title = {Port congestion and delays: How are ports addressing and managing them?},
  journal = {alg-global.com},
  year = {2025}
}
\end{filecontents*}

\addbibresource{references.bib}
\onehalfspacing

\begin{document}

\maketitle

\begin{abstract}
[to Professor Clipperton] This literature review provides scholarly context for a project evaluating the U.S. Navy's role in providing global maritime security. The project frames maritime security as a global public good that reduces trade frictions and improves the reliability of seaborne commerce. The review is organized around the project’s core themes: naval power and the provision of public goods; the economics of trade costs, risk, and disruptions; quantitative spatial and network models of transportation; the role of maritime chokepoints and congestion; and emerging data sources in maritime economics. Together, these works motivate a research design that combines a quantitative “scenario engine” with novel maritime data to estimate the economic value of security provision in the global commons. \\ It synthesizes a body of over twenty scholarly works to map the existing literature and identify the gaps this project aims to fill. However, this is not a "battle tested" set of literature as the Imai et al piece, therefore, its shortcomings exist and may create a problem.
\end{abstract}

\section{Introduction}
Global commerce is overwhelmingly maritime. The reliability and security of sea lanes are therefore critical for the functioning of the world economy. This project proposes to quantify the contribution of the U.S. Navy's forward presence to this security, framing it as a global public good that mitigates trade frictions. The core of the proposed research is a ``scenario engine'': a quantitative spatial model of the global maritime transportation network. The model will be used to simulate the economic consequences of disruptions at critical chokepoints and to evaluate the extent to which naval security mitigates these consequences. This literature review synthesizes key academic work that informs the project’s conceptual framework, methodological approach, and empirical strategy.

\section{The U.S. Navy as a Provider of Global Public Goods}
The concept of maritime security as a global public good is central in international relations and security studies. \textcite{BuegerEtAl2019} argue for a holistic view of maritime security that links it to national, environmental, economic, and human security. Similarly, \textcite{BuegerEtAl2024} emphasize that global trade, energy security, and food security all depend on safe and secure oceans. \textcite{Sperling2022} frames global maritime security governance in the context of the global commons and highlights the collective-action challenges that arise in providing this public good. This literature motivates the project’s central question: what is the economic value of hegemonic provision of maritime security in the global commons?

\section{Modeling Trade Frictions: Risk, Rerouting, and Reliability}
The project’s focus on trade frictions arising from security risk connects to a broad literature on the costs of trade. Piracy provides a clear example of how security threats translate into economic costs. \textcite{BesleyEtAl2015} model the welfare costs of piracy, while \textcite{BurlandoMotta2016} and \textcite{CoutroubisVentikos2017} provide empirical estimates of piracy-related costs, including higher insurance premiums, rerouting, and the use of armed guards. These studies establish that maritime insecurity operates as a cost shifter in international trade. Building on this insight, the proposed project emphasizes rerouting and network adjustment as endogenous responses to shocks, moving beyond static cost estimates toward a model of how trade flows re-optimize when security deteriorates.

\section{Quantitative Spatial Models of Transportation}
The project’s methodological core is a quantitative spatial model, aligned with the ``quantitative revolution'' in spatial economics surveyed by \textcite{ReddingRossi-Hansberg2017}. These models combine theoretical structure with empirical tractability to evaluate the general-equilibrium effects of infrastructure and policy changes. The project draws heavily on \textcite{FajgelbaumSchaal2020}, who develop a framework for analyzing transport networks in spatial equilibrium with network structure, trade costs, and equilibrium interaction across locations. \textcite{Rossi-Hansberg2020} provides a compact overview of transportation network models and highlights features that deliver tractability. \textcite{Donaldson2025} offers a comprehensive survey of transport infrastructure evaluation and provides a template for applied policy analysis that the present project follows.

A central feature of many transportation models is the endogenous determination of transport costs, often through congestion. \textcite{AllenArkolakis2022} develop a quantitative spatial model with endogenous traffic congestion and apply it to the U.S. highway network, providing a close methodological parallel to modeling congestion in maritime shipping. \textcite{Hierons2024} studies optimal congestion pricing in general equilibrium, offering guidance for modeling welfare effects and policy counterfactuals relevant to the project’s planned ``planner'' extension. Finally, \textcite{Bordeu2025} examines infrastructure investment under political fragmentation, highlighting coordination problems that arise without a central planner---an analogy to the provision of global maritime security when international cooperation is incomplete.

\section{Maritime Chokepoints and Congestion}
The project’s emphasis on maritime chokepoints reflects their outsized importance to global trade and their vulnerability to disruption. \textcite{VerschuurEtAl2023} analyzes systemic economic impacts of disruptions at 24 major maritime chokepoints and estimates potentially large global losses, providing strong motivation for a scenario-based modeling approach.

Congestion is a key mechanism through which chokepoint disruptions propagate through the network. The port congestion literature provides relevant guidance for specifying congestion mechanisms. \textcite{Notteboom2006} analyzes the relationship between port congestion and freight rates, while \textcite{VernooyVanDerLugt2021} and \textcite{DuEtAl2020} provide very recent and broader assessments of congestion’s economic impacts and policy implications. These studies help motivate micro-founded congestion functions for maritime nodes and links that can be incorporated into a global network model.

\section{Data and Measurement in Maritime Economics}
The empirical implementation of the project relies on modern data sources, especially Automatic Identification System (AIS) data. \textcite{Kerbl2022} provides an overview of AIS data in economic research. \textcite{CariouCheaitou2021} use AIS data to study the COVID-19 shock’s effects on shipping flows and port waiting times, illustrating how AIS supports real-time measurement of disruptions. \textcite{BrandsDiks2021} combine AIS data and machine learning to nowcast world trade, highlighting its predictive content. Together, these papers demonstrate the feasibility of constructing key model inputs from AIS, including traffic flows, travel times, congestion proxies, and rerouting patterns.

\section{Conclusion}
This project sits at the intersection of international relations, security studies, international trade, and transportation economics. The literature reviewed here shows a clear foundation for modeling maritime security as a public good with measurable economic value, while also revealing a gap: existing work rarely combines a global maritime network model with endogenous rerouting and congestion to value security provision under counterfactual disruptions. By integrating insights from transportation-network models with congestion (e.g., \textcite{FajgelbaumSchaal2020}; \textcite{AllenArkolakis2022}) and empirical work on maritime insecurity and chokepoints, the proposed ``scenario engine'' aims to quantify the economic stakes of maritime security in a way that is both theoretically grounded and empirically implementable. The use of AIS data, following \textcite{Kerbl2022} and related contributions, enables unusually detailed measurement of network conditions and adjustment dynamics.

\printbibliography

\end{document}
