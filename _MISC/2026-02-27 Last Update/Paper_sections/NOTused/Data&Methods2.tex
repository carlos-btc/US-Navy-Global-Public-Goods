
\documentclass{article}
\usepackage{geometry}
\usepackage{amsmath}
\usepackage{amssymb}
\usepackage{hyperref}
\usepackage{natbib}

% Formatting for readability in draft
\geometry{margin=1in}


\title{Data & Methods:}
\author{Carlos Carpi}
\date{2026-01-29}

\begin{document}
	
	

\section{Preamble to Data \& Methods}

\subsection{For Professor Clipperton's consumption: Track 2 Design overview and why ``bottleneck scenarios''}
	https://www.overleaf.com/read/jgsvsjqsrsyn#a5ef30 \\
	\\
	This section is under Data&Methods2.tex in the project. \\ \\
This proposal evaluates the U.S.\ Navy as a provider of global public goods by quantifying how maritime security and reliability reduce trade frictions. Incorporating Professor Rossi-Hansberg's feedback , the core methodological emphasis shifts from attempting to directly ``read off'' the causal effect of naval operations from partially observed deployments toward a \emph{scenario-based quantitative spatial model} of global maritime transportation.

The motivating idea is that global shipping operates on a network with well-known physical bottlenecks (straits, canals, narrow sea lanes, port approaches). When a bottleneck becomes impaired (closure, capacity reduction, elevated risk, congestion), trade flows reroute endogenously. A credible empirical design must therefore (i) model \emph{route substitution} explicitly, and (ii) be robust to counterfactuals in which the same trade flows face different feasible paths. The paper will therefore build a tractable network model of maritime transportation costs and flows, calibrate it to observed traffic and trade patterns, and run scenario analyses that impose bottleneck shocks. The ``hegemonic dividend'' of U.S.\ naval presence is then measured as the reduction in the welfare/trade-cost losses under these scenarios when security provision is higher (or optimally allocated), relative to a lower-security baseline.

This approach is directly inspired by quantitative spatial network frameworks that (a) represent locations as nodes on a graph, (b) represent transportation costs on edges as iceberg costs that depend on traffic (congestion) and ``infrastructure quality,'' and (c) solve for equilibrium flows and counterfactuals with strong computational tractability. \\

That said, there are still major issues in this section thanks to the problem with citations. To fix this issue, [ADD CITATION] shows up as a marker to allow for ease of future problem solving and generally most places have them only in footnotes to avoid cluttering the text. That said, where appropriate, there is also an explanatory note of where and why this or that reference is being cited. Finally, at no point should ANYTHING in this section be construed as my OWN PERSONAL "NEW" idea, outside of the construction of the whole. i.e. this output should not be taken as my "original" work sent in for a grade because I forgot a citation - for example, maybe, and this is likely a hypothetical that could ring a bit too true, I may forget or have technical issues citing sources for a particular conclusion, that does not mean that I did not intend to properly cite the actual authors. At no point was there any expectation that the citation would not be actually done or any attempt to deceive the reader to pass other people's work as my own. The technical problems with citations, however, created a problem. The reader has been warned. That said, any mistakes are completely my fault. 
\section{Data \& Methods}
\subsection{Empirical strategy: bottleneck scenario analysis on a maritime transport network }
The central challenge in quantifying the U.S.\ Navy as a provider of global public goods\footnote{[ADD CITATION]}  is that the most policy-relevant outcomes (trade reliability, routing resilience, and avoided trade-cost spikes) are \emph{equilibrium objects} of a transportation network. When a chokepoint is disrupted---through conflict risk, congestion, or physical closure---global shipments re-route. Any credible evaluation must therefore model \emph{trade-flow substitution across routes} rather than treat observed routes as fixed.

Following the ``transportation networks'' framework,\footnote{This is based on slide deck used by Professor Rossi-Hansberg for Week 4 in the class ECON 33550 Spatial Economics of Winter 2026. A complete citation will be created eventually [ADD CITATION]} the paper will build a quantitative spatial model of global maritime shipping on a graph in which: (i) shipments choose routes through the network; (ii) link-level costs depend on traffic (congestion) and on link quality/capacity; and (iii) counterfactuals are conducted by shocking bottleneck links and recomputing equilibrium flows and delivered trade costs. The Navy's global public good is introduced as a \emph{cost-reducing and risk-reducing shifter} on bottleneck-adjacent links: higher security provision lowers effective iceberg trade costs and improves reliability, which becomes especially valuable in disruption scenarios.

\paragraph{Main deliverables.}
The final output is a ``scenario engine'' that maps a set of bottleneck disruptions into:
(i) predicted rerouting patterns and congestion spillovers,
(ii) changes in delivered trade costs for major origin--destination (OD) pairs,
(iii) first-order and model-based welfare stakes, and
(iv) a ``marginal value of security'' ranking across bottlenecks/links.


\subsection{Data}
\subsubsection{Core datasets (public or commercially licensable)}
The analysis requires four classes of data. All will be harmonized to a common spatiotemporal panel (monthly baseline; weekly where feasible).

\subsubsection{Harmonization}
The analysis uses a link-month panel as the primary unit of observation. All datasets are harmonized to WGS84 coordinates \footnote{ For this choice reasoning see for example: https://www.usgs.gov/centers/eros/why-does-annual-nlcd-use-wgs84-its-datum} and aggregated to a common monthly calendar. Robustness uses higher frequency (weekly) around acute disruption episodes.

\paragraph{(1) Maritime traffic (AIS-based).}
Automatic Identification System (AIS) data provide ship positions, speed, and timestamps. These are used to (i) construct empirical shipping corridors, (ii) measure traffic intensity by corridor segment, and (iii) validate route choice and congestion mechanisms. Put simply, AIS position reports provide vessel trajectories. The project will construct link-level traffic measures and travel-time measures by:
\begin{enumerate}
	\item Filtering AIS to cargo-relevant vessel classes (container, tanker, bulk carrier; robustness: include all large commercial).
	\item Map-matching vessel trajectories to a discretized sea-lane graph (Section \ref{subsec:graph}).
	\item Aggregating to link-month traffic: $\Xi_{e,t}$ = number of transits (or deadweight tonnage-weighted transits) on edge $e$ in month $t$.
	\item Computing realized link travel times from timestamped entries/exits and converting to an empirical ``speed'' measure for congestion estimation (Section \ref{subsec:estimation}).
\end{enumerate}



\paragraph{Constructed variables (monthly baseline).}
For each directed link $e$ and month $t$:
\begin{itemize}
	\item Traffic: $\Xi_{e,t}$ = number of transits, or tonnage-weighted transits.
	\item Travel time (proxy for generalized cost): $\text{TT}_{e,t}$ from entry/exit timestamps.
	\item Reliability: $\text{Var}(\text{TT})_{e,t}$ or tail travel-time percentiles.
\end{itemize}

\textit{Output artifacts:} (i) a link-month panel of $\Xi_{e,t}$ and travel time; (ii) OD matrices of vessel movements between ports/regions.



\paragraph{(2) Trade and economic mass.}
To discipline origin-destination demand for shipping services and measure economic stakes:
\begin{itemize}
	\item Bilateral trade flows by country/port-region (value and, when available, weight/volume) at monthly or annual frequency.
	\item Port throughput and connectivity indices (container throughput, liner connectivity proxies).
	\item Country or region-level GDP and sectoral composition (for welfare calculations and heterogeneity).
\end{itemize}

\paragraph{(3) Bottlenecks and geography.}
We construct a geospatial layer of maritime bottlenecks (e.g., canals, straits, and approach lanes). Each bottleneck $b$ is mapped to an \emph{edge set} $E(b)$ on the maritime graph that captures the links whose impairment constitutes disruption of $b$. In short, we will compile a geospatial layer of maritime bottlenecks:
\begin{itemize}
	\item Polygons/lines for major straits/canals and associated approach lanes.
	\item ``Edge sets'' $E(b)$ mapping each bottleneck $b$ to the subset of graph edges that constitute that bottleneck.
\end{itemize}
This layer is used both for scenario shocks (closing or degrading $E(b)$) and for treatment exposure (naval presence near $b$).

\paragraph{(4) Security and risk proxies.}
Because classified operational detail may be unavailable, security provision will be proxied using:
\begin{itemize}
	\item Publicly observable U.S.\ naval forward presence measures (base locations; port calls; major exercises; publicly reported patrols; FONOP events).
	\item Piracy and maritime security incidents (counts and severity in spatial bins).
	\item Optional (if accessible): war-risk insurance indicators or shipping risk indices.
\end{itemize}
We will construct a link-month ``security intensity'' index $S_{e,t}$ using distance-weighted exposure to bases/port calls/patrol-relevant events and/or presence in designated operational areas.

\subsubsection{Units of observation and harmonization}
\begin{itemize}
	\item \textbf{Spatial unit:} directed edges $e=(k \rightarrow \ell)$ on a maritime graph (Section \ref{subsec:graph}), plus a mapping from edges to bottlenecks. 
	\item \textbf{Temporal unit:} month $t$ (baseline). Higher frequency (weekly/daily) will be used for select event windows (robustness checks).
\end{itemize}
All datasets are projected to WGS84 coordinates and joined by spatial overlay (edge geometries, bottleneck polygons, risk-event points) and by time (month).

\subsection{Constructing the maritime transportation graph}
\label{subsec:graph}
Following network-based spatial models, the world is represented as a weighted directed graph $G=(J,E)$:
\begin{itemize}
	\item \textbf{Nodes $J$:} ports (UN/LOCODE \footnote{See https://unece.org/trade/cefact/unlocode-code-list-country-and-territory see also https://unece.org/trade/uncefact/unlocode ("Currently, UN/LOCODE includes over 103,034 locations in 249 countries and territories.")} or similar), plus ``routing waypoints'' that discretize open-ocean corridors. A coarse version uses only major ports and chokepoints; a richer version uses a raster grid of ocean cells with adjacency links.
	\item \textbf{Edges $E$:} feasible shipping links between neighboring nodes. Edge geometry follows known sea lanes where possible; otherwise edges connect adjacent ocean cells that are navigable.
\end{itemize}
We will build two nested graphs:
\begin{enumerate}
	\item \textbf{Chokepoint graph (low-dimensional):} nodes are major ports and bottlenecks; edges are typical corridor segments. Used for transparent scenario work and fast iteration.
	\item \textbf{Ocean-cell graph (higher-dimensional):} nodes are $\approx$0.5$^\circ$ (or similar) ocean cells; edges connect neighbors (e.g., 8-neighbor adjacency). Used for robustness and richer rerouting.
\end{enumerate}
AIS trajectories validate the graph: edges with negligible observed usage can be pruned; heavily used corridors can be densified.

\subsection{Model: endogenous rerouting under congestion and security}
\subsubsection{Edge costs as iceberg trade costs}
Let $t_{e,t} \ge 1$ denote the iceberg cost of traversing edge $e$ in month $t$ (a multiplicative cost in ``units of the shipped good''). Inspired by congestion-based network models, we parameterize:
\begin{equation}
	t_{e,t} = \bar{t}_{e,t}\,(\Xi_{e,t})^{\lambda},
	\label{eq:congestion}
\end{equation}
where $\Xi_{e,t}$ is traffic on edge $e$ and $\lambda \ge 0$ captures congestion (or more broadly, increasing marginal cost with intensity). The baseline component $\bar{t}_{e,t}$ depends on distance, physical constraints, fees, and security/risk:
\begin{equation}
	\ln \bar{t}_{e,t} = \delta_0 + \delta_1 \ln(\text{dist}_e) + \delta_2\,\text{Hazard}_{e,t} - \delta_3\,S_{e,t} + \delta_4 X_{e} + \mu_t,
	\label{eq:baseline_cost}
\end{equation}
where $S_{e,t}$ is security intensity (higher values reduce costs), $\text{Hazard}_{e,t}$ captures piracy/conflict risk, $X_e$ are time-invariant edge characteristics (e.g.\ narrowness, mandatory traffic separation), and $\mu_t$ are global time effects (fuel prices, macro shocks).

Interpretation: the Navy acts as a ``global public good'' provider by reducing effective trade frictions on critical edges (especially around bottlenecks) through deterrence, deconfliction, and reliability provision.

\subsubsection{Route aggregation and OD trade costs}
For each origin $o$ and destination $d$, shipments can traverse many feasible routes $r \in \mathcal{R}_{od}$. Let a route be a sequence of edges; its iceberg cost is the product of edge costs:
\[
T_{od,r,t} = \prod_{e \in r} t_{e,t}.
\]
To capture substitution across routes in a smooth and computationally convenient way, define the OD trade cost as a ``log-sum'' aggregator:
\begin{equation}
	\tau_{od,t} \equiv
	\left(\sum_{r\in \mathcal{R}_{od}} T_{od,r,t}^{-\theta}\right)^{-1/\theta},
	\label{eq:logsum}
\end{equation}
where $\theta>0$ governs dispersion in route choice. As $\theta \to \infty$, $\tau_{od,t}$ approaches the least-cost (shortest-path) route cost; in practice, this provides a convenient and transparent baseline, with finite-$\theta$ used for robustness.

\subsubsection{From OD demand to link traffic}
Let $X_{od,t}$ be the shipped quantity (or value) from $o$ to $d$ in month $t$. Link traffic is then the sum of OD shipments weighted by the share of each OD flow that traverses each edge. Under the log-sum formulation, the implied intensity (share) with which edge $e=(k\rightarrow \ell)$ is used for an OD pair is:
\begin{equation}
	\pi^{od}_{e,t} =
	\left(\frac{\tau_{od,t}}{\tau_{ok,t}\, t_{e,t}\, \tau_{\ell d,t}}\right)^{\theta},
	\label{eq:edge_share}
\end{equation}
and link traffic is
\begin{equation}
	\Xi_{e,t} = \sum_{o,d} X_{od,t}\,\pi^{od}_{e,t}.
	\label{eq:traffic}
\end{equation}
Equations \eqref{eq:congestion}--\eqref{eq:traffic} define a fixed point: traffic affects edge costs (congestion), which affects route choice and therefore traffic. We will solve this fixed point by iteration using sparse graph algorithms.

\subsubsection{Welfare and ``Navy dividend'' metrics}
The proposal focuses on welfare-relevant and policy-relevant outputs:
\begin{itemize}
	\item \textbf{Trade-cost changes:} $\Delta \ln \tau_{od,t}$ for key OD pairs (countries, port-regions).
	\item \textbf{Rerouting:} changes in $\pi^{od}_{e,t}$ and in bottleneck utilization.
	\item \textbf{Aggregate stakes:} changes in delivered trade costs for a region's import basket, and implied real-income changes in a standard CES-import setting (implemented with transparent assumptions and robustness ranges).
\end{itemize}
Define a scenario $s$ (e.g.\ ``Hormuz capacity cut by 50\%''), and let outcomes under security level $S$ be $Y(s;S)$. The \emph{hegemonic dividend} is:
\[
\text{Dividend}(s) \equiv Y(s;S^{\text{high}}) - Y(s;S^{\text{low}}),
\]
reported for multiple scenarios and multiple outcome metrics (trade-cost losses, rerouting distances, welfare loss).

\subsection{Scenario analysis: maritime bottlenecks as stress tests}
\label{subsec:scenarios}
The central empirical object is the distribution of counterfactual outcomes under disruptions to bottlenecks. Scenarios will be defined as shocks to subsets of edges $E(b)$ associated with bottleneck $b$:
\begin{itemize}
	\item \textbf{Closure:} set $t_{e,t} \rightarrow \infty$ for all $e\in E(b)$.
	\item \textbf{Capacity reduction / congestion amplification:} increase $\lambda$ locally or scale $t_{e,t} \rightarrow m\, t_{e,t}$ for $e\in E(b)$.
	\item \textbf{Risk spike:} increase $\text{Hazard}_{e,t}$ for edges near $b$.
\end{itemize}
We will run a scenario grid for each bottleneck:
\[
(b, \text{severity}, \text{duration}, \text{security regime}),
\]
where \emph{security regime} toggles between observed security intensity, a ``low presence'' counterfactual, and (optionally) an ``optimized presence'' allocation described below.

\paragraph{Optional planner extension (actionable US Navy output).}
Analogous to ``optimal network investment'' in transportation models, we can treat security deployment as a resource-constrained choice variable:
\[
\sum_{e} \kappa_e\, S_{e} \le K,
\]
where $K$ is total ship-days (or an abstract security budget) and $\kappa_e$ is the cost of providing security on edge $e$. \footnote{This is where the integration to US Navy databases and cookbook rather than implementation comes in [CITATION NEEDED]} The model can then compute \emph{marginal welfare elasticities} of improving security on each edge (a ``where is security most valuable?'' ranking), and simulate optimal reallocation under bottleneck stress.

\subsection{Estimation and calibration}
\label{subsec:estimation}
Because the primary goal is a defensible scenario engine, we prioritize parameters that (i) can be disciplined by available data and (ii) matter for rerouting and stakes.

\subsubsection{Step 1: Calibrating OD shipment demands $X_{od,t}$}
We will construct $X_{od,t}$ using trade data mapped to ports/regions:
\begin{enumerate}
	\item Map countries to primary ports/port-regions (baseline: one port-region per country; robustness: multiple).
	\item Convert annual trade to monthly using seasonal patterns from port throughput where available.
	\item Scale OD flows to match aggregate AIS traffic volumes in major corridors (our 'consistency' check).
\end{enumerate}

\subsubsection{Step 2: Estimating congestion $\lambda$ from AIS travel times}
Equation \eqref{eq:congestion} implies that higher traffic raises costs. Using AIS-derived speed/travel time on each edge:
\begin{enumerate}
	\item Compute edge travel time $\text{time}_{e,t}$ and define $\ln t_{e,t}$ as proportional to $\ln \text{time}_{e,t}$ (a standard assumption when costs are linear in time).
	\item Estimate:
	\[
	\ln t_{e,t} = \ln \bar{t}_{e,t} + \lambda \ln \Xi_{e,t} + \varepsilon_{e,t}.
	\]
	\item Address endogeneity of traffic (slow edges attract less traffic). We will test two IV-style strategies adapted from ``network geometry'' intuition: \footnote{See slide deck mentioned before [CITATION NEEDED]}
	\begin{itemize}
		\item \textbf{Mechanical complexity instrument:} edge curvature / mandatory separation schemes / number of heading changes induced by safe routing constraints, conditional on distance.
		\item \textbf{Exogenous shifters:} weather/current anomalies or port-level demand shocks that shift traffic but are plausibly orthogonal to local free-flow conditions after controls (to be validated empirically\footnote{Literature review neeed [CITATION NEEDED]}).
	\end{itemize}
\end{enumerate}

\subsubsection{Step 3: Security as a cost shifter (reduced form + robustness)}
We incorporate security via $S_{e,t}$ in \eqref{eq:baseline_cost}. These are the two complementary approaches:
\begin{itemize}
	\item \textbf{Calibrated range:} choose $\delta_3$ to match observed differences in incident-adjusted speeds/route choices between high- and low-presence corridors, reporting sensitivity.
	\item \textbf{Reduced-form validation:} regress residualized travel-time (or hazard-adjusted risk proxies) on $S_{e,t}$ with rich fixed effects (edge and time), interpreted cautiously as correlational validation rather than definitive causality.
\end{itemize}
The key deliverable is not a single point estimate of $\delta_3$, but a \emph{robust} mapping from plausible security effects to scenario losses avoided.

\subsection{Computation: solving the fixed point and generating counterfactuals}
\subsubsection{Baseline solution algorithm}
For each month $t$:
\begin{enumerate}
	\item Initialize edge costs using $\bar{t}_{e,t}$ from \eqref{eq:baseline_cost}.
	\item Compute OD trade costs $\tau_{od,t}$ using:
	\begin{itemize}
		\item \textbf{Shortest-path} (large $\theta$) for the baseline implementation; and
		\item \textbf{Log-sum} costs \eqref{eq:logsum} for robustness.
	\end{itemize}
	\item Compute edge shares $\pi^{od}_{e,t}$ and update traffic via \eqref{eq:traffic}.
	\item Update edge costs using congestion \eqref{eq:congestion}.
	\item Iterate until $\max_e |\Xi^{(m+1)}_{e,t}-\Xi^{(m)}_{e,t}|$ is below tolerance.
\end{enumerate}
This produces a baseline set of costs, routes, and traffic that is consistent with the model and disciplined by AIS traffic.

\subsubsection{Counterfactual/scenario solution}
For each scenario $s$ (Section \ref{subsec:scenarios}), modify edge costs on affected edges and re-run the fixed-point algorithm. Outputs include:
\begin{itemize}
	\item changes in bottleneck utilization,
	\item changes in average route length/time,
	\item changes in OD trade costs and welfare proxies,
	\item avoided losses under higher security.
\end{itemize}

\subsection{Evaluation: advantages and limitations}
\subsubsection{Advantages}
\begin{itemize}
	\item \textbf{Directly addresses rerouting and general equilibrium in routes.} Bottlenecks matter precisely because flows reallocate. The network approach makes substitution a first-class object rather than an afterthought.
	\item \textbf{Transparent, stress-testable, and Navy-usable.} Scenario outputs (``If bottleneck $b$ is impaired by X\%, how much trade cost increases and where traffic reroutes'') are legible to practitioners and naturally tied to planning.
	\item \textbf{Works with partially observed security data.} Even without granular classified deployment measures, security can enter as a cost shifter with sensitivity analysis; the main inference lever is the physical network plus observed traffic and trade.
	\item \textbf{Produces actionable marginal values.} The planner extension yields ``welfare elasticities'' of improving security on each edge/bottleneck, enabling ROI-like rankings of presence.
\end{itemize}

\subsubsection{Limitations}
\begin{itemize}
	\item \textbf{Parameter uncertainty (especially security effectiveness).} Mitigation: report ranges and threshold analyses (``how large must $\delta_3$ be for security to offset scenario losses of size Y'').
	\item \textbf{Potential mismatch between AIS traffic and trade value.} Mitigation: weight flows by vessel class/deadweight tonnage; validate using port throughput and known trade corridors.
	\item \textbf{Partial equilibrium welfare mapping.} Mitigation: implement a minimal CES import-cost welfare mapping and report results as ``trade-cost equivalent'' plus sensitivity.
	\item \textbf{Endogeneity in estimating congestion.} Mitigation: IV strategies using network geometry and exogenous shifters; also report OLS and bounded estimates as robustness.
\end{itemize}

\subsection{Proof of concept (implementation mock-up)}
\label{subsec:poc}

To demonstrate feasibility immediately (even before assembling full AIS and trade data), we present a cookbook \footnote{As discussed in class with Professor Clipperton.} of a toy network scenario engine that:
(i) constructs a directed graph,
(ii) computes baseline shortest-path routes,
(iii) ``closes'' a bottleneck edge set,
(iv) recomputes rerouting and cost increases.

\paragraph{Minimal Python prototype (toy example).}
The following cookbook code is a minimun viable product type output to verify the rerouting logic; it will be replaced by AIS-calibrated distances/costs once data are assembled.

\begin{verbatim}
	import networkx as nx
	
	# Toy maritime graph: nodes are ports/waypoints; edges have baseline "time cost"
	G = nx.DiGraph()
	edges = [
	("Asia", "Suez", 10), ("Suez", "Europe", 10),
	("Asia", "Cape", 18), ("Cape", "Europe", 18),
	("Asia", "Hormuz", 6), ("Hormuz", "Suez", 8)
	]
	for u, v, w in edges:
	G.add_edge(u, v, weight=w)
	
	def sp_cost(o, d):
	path = nx.shortest_path(G, o, d, weight="weight")
	cost = nx.shortest_path_length(G, o, d, weight="weight")
	return path, cost
	
	print("Baseline:", sp_cost("Asia", "Europe"))
	
	# Scenario: close Suez -> remove associated edges
	G_s = G.copy()
	G_s.remove_node("Suez")  # crude closure
	G = G_s
	
	print("Suez closed:", sp_cost("Asia", "Europe"))
\end{verbatim}

\paragraph{Planned ``real data'' proof of concept:}
Using a small geographic subset (e.g.\ one bottleneck and its alternative path), we will:
\begin{enumerate}
	\item Download AIS sample for a limited window and region.
	\item Build a corridor graph around the bottleneck.
	\item Compute observed traffic $\Xi_{e,t}$ and compare baseline predicted traffic concentration to observed concentration.
	\item Run closure/capacity scenarios and export maps/tables of rerouting.
\end{enumerate}
These outputs would like make plots, maps, and the notebook link. \footnote{We add to the Section \ref{subsec:dmappendix} an alternative to this section's code as simulation instead.}




\subsection{Reproducibility}
All steps will be scripted (Python/R) with:
\begin{itemize}
	\item Data dictionary defining each variable and transformation.
\item Geospatial files for nodes/edges and bottleneck polygons with edge-set mappings $E(b)$.
\item Notebooks for calibration and scenario runs.
\item A single driver script (\texttt{run\_all}) that reproduces baseline and counterfactual outputs from raw inputs.
\end{itemize}





\subsection{Appendix}
\label{subsec:dmappendix}

\subsection{Proof of concept}

To demonstrate feasibility now, we present a minimal network rerouting simulation that (i) constructs a directed graph, (ii) computes baseline least-cost routes, and (iii) imposes a bottleneck closure to quantify rerouting and cost increases.

\paragraph{Alternative Toy example code (replace nodes/edges with AIS-calibrated links).}
\begin{verbatim}
	import networkx as nx
	
	# Nodes: simplified regions/waypoints. Edges: baseline travel-time costs.
	G = nx.DiGraph()
	edges = [
	("Asia", "Suez", 10), ("Suez", "Europe", 10),
	("Asia", "Cape", 18), ("Cape", "Europe", 18),
	("Asia", "Hormuz", 6), ("Hormuz", "Suez", 8)
	]
	for u, v, w in edges:
	G.add_edge(u, v, weight=w)
	
	def shortest_path_cost(G, o, d):
	path = nx.shortest_path(G, o, d, weight="weight")
	cost = nx.shortest_path_length(G, o, d, weight="weight")
	return path, cost
	
	print("Baseline:", shortest_path_cost(G, "Asia", "Europe"))
	
	# Scenario: close Suez (remove node or edges in the bottleneck set)
	G2 = G.copy()
	G2.remove_node("Suez")
	
	print("Suez closed:", shortest_path_cost(G2, "Asia", "Europe"))
\end{verbatim}




\end{document}
% =========================
% End of Data & Methods section
% =========================
